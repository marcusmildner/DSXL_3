\EXERCICE \SeulementModeEnonce{\poids{100}}
\enonce{ %début énoncé 

Soit ABC un triangle rectangle en B. On sait que AB = $\var{a}{3}$ et BC = $\var{b}{4}$

\definecolor{uuuuuu}{rgb}{0.26666666666666666,0.26666666666666666,0.26666666666666666}
\begin{tikzpicture}[line cap=round,line join=round,>=triangle 45,x=1.0cm,y=1.0cm]
\clip(-4.3,-3.02) rectangle (7.28,6.3);
\draw [line width=2.pt] (4.,3.)-- (-2.,0.);
\draw [line width=2.pt] (-2.,0.)-- (4.,0.);
\draw [line width=2.pt] (4.,0.)-- (4.,3.);
\draw [line width=2.pt,domain=-4.3:7.28] plot(\x,{(--24.-6.*\x)/3.});
\begin{scriptsize}
\draw [fill=black] (-2.,0.) circle (2.5pt);
\draw[color=black] (-1.86,0.37) node {$A$};
\draw [fill=black] (4.,0.) circle (2.5pt);
\draw[color=black] (4.14,0.37) node {$B$};
\draw [fill=black] (4.,3.) circle (2.5pt);
\draw[color=black] (4.14,3.37) node {$C$};
\draw[color=black] (1.3,6.15) node {(d)};
\draw [fill=uuuuuu] (2.8,2.4) circle (2.0pt);
\draw[color=uuuuuu] (2.22,2.67) node {$H$};
\end{scriptsize}
\end{tikzpicture}


} % fin énoncé 

\begin{description}
\item[1.] \enonce{ %début énoncé 
Calculer la longueur de AC 
On donnera une valeur approchée au millième. 
} % fin énoncé 

\correction{ %début énoncé 
\poids{40} %en pourcentage 
 Le triangle ABC est rectangle en B, donc avec le théorème de Pythagore on a : 
\begin{eqnarray*}
AB^2 + BC^2 &=& AC^2 \\
\var{a}{3}^2 + \var{b}{4}^2 &=& AC^2 \\
\var{h2}{25} &=&AC^2 \\
& & \mbox{ Donc : } AC = \sqrt{\var{h2}{25}}\approx \var{h}{5.000} 
\end{eqnarray*}
} % fin correction 

\item[2.] \enonce{ %début énoncé 
 Calculer l'aire $S$ du triangle ABC
} % fin énoncé 

\correction{ %début énoncé 
\poids{20} %en pourcentage
Comme le triangle ABC est rectangle en B on a :
\begin{eqnarray*} 
S &=& \frac{AB \times BC}{2} \\
S &=& \frac{\var{a}{3}\times \var{b}{4}}{2} \\
S &=& \var{w}{6} \mbox{ unité d'aire (u.a.)}
\end{eqnarray*}
} % fin correction 

\item[3.] \enonce{ %début énoncé 
 Soit H le pied de la hauteur du triangle ABC issu de B. Exprimer l'aire S en fonction de BH et AC. 
} % fin énoncé 


\correction{ %début énoncé 
\poids{20} %en pourcentage
 On a : $ S = \frac{BH \times AC}{2} $ 
} % fin correction 

\item[4.] \enonce{ %début énoncé 
 Determiner HB (on donnera une valeur arrondie au centième).
} % fin énoncé 

\correction{ %début énoncé 
 \poids{20} %en pourcentage
 On a : 
\begin{eqnarray*} 
S &=& \frac{BH \times AC}{2} \\
\var{w}{6}  &=& \frac{BH \times \var{h}{5.000}}{2} \\
2\times \var{w}{6} &=& BH \times \var{h}{5.000} \\
\frac{2\times \var{w}{6}}{\var{h}{5.000}} & =& BH \\
&& \mbox{Donc : }BH \approx \var{BH}{2.40}
\end{eqnarray*}
} % fin correction 

\end{description}


\SeulementModeEnonce{}




