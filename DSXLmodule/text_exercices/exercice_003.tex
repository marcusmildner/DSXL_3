\EXERCICE 
\SeulementModeEnonce{ \poids{100}} 

\medskip
\begin{minipage}{0.48\linewidth}
On considère la figure ci-contre. On donne les mesures suivantes:

\begin{itemize}
\item[$\bullet~$] AN = $ \var{AN}{13}$ cm
\item[$\bullet~$] LN = $\var{LN}{5}$ cm
\item[$\bullet~$] AL = $\var{AL}{12}$ cm
\item[$\bullet~$] ON = $\var{ON}{3}$  cm
\item[$\bullet~$] O appartient au segment [LN]
\item[$\bullet~$] H appartient au segment [NA]
\end{itemize}
\end{minipage}\hfill
\begin{minipage}{0.48\linewidth}
\definecolor{uuuuuu}{rgb}{0.26666666666666666,0.26666666666666666,0.26666666666666666}
\begin{tikzpicture}[line cap=round,line join=round,>=triangle 45,x=1.0cm,y=1.0cm]
\clip(-2.14,-1.62) rectangle (9.44,10.7);
\draw [line width=2.pt] (6.,0.)-- (0.,9.);
\draw [line width=2.pt] (0.,9.)-- (0.,0.);
\draw [line width=2.pt] (0.,0.)-- (6.,0.);
\draw [line width=2.pt] (3.36,0.)-- (3.36,3.96);
\draw [line width=2.pt] (3.36,0.)-- (3.36,0.36);
\draw [line width=2.pt] (3.36,0.36)-- (3.92,0.34);
\draw [line width=2.pt] (3.92,0.34)-- (3.92,0.);
\begin{scriptsize}
\draw [fill=uuuuuu] (0.,0.) circle (2.0pt);
\draw[color=uuuuuu] (-0.04,-0.35) node {$L$};
\draw [fill=black] (6.,0.) circle (2.5pt);
\draw[color=black] (6.14,0.37) node {$N$};
\draw [fill=black] (0.,9.) circle (2.5pt);
\draw[color=black] (0.14,9.37) node {$A$};
\draw [fill=black] (3.36,0.) circle (2.5pt);
\draw[color=black] (3.32,-0.31) node {$O$};
\draw [fill=uuuuuu] (3.36,3.96) circle (2.0pt);
\draw[color=uuuuuu] (3.5,4.29) node {$H$};
\end{scriptsize}
\end{tikzpicture}
\end{minipage}


\begin{enumerate}
% Question 1 : 
\item 
\enonce{ %début énoncé 

Montrer que le triangle LNA est rectangle en L.
} % fin énoncé 

\correction{ %début énoncé 

$ AN^{2} = \var{AN}{13}^{2} =  \var{ANcarre}{169}$ .

$LN^{2} + AL^{2} = \var{LN}{5}^{2} + \var{AL}{12}^{2} =\var{LNcarre}{25} + \var{ALcarre}{12}= \var{ANcarre}{169}$

donc $AN^{2} = LN^{2} + AL^{2}$.

D'après la réciproque du théorème de Pythagore, le triangle $LNA$ est bien rectangle en $L$. 
\poids{20} %en pourcentage
} % fin correction 

% Question 2 : 
\item 
\enonce{ %début énoncé 

Montrer que la longueur OH est égale à $\var{OH}{7.25}$~cm.
} % fin énoncé 

\correction{ %début énoncé 

D'après la question précédente, $(AL) \perp (LN)$.

D'après le codage de l'énoncé, $(HO) \perp (LN)$.

Donc les droites $(AL)$ et $(HO)$ perpendiculaires à une même droite,
sont parallèles. D'autre part

Les points $N,H,A $ et $N, O, L $ sont alignés.

Les droites $(AL)$ et $(HO)$ sont parallèles.

D'après le théorème de Thalès

\(\displaystyle	\dfrac{NO}{NL}=\dfrac{NH}{NA}=\dfrac{OH}{AL}\) ~~ soit ~~ \(\displaystyle	\dfrac{\var{ON}{3}}{\var{LN}{5}}= \dfrac{NH}{\var{AN}{13}}=\dfrac{OH}{\var{AL}{12}}\),  d'où ~~\(\displaystyle OH = \dfrac{\var{AL}{12} \times \var{ON}{3}}{\var{LN}{5}}  = \var{OH}{7.25}~(\text{cm}) \).
\poids{20} %en pourcentage
} % fin correction 


% Question 3 : 
\item 
\enonce{ %début énoncé 

Calculer la mesure de l'angle $\widehat{\text{LNA}}$. Donner une valeur approchée à l'unité près. 
} % fin énoncé 

\correction{ %début énoncé 

Dans le triangle $ LNA $ rectangle en $ L $, 
\(\displaystyle \cos({\widehat{LNA}})=\dfrac{\text{côté adjacent}}{\text{hypoténuse}}=\dfrac{LN}{AN}= \dfrac{\var{LN}{5}}{\var{AN}{13}}\).

La calculatrice donne avec la fonction inverse de la fonction cosinus : $\widehat{LNA} \approx \var{angleLNA}{67}^\circ$.
\poids{20} %en pourcentage
} % fin correction 

% Question 4 : 
\item 
\enonce{ %début énoncé 

Pourquoi les triangles LNA et ONH sont-ils semblables ?
} % fin énoncé 

\correction{ %début énoncé 

L'angle $\widehat{LNA}$ est un angle commun aux deux triangles.
	
$\widehat{HON}=\widehat{ALN}=90 ~ \text{degrés}$.

Donc les triangles $ LNA $ et $ OHN $ ont deux paires d'angles de même mesures, donc ils sont semblables.
\poids{10} %en pourcentage
} % fin correction 

\item 
\begin{enumerate}
% Question 5 a. : 
\item 
\enonce{ %début énoncé 

Quelle est l'aire du quadrilatère LOHA ?
} % fin énoncé 

\correction{ %début énoncé 

 On calcule les différentes aires :

\(\displaystyle A_{LNA}=\dfrac{\var{LN}{5}\times \var{AL}{12}}{2}=\var{aireLNA}{30}~(\text{cm}^{2})\).

\(\displaystyle A_{OHN}=\dfrac{\var{ON}{3}\times  \var{OH}{7.25}}{2}=\var{aireOHN}{10.8}~~(\text{cm}^{2})\).

\(\displaystyle A_{LOHA}=A_{LNA} - A_{OHN}=\var{aireLOHA}{19.2}~~(\text{cm}^{2})\).
\poids{20} %en pourcentage
} % fin correction 

\item 
% Question 5 b. : 
\enonce{ %début énoncé 

Quelle proportion de l'aire du triangle LNA représente l'aire du quadrilatère LOHA ?
} % fin énoncé 

\correction{ %début énoncé 

\(\displaystyle \dfrac{A_{LOHA}}{A_{LAN}}=\dfrac{\var{aireLOHA}{19.2}}{\var{aireLNA}{30}}=\var{propAire}{0.64}=\dfrac{\var{NUMpropAire}{64}}{\var{DENpropAire}{100}}\).

La proportion est donc \(\displaystyle \dfrac{\var{NUMpropAire}{64}}{\var{DENpropAire}{100}}\).
\poids{10} %en pourcentage
} % fin correction 

\end{enumerate}
\end{enumerate}



