\EXERCICE \SeulementModeEnonce{\poids{100}}

\medskip

Une entreprise a créé une Foire Aux Questions ("FAQ") sur son site internet.

\medskip

On étudie le nombre de questions qui y sont posées chaque mois.

\bigskip

\textbf{Partie A : Première modélisation}

\medskip

Dans cette partie, on admet que, chaque mois :

\begin{itemize}
\item[$\bullet~~$]$\var{pourc}{90}$\,\% des questions déjà posées le mois précédent sont conservées sur la FAQ ;
\item[$\bullet~~$]$\var{nouvques}{130}$ nouvelles questions sont ajoutées à la FAQ.
\end{itemize}

Au cours du premier mois, $\var{Nquesun}{300}$ questions ont été posées.

\medskip

Pour estimer le nombre de questions, en centaines, présentes sur la FAQ le $n$-ième
mois, on modélise la situation ci-dessus à l'aide de la suite $\left(u_n\right)$ définie par : 
\begin{center}$u_1 =\var{u1}{3}$ \quad et, pour tout entier naturel $n \geqslant 1,\:u_{n+1} = \var{q}{0.9}u_n + \var{m}{1,3}$.
\end{center}

\smallskip

\begin{enumerate}
\item 
\enonce{ %début énoncé 

Calculer $u_2$ et $u_3$ et proposer une interprétation dans le contexte de l'exercice. 
} % fin énoncé 

\correction{ %début correction 

$u_2 = \var{q}{0.9}\times u_1 + \var{m}{1,3} = \var{q}{0.9}\times \var{u1}{3} +  \var{m}{1,3}= \var{u2}{4}$ et 
$u_3 =\var{q}{0.9}\times u_2 + \var{m}{1,3}= \var{q}{0.9}\times  \var{u2}{4} +  \var{m}{1,3}= \var{u3}{4.9} $
% et proposer une interprétation dans le contexte de l'exercice.2. 
On peut estimer à $\var{u2cent}{400}$ le nombre de questions le 2ème mois, et à $\var{u3cent}{490}$ le 3ème mois.
} % fin correction 


\item 
\enonce{ %début énoncé 

Montrer par récurrence que pour tout entier naturel $n \geqslant 1$ :
\[u_n = \var{uinf}{13} - \var{Au}{\dfrac{100}{9}} \times \var{q}{0.9}^n.\]
} % fin énoncé 

\correction{ %début correction 

On va montrer par récurrence la propriété  $u_n = \var{uinf}{13} - \var{Au}{\dfrac{100}{9}} \times \var{q}{0.9}^n$ pour tout $n\geqslant1$.

\begin{list}{\textbullet}{}
\item \textbf{Initialisation}

Pour $n=1$, on a $u_1= \var{u1}{3}$ et $\var{uinf}{13} -\var{Au}{\dfrac{100}{9}}\times \var{q}{0.9}^1=\var{u1}{3}$.

La propriété est vérifiée au rang 1.
\item \textbf{Hérédité}

On suppose la propriété vraie au rang $n$ avec $n\geqslant 1$; autrement dit 
$$u_n = \var{uinf}{13} - \var{Au}{\dfrac{100}{9}} \times \var{q}{0.9}^n\mbox{ ;}$$
c'est l'hypothèse de récurrence.

\begin{eqnarray*}
u_{n+1} &=& \var{q}{0.9}u_n+\var{m}{1,3} = \var{q}{0.9} \left (\var{uinf}{13} - \var{Au}{\dfrac{100}{9}} \times \var{q}{0.9}^n\right ) +\var{m}{1,3} \\
&=& \var{q}{0.9}\times \var{uinf}{13} - \var{Au}{\dfrac{100}{9}}\times \var{q}{0.9}^{n+1}+\var{m}{1,3} \\
&=&  \var{uinf}{13}- \var{Au}{\dfrac{100}{9}}\times \var{q}{0.9}^{n+1}
\end{eqnarray*}


Donc la propriété est vraie au rang $n+1$.
\item \textbf{Conclusion}

La propriété est vraie au rang 1 et elle est héréditaire pour tout $n\geqslant 1$; d'après le principe de récurrence, elle est vraie pour tout $n\geqslant 1$.
\end{list}

Donc pour tout $n\geqslant 1$, on a: $u_n = \var{uinf}{13} - \var{Au}{\dfrac{100}{9}} \times \var{q}{0.9}^n$.
} % fin correction 

\end{enumerate}

\begin{minipage}{0.62\linewidth}
\begin{enumerate}
\item  
\enonce{ %début énoncé 

En déduire que la suite $\left(u_n\right)$ est croissante.
} % fin énoncé 

\correction{ %début correction 

Pour tout $n\geqslant 1$, on a:

\begin{eqnarray*}
u_{n+1}-u_n &=& \left (\var{uinf}{13}-\var{Au}{\dfrac{100}{9}}\times \var{q}{0.9}^{n+1}\right ) - \left (\var{uinf}{13}-\var{Au}{\dfrac{100}{9}}\times \var{q}{0.9}^{n}\right ) \\
&= &\var{uinf}{13}-\var{Au}{\dfrac{100}{9}}\times \var{q}{0.9}^{n+1} - \var{uinf}{13}+\var{Au}{\dfrac{100}{9}}\times \var{q}{0.9}^{n}\\
\phantom{u_{n+1}-u_n} &=& \var{Au}{\dfrac{100}{9}}\times \var{q}{0.9}^{n}\left (1-\var{q}{0.9}\right ) >0
\end{eqnarray*}

donc la suite $(u_n)_{n\in\N*}$ est croissante.

} % fin correction 

\item 
\enonce{ %début énoncé 

On considère le programme ci-contre, écrit en langage Python.

Déterminer la valeur renvoyée par la saisie de seuil($\var{useuil}{8.5}$) et l'interpréter dans le contexte de l'exercice.
} % fin énoncé 

\correction{ %début correction 

On considère le programme ci-contre, écrit en langage Python.

\begin{minipage}{0.72\linewidth}

%Déterminer la valeur renvoyée par la saisie de seuil(8.5) et l'interpréter dans le contexte de l'exercice.

Ce programme renvoie la plus petite valeur $n$ telle que $u_n>\texttt{p}$; donc \texttt{seuil($\var{useuil}{8.5}$)} renvoie la plus petite valeur $n$ telle que $u_n>\var{useuil}{8.5}$. On résout cette inéquation.

\begin{eqnarray*}
u_n&>&\var{useuil}{8.5} \\
& \iff &\var{uinf}{13} -\var{Au}{\dfrac{100}{9}}\times \var{q}{0.9}^n > \var{useuil}{8.5}\\
& \iff & \var{uDelta}{4.5} > \var{Au}{\dfrac{100}{9}}\times \var{q}{0.9}^n\\
& \iff &\phantom{u_n>\var{useuil}{8.5}}  \\
& \iff & \dfrac{\var{uDelta}{4.5}\times \var{AuDenom}{9}}{\var{AuNum}{100}}> \var{q}{0.9}^n \\
& \iff &\ln(\var{tmp1}{0,405}) > \ln \left (\var{q}{0.9}^n\right )\\
& \iff &\phantom{u_n>\var{useuil}{8.5}} \\
& \iff &\ln(\var{tmp1}{0,405}) >n \times \ln  (\var{q}{0.9}) \\
& \iff &\dfrac{\ln(\var{tmp1}{0,405})}{\ln(\var{q}{0.9})} <n
\end{eqnarray*}

\end{minipage}\hfill


$\dfrac{\ln(\var{tmp1}{0,405})}{\ln(\var{q}{0.9})} \approx \var{seuilDec}{8.6}$ donc la valeur renvoyée par \texttt{seuil($\var{useuil}{8.5}$)} est $\var{seuilAlg}{9}$.
} % fin correction 

\end{enumerate}
\end{minipage}\hfill
\begin{minipage}{0.32\linewidth}
\renewcommand\arraystretch{0.9}

\begin{tabular}{|l|} \hline
def seuil(p) :\\
\qquad n=1\\
\qquad u=$\var{u1}{3}$\\
\qquad while u<=p :\\
\qquad  \qquad n=n+1\\
\qquad \qquad u=$\var{q}{0.9}$*u+$\var{m}{1,3}$ \\
\qquad return n\\ \hline
\end{tabular}
\end{minipage}

\bigskip

\textbf{Partie B : Une autre modélisation}

\medskip

Dans cette partie, on considère une seconde modélisation à l'aide d'une nouvelle
suite $\left(v_n\right)$ définie pour tout entier naturel $n \geqslant 1$ par:
\[v_n = \var{vinf}{9} - \var{Av}{6} \times \text{e}^{\var{Lv}{-0,19}\times(n - 1)}.\]

Le terme $v_n$ est une estimation du nombre de questions, en centaines, présentes le $n$-ième mois sur la FAQ.

\medskip

\begin{enumerate}
\item 
\enonce{ %début énoncé 

Préciser les valeurs arrondies au centième de $v_1$ et $v_2$.
} % fin énoncé 

\correction{ %début correction 

$v_1=\var{vinf}{9}-\var{Av}{6} \e^{0}=\var{v1}{3}$ et $v_2=\var{vinf}{9}-\var{Av}{6}\e^{\var{Lv}{-0,19}}\approx \var{v2}{4,04}$
} % fin correction 

\item 
\enonce{ %début énoncé 

Déterminer, en justifiant la réponse, la plus petite valeur de $n$ telle que $v_n > \var{useuil}{8.5}$.
} % fin énoncé 

\correction{ %début correction 

On résout l'inéquation $v_n > \var{useuil}{8.5}$.

\begin{eqnarray*}
v_n &>& \var{useuil}{8.5} \\
\iff && \var{vinf}{9} - \var{Av}{6} \times \e^{\var{Lv}{-0,19}\times(n - 1)} > \var{useuil}{8.5} \\
\iff &&\var{tmp2}{0,5} >  \var{Av}{6} \times \e^{\var{Lv}{-0,19}\times(n - 1)} \\
\iff&&\dfrac{\var{tmp2}{0,5}}{\var{Av}{6}} > \e^{\var{Lv}{-0,19}\times(n - 1)} \\
&&\mbox{ et par croissance de la fonction logarithme népérien } \\
\iff&& \ln \left (\dfrac{\var{tmp2}{0,5}}{\var{Av}{6}}\right ) > \var{Lv}{-0,19}\times (n-1) \\
\iff&&-\dfrac{\ln \left (\frac{\var{tmp2}{0,5}}{\var{Av}{6}}\right )}{\var{pLv}{0,19}} < n - 1 \\
n &>& 1 -\dfrac{\ln \left (\frac{\var{tmp2}{0,5}}{\var{Av}{6}}\right )}{\var{pLv}{0,19}}
\end{eqnarray*}



Or $-\dfrac{\ln \left (\frac{\var{tmp2}{0,5}}{\var{Av}{6}}\right )}{\var{pLv}{0,19}}\approx \var{tmp3}{-13,08}$ et $1 -( \var{tmp3}{-13,08}) = \var{nvapprox}{14,08}$ donc la plus petite valeur telle que $v_n > \var{useuil}{8.5}$ est $n = \var{nv}{15}$.
} % fin correction 

\end{enumerate}

\bigskip

\textbf{Partie C : Comparaison des deux modèles}

\medskip

\begin{enumerate}
\item 
\enonce{ %début énoncé 

L'entreprise considère qu'elle doit modifier la présentation de son site lorsque plus de $\var{useuilcent}{850}$ questions sont présentes sur la FAQ. 

Parmi ces deux modélisations, laquelle conduit à procéder le plus tôt à cette modification ?

Justifier votre réponse.
} % fin énoncé 

\correction{ %début correction 

\begin{list}{\textbullet}{}
\item Selon le 1er modèle, il y a plus de $\var{useuilcent}{850}$ questions sur la FAQ quand $u_n>\var{useuil}{8.5}$, c'est-à-dire le $\var{seuilAlg}{9}$ème mois.
\item Selon le 2ème modèle, il y a plus de $\var{useuilcent}{850}$ questions sur la FAQ quand $v_n>\var{useuil}{8.5}$, c'est-à-dire le $ \var{nv}{15}$ème mois.
\end{list}

C'est la $\var{choixmod}{1ere}$ modélisation qui conduit à procéder le plus tôt à cette modification.

} % fin correction 

\item 
\enonce{ %début énoncé 

En justifiant la réponse, pour quelle modélisation y a-t-il le plus grand nombre de questions sur la FAQ à long terme?
} % fin énoncé 

\correction{ %début correction 


\begin{list}{\textbullet}{}
\item $u_n=\var{uinf}{13} - \var{Au}{\dfrac{100}{9}}\times \var{q}{0.9}^n$

$-1<\var{q}{0.9} <1$ donc $\lim_{n\to +\infty} \var{q}{0.9}^n=0$ donc $\lim_{n\to +\infty} u_n=\var{uinf}{13}$.

À long terme il y aura $\var{uinfcent}{1300}$ questions pour la 1er modélisation..
\item $v_n = \var{vinf}{9} - \var{Av}{6} \times \e^{\var{Lv}{-0,19}\times(n - 1)}$

$\lim_{n\to +\infty} \var{Lv}{-0,19}\times (n-1) = -\infty$ et $\lim_{X\to -\infty} \e^{X}=0$ donc $\lim_{n\to +\infty} \e^{\var{Lv}{-0,19}\times(n - 1)}=0$; on en déduit que $\lim_{n\to +\infty} v_n=\var{vinf}{9}$.

À long terme il y aura $\var{vinfcent}{900}$ questions pour la 2ème modélisation..
\end{list}

C'est donc pour la $\var{choixmod2}{1ere}$ modélisation qu'il y aura le plus de questions à long terme.
} % fin correction 

\end{enumerate}

\bigskip



