\EXERCICE
\SeulementModeEnonce{\poids{100} }
\begin{tikzpicture}[line cap=round,line join=round,>=triangle 45,x=1.0cm,y=1.0cm]
\begin{axis}[
x=1.5cm,y=0.5cm,
axis lines=middle,
grid style=dashed,
ymajorgrids=true,
xmajorgrids=true,
xmin=-5.2,
xmax=6,
ymin=-8.0,
ymax=22.25,
xtick={-5.0,-4.5,...,6},
ytick={-8.0,-7.0,...,22.0},]
\clip(-5.17031,-8.06760180995476) rectangle (8.9,22.2);
\draw[line width=2.pt,smooth,samples=100,domain=\var{xA}{-3}:\var{xB}{4.}] plot(\x,{\var{ftikz}{(\x)^(2.0)-2*(\x)-5}});
%\draw[line width=2.pt,smooth,samples=100,domain=-3.0:4.0] plot(\x,{(\x)^(2.0)-2*(\x)-5});
\begin{scriptsize}
\draw[color=black] (\var{xB}{4.}+1,\var{yB}{3.}) node {$(C_f)$};
\draw[color=black] (5.8,0.5) node {{\bf\it x}};
\draw[color=black] (0.2,21.5) node {{\bf\it y}};
\draw [fill=black] (\var{xA}{-3},\var{yA}{10}) circle (2.5pt);
%\draw [fill=black] (-3.,10.) circle (2.5pt);
\draw [fill=black] (\var{xB}{4.},\var{yB}{3.}) circle (2.5pt);
%\draw [fill=black] (4.0,3.) circle (2.5pt);
\end{scriptsize}
\end{axis}
\end{tikzpicture}

On considère $(C_f)$ la courbe représentative d'une fonction $f $ dans un repère.

\bigskip
\centerline{\bf Partie A}

\begin{description}
\item[1)] 
\enonce{ %début énoncé 

Déterminer son ensemble de définition $D$ .
} % fin énoncé 

\correction{ %début correction 

L'ensemble de définition est $D = [\var{xA}{-3}\,;\,\var{xB}{4}]$. \poids{5} %en pourcentage
} % fin correction 

\item[2)] 
\enonce{ %début énoncé 

Déterminer le maximum et le minimum sur $D$ .
} % fin énoncé 

\correction{ %début correction 

Le maximum de $f$ sur $D$ est $\var{ymaxpAq2}{10}$ \poids{5} %en pourcentage\\
Le minimum de $f$ sur $D$ est $\var{yminpAq2}{-6}$.  \poids{5} %en pourcentage
} % fin correction 

\item[3)]  
\begin{description}
 \item [a.] 
\enonce{ %début énoncé 

Quelle est l'image de $0$ ?
 } % fin énoncé 

\correction{ %début correction 

 L'image de $0$ est $f(0)=\var{ypAq3a}{-5}$. \poids{5} %en pourcentage
 } % fin correction 

 \item[b.] 
\enonce{ %début énoncé 

Quels sont les antécédents de $\var{ypAq3b2ant}{2}$ ?
 } % fin énoncé 

\correction{ %début correction 

 Les antécédents de $\var{ypAq3b2ant}{2}$ sont (valeurs approchées) $\var{xpAq3bsol1}{-1.8}$ et $\var{xpAq3bsol2}{3.7}$ \poids{5} %en pourcentage
 } % fin correction 

\end{description}

\item[4)] 
\enonce{ %début énoncé 

Résoudre graphiquement les équations 
} % fin énoncé 

\begin{description}
\item[a.] 
\enonce{ %début énoncé 

$f(x) =  \var{ypAq4a1ant}{1}$
} % fin énoncé 

\correction{ %début correction 

$f(x) = \var{ypAq4a1ant}{1}$  $\var{ques4aSol}{\mbox{pour }x\approx 3.6\mbox{ et }x\approx -1.6}$ \poids{5} %en pourcentage
} % fin correction 

\item[b.] 
\enonce{ %début énoncé 

$f(x) = 0$.
} % fin énoncé 

\correction{ %début correction 

$f(x) = \var{ypAq4a0ant}{0}$ $\var{quespA4bSol}{\mbox{pour}x\approx -1.5\mbox{ et }x\approx 3.5}$ \poids{5} %en pourcentage
} % fin correction 

\end{description}

\item[5)]  
\enonce{ %début énoncé 

Résoudre graphiquement l'inéquation $f(x)\geq \var{ypAq5}{-3}$.
} % fin énoncé 

\correction{ %début correction 

Par lecture graphique on trouve $S =\var{ques5solS}{ [-3\,;\,-0.7] \cup  [2.7\,;\,4]}$ \poids{5} %en pourcentage
} % fin correction 

 \item[6)] 
\enonce{ %début énoncé 

Dresser la tableau de variation sur $D$ .
} % fin énoncé 

\correction{ %début correction 

\begin{center}
\begin{tabular}{|l|lllll|} \hline
$x$    & $\var{xA}{-3}$     &                            & $\var{x0}{1}$        &                            & $\var{xB}{4}$  \\ \hline
       & $\var{yAp}{}$    &                            & $\var{y0m}{-6}$      &                            & $\var{yBp}{3}$   \\ 
$f(x)$ &                    & $\var{varGpAq6}{\searrow}$ &                      & $\var{varDpAq6}{\nearrow}$ &                \\
       & $\var{yAm}{10}$    &                            & $\var{y0p}{}$      &                            & $\var{yBm}{}$    \\ \hline
\end{tabular}
\end{center}
\poids{10} %en pourcentage
} % fin correction   

\end{description}

\bigskip
\centerline{\bf Partie B}
\bigskip
\enonce{ %début énoncé 

On sait maintenant, en plus, que f est définie par $f(x) =\var{fdev}{ x^2-2 x -5}$.
} % fin énoncé 

\begin{description}
\item[1)] 
\enonce{ %début énoncé 

Déterminer les images de $0$, $-1$ et $\sqrt{2}$.
} % fin énoncé 

\correction{ %début correction 

On a : 
\begin{eqnarray*}
 f(0) &=&\var{f0pBq1}{ 0^2 - 2 \times 0 -5 = -5 }\\
  f(-1) &=& \var{fm1pBq1}{(-1)^2 - 2 \times (-1) -5 = -2}\\
    f(\sqrt{2}) &=&\var{fm2pBq1}{ (\sqrt{2})^2 - 2 \times \sqrt{2} -5 = 2 - 2 \times \sqrt{2} -5 =  -3 - 2  \sqrt{2}  }
\end{eqnarray*}
\poids{15} %en pourcentage
} % fin correction 

\item[2)] 
\enonce{ %début énoncé 

Montrer que $f(x) =\var{fcan}{(x-1)^2 -6}$. 
} % fin énoncé 

\correction{ %début correction 

\begin{eqnarray*}
 \var{fcan}{(x-1)^2 -6}  &=& \var{formecanpBq2et1}{x^2 - 2 x\times 1 + 1^2 - 6} \\
  &=& \var{formecanpBq2et2}{x^2 - 2 x -5 }\\
  &=& f(x) 
\end{eqnarray*}
\poids{10} %en pourcentage
} % fin correction 


\item[3)] 
\enonce{ %début énoncé 

Déterminer les éventuels antécédents de $\var{ypBq3deuxant}{0}$ et $\var{ypBq3unant}{5}$ . On donnera les solutions exactes. 
} % fin énoncé 

\correction{ %début correction 

Il faut résoudre $f(x)=\var{ypBq3deuxant}{0}$ : 

$$
\var{quespBq3sol2ant}{
\begin{array}{lll}
 f(x) &=& 0 \\
 (x-1)^2 -6  &=&0 \\
 (x-1)^2 -(\sqrt{6})^2  &=&0 \\
 (x-1-\sqrt{6})  (x-1+\sqrt{6}) &=&0 \\
\end{array}
}
$$


Donc (propriété équation-produit), comme $\var{x1deuxant}{1+\sqrt{6}\in D}$ 
et $\var{x2deuxant}{1-\sqrt{6}\in D}$, on a l'ensemble des solutions 
$ S =\var{ensSdeuxant}{\{1+\sqrt{6} \,;\,1-\sqrt{6} \}} $\poids{10} %en pourcentage

Il faut résoudre $f(x)=\var{ypBq3unant}{5}$ : 
$$
\var{quespBq3sol1ant}{
\begin{array}{lll}
 f(x) &=&5 \\
 (x-1)^2 -6  &=&5 \\
  (x-1)^2 -11  &=&0 \\
 (x-1)^2 -(\sqrt{11})^2  &=&0 \\
 (x-1-\sqrt{11})  (x-1+\sqrt{11}) &=&0 
\end{array}
}
$$

Donc (propriété équation-produit), comme $\var{x1unant}{1+\sqrt{11}\not\in D}$ 
et $\var{x2unant}{1-\sqrt{11}\in D}$, on a l'ensemble des solutions 
$ S =\var{ensSunant}{ \{1-\sqrt{11} \}}$\poids{15} %en pourcentage
} % fin correction 

\end{description}








