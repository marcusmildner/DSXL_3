%DEBUT EnTete
\documentclass[10pt,a4paper]{article}
\usepackage[T1]{fontenc}
\usepackage{amsmath}
\usepackage{amsfonts}
\usepackage{amssymb}
\usepackage{graphicx}
\usepackage{calc}
\usepackage{pgf,tikz}
%https://stackoverflow.com/questions/67845020/how-to-avoid-latex-error-environment-axis-undefined-when-using-include-tikz : 
\usepackage{pgfplots}
% https://tex.stackexchange.com/questions/81899/what-does-running-in-backwards-compatibility-mode-mean-and-what-should-i-fix-t
\pgfplotsset{compat=1.17}
\usepackage{pstricks-add}
\usetikzlibrary{arrows}
\usepackage{mathrsfs}
\usepackage{ifthen}
\usepackage{pdfpages}
\usepackage[left=2cm,right=2cm,top=2cm,bottom=2cm]{geometry}
\usepackage{datatool}
\everymath{\displaystyle}
\usepackage{fancyhdr} 
\usepackage{hyperref}
%\usepackage{sagetex}
%FIN EnTete
%DEBUT MesCommandes

\setlength{\parindent}{0em}



% *********************************************************************
% MODIFIER LA LIGNE SELON LE MODE VOULU : **********************
\newcommand{\EnonceCorrection}{C}	% E = mode énoncé   C = mode correction 
% **********************************************************************



%%%%%%%%%%%%% MetaDonnees.tex %%%%%%%%%%%%%%%%%%%%%
\newcommand{\AfficheMetaAuteur}{
% Pour imprimer les métadonnées (à effacer) : 
{\bf Auteur : \AuteurEx    \hfill Nom de l'exercice : \varnom  \hfill  Version :   \VersionEx }

{\bf Date de création : \DateCreationEx \hfill Date de modification : \DateModificationEx }

{\bf Source de l'exercice : \SourceEx }

{\bf Description : \DescriptionEx}

\hrulefill

}
%%%%%%%%%%%%% TitreDuTexte %%%%%%%%%%%%%%%%%%%%%
% #1 :  niveau                   #2 : Titre
\newcommand{\TitreDuTexte}{
{\bf \varnom \large   \hfill \NiveauEx Mathématiques \hfill   $ e^{i\pi}+1=0$ }

\bigskip
\centerline{\bf \large \NomEx}
\bigskip
}

%%%%% Quelques commandes personnalisées %%%%%%%%%%%%%%%%%%
\newcommand{\euro}{\eurologo{}} 
\newcommand{\e}{\mbox{e}}
\newcommand{\R}{\mathbb{R}}
\newcommand{\N}{\mathbb{N}}
\newcommand{\D}{\mathbb{D}}
\newcommand{\Z}{\mathbb{Z}}
\newcommand{\Q}{\mathbb{Q}}
%\newcommand{\C}{\mathbb{C}}
\renewcommand{\mathbf}{}
\renewcommand{\vec}[1]{\overrightarrow{#1}}

%============================================================
% POUR L'ENONCE : Le paramètre est l'énoncé \enonce{Ceci est l'énoncé}
\newcommand{\enonce}[1]{
% \EnonceCorrection = E (énoncé seulement) ou C (énoncé et correction)
\ifthenelse{\equal{\EnonceCorrection}{E}}{#1}{ {\bf \it #1}}{} }

% POUR LA CORRECTION : Le paramètre est la solution \correction{Ceci est la correction}
\newcommand{\correction}[1]{
% \EnonceCorrection = E (énoncé seulement) ou C (énoncé et correction)
\ifthenelse{\equal{\EnonceCorrection}{C}}{ \ \ \newline 
 {\bf SOLUTION : } #1}{}
}

% VARIABLE points par exercices 
\newcounter{pointsexercice}
% Cette variable se remets à 0 à chaque début d'exercice par la commande : \setcounter{pointexercice}{0}

% VARIABLE qui contient le total des points 
\newcounter{pointssujetTOTAL}
\setcounter{pointssujetTOTAL}{0}

% VARIABLE qui contient le numéro de l'exercice : 
\newcounter{numeroexercice}

\newcommand{\Affichepointsexercice}{
% \EnonceCorrection = E (énoncé seulement) ou C (énoncé et correction)
\ifthenelse{\equal{\EnonceCorrection}{C}}{ \ \ \hfill {\bf  Cet exercice comporte  $\thepointsexercice$  points }}{}

}

\newcommand{\AffichepointssujetTOTAL}{
% \EnonceCorrection = E (énoncé seulement) ou C (énoncé et correction)
\ifthenelse{\equal{\EnonceCorrection}{C}}{ \ \ \newline  {\bf Le sujet compte un total de \thepointssujetTOTAL \ points.} }{}
}

\newcommand{\points}[1]{ {\bf [#1 point(s)]}}

\newcommand{\poids}[1]{ {\bf [poids = #1 \%]}}

\newcommand{\EXERCICE}{
% On ajoute 1 au numéro de l'exercice :
\setcounter{numeroexercice}{\thenumeroexercice + 1}
% On remet à zéro les points de l'exercice 
\setcounter{pointsexercice}{0}

\bigskip
{\bf EXERCICE \thenumeroexercice \ (Tous les résultats doivent être justifiés)} 

}

% Cette commande insère une nouvelle page en mode ENONCE 
\newcommand{\NouvellePageModeEnonce}{
% \EnonceCorrection = E (énoncé seulement) ou C (énoncé et correction)
\ifthenelse{\equal{\EnonceCorrection}{E}}{ \newpage }{} }

% Cette commande insère une nouvelle page en mode CORRECTION 
\newcommand{\NouvellePageModeCorrection}{
% \EnonceCorrection = E (énoncé seulement) ou C (énoncé et correction)
\ifthenelse{\equal{\EnonceCorrection}{C}}{ \newpage }{} }


% Cette commande seulement mode ENONCE 
\newcommand{\SeulementModeEnonce}[1]{
% \EnonceCorrection = E (énoncé seulement) ou C (énoncé et correction)
\ifthenelse{\equal{\EnonceCorrection}{E}}{ #1 }{} }

% Cette commande seulement en mode CORRECTION 
\newcommand{\SeulementModeCorrection}[1]{
% \EnonceCorrection = E (énoncé seulement) ou C (énoncé et correction)
\ifthenelse{\equal{\EnonceCorrection}{C}}{ #1 }{} }

% Commande balise pour définir une variable : \var{p}{10} affichera juste 10, mais 
% le programme reconnaitra la variable p 
\newcommand{\var}[2]{\fbox{#2}_{\fbox{\mbox{\bf #1}}}}


%FIN MesCommandes
%DEBUT MetaAuteur
\newcommand{\AuteurEx}{} 
\newcommand{\VersionEx}{} 
\newcommand{\DateCreationEx}{} 
\newcommand{\DateModificationEx}{} 
\newcommand{\SourceEx}{} 
\newcommand{\NotionsEx}{} 
\newcommand{\NiveauEx}{} 
\newcommand{\NomEx}{}
%FIN MetaAuteur

\begin{document}
\EXERCICE 
\medskip
\begin{minipage}{0.48\linewidth}
On considère la figure ci-contre. On donne les mesures suivantes:

\begin{itemize}
\item[$\bullet~$] $\var{AN}{13}$ = 13 cm
\item[$\bullet~$] $\var{LN}{5}$ = 5 cm
\item[$\bullet~$] $\var{AL}{12}$ = 12 cm
\item[$\bullet~$] $\var{ON}{3}$ = 3 cm
\item[$\bullet~$] O appartient au segment [LN]
\item[$\bullet~$] H appartient au segment [NA]
\end{itemize}
\end{minipage}\hfill
\begin{minipage}{0.48\linewidth}
\definecolor{uuuuuu}{rgb}{0.26666666666666666,0.26666666666666666,0.26666666666666666}
\begin{tikzpicture}[line cap=round,line join=round,>=triangle 45,x=1.0cm,y=1.0cm]
\clip(-2.14,-1.62) rectangle (9.44,10.7);
\draw [line width=2.pt] (6.,0.)-- (0.,9.);
\draw [line width=2.pt] (0.,9.)-- (0.,0.);
\draw [line width=2.pt] (0.,0.)-- (6.,0.);
\draw [line width=2.pt] (3.36,0.)-- (3.36,3.96);
\draw [line width=2.pt] (3.36,0.)-- (3.36,0.36);
\draw [line width=2.pt] (3.36,0.36)-- (3.92,0.34);
\draw [line width=2.pt] (3.92,0.34)-- (3.92,0.);
\begin{scriptsize}
\draw [fill=uuuuuu] (0.,0.) circle (2.0pt);
\draw[color=uuuuuu] (-0.04,-0.35) node {$L$};
\draw [fill=black] (6.,0.) circle (2.5pt);
\draw[color=black] (6.14,0.37) node {$N$};
\draw [fill=black] (0.,9.) circle (2.5pt);
\draw[color=black] (0.14,9.37) node {$A$};
\draw [fill=black] (3.36,0.) circle (2.5pt);
\draw[color=black] (3.32,-0.31) node {$O$};
\draw [fill=uuuuuu] (3.36,3.96) circle (2.0pt);
\draw[color=uuuuuu] (3.5,4.29) node {$H$};
\end{scriptsize}
\end{tikzpicture}
\end{minipage}


\begin{enumerate}
% Question 1 : 
\item 
%E>
Montrer que le triangle LNA est rectangle en L.
%E<
%C>
$ AN^{2} = \var{AN}{13}^{2} =  \var{ANcarre}{169}$ .

$LN^{2} + AL^{2} = \var{LN}{5}^{2} + \var{AL}{12}^{2} =\var{LNcarre}{25} + \var{ALcarre}{12}= \var{ANcarre}{169}$

donc $AN^{2} = LN^{2} + AL^{2}$.

D'après la réciproque du théorème de Pythagore, le triangle $LNA$ est bien rectangle en $L$. 
\poids{20} %en pourcentage
%C<
% Question 2 : 
\item 
%E>
Montrer que la longueur OH est égale à $\var{OH}{7.25}$~cm.
%E<
%C>
D'après la question précédente, $(AL) \perp (LN)$.

D'après le codage de l'énoncé, $(HO) \perp (LN)$.

Donc les droites $(AL)$ et $(HO)$ perpendiculaires à une même droite,
sont parallèles. D'autre part

Les points $N,H,A $ et $N, O, L $ sont alignés.

Les droites $(AL)$ et $(HO)$ sont parallèles.

D'après le théorème de Thalès

\(\displaystyle	\dfrac{NO}{NL}=\dfrac{NH}{NA}=\dfrac{OH}{AL}\) ~~ soit ~~ \(\displaystyle	\dfrac{\var{ON}{3}}{\var{LN}{5}}= \dfrac{NH}{\var{AN}{13}}=\dfrac{OH}{\var{AL}{12}}\),  d'où ~~\(\displaystyle OH = \dfrac{\var{AL}{12} \times \var{ON}{3}}{\var{LN}{5}}  = \var{OH}{7.25}~(\text{cm}) \).
\poids{20} %en pourcentage
%C<

% Question 3 : 
\item 
%E>
Calculer la mesure de l'angle $\widehat{\text{LNA}}$. Donner une valeur approchée à l'unité près. 
%E<
%C>
Dans le triangle $ LNA $ rectangle en $ L $, 
\(\displaystyle \cos({\widehat{LNA}})=\dfrac{\text{côté adjacent}}{\text{hypoténuse}}=\dfrac{LN}{AN}= \dfrac{\var{LN}{5}}{\var{AN}{13}}\).

La calculatrice donne avec la fonction inverse de la fonction cosinus : $\widehat{LNA} \approx \var{angleLNA}{67}^\circ$.
\poids{20} %en pourcentage
%C<
% Question 4 : 
\item 
%E>
Pourquoi les triangles LNA et ONH sont-ils semblables ?
%E<
%C>
L'angle $\widehat{LNA}$ est un angle commun aux deux triangles.
	
$\widehat{HON}=\widehat{ALN}=90 ~ \text{degrés}$.

Donc les triangles $ LNA $ et $ OHN $ ont deux paires d'angles de même mesures, donc ils sont semblables.
\poids{10} %en pourcentage
%C<
\item 
\begin{enumerate}
% Question 5 a. : 
\item 
%E>
Quelle est l'aire du quadrilatère LOHA ?
%E<
%C>
 On calcule les différentes aires :

\(\displaystyle A_{LNA}=\dfrac{\var{LN}{5}\times \var{AL}{12}}{2}=\var{aireLNA}{30}~(\text{cm}^{2})\).

\(\displaystyle A_{OHN}=\dfrac{\var{ON}{3}\times  \var{OH}{7.25}}{2}=\var{aireOHN}{10.8}~~(\text{cm}^{2})\).

\(\displaystyle A_{LOHA}=A_{LNA} - A_{OHN}=\var{aireLOHA}{19.2}~~(\text{cm}^{2})\).
\poids{20} %en pourcentage
%C<
\item 
% Question 5 b. : 
%E>
Quelle proportion de l'aire du triangle LNA représente l'aire du quadrilatère LOHA ?
%E<
%C>
\(\displaystyle \dfrac{A_{LOHA}}{A_{LAN}}=\dfrac{\var{aireLOHA}{19.2}}{\var{aireLNA}{30}}=\var{propAire}{0.64}=\dfrac{\var{NUMpropAire}{64}}{\var{DENpropAire}{100}}\).

La proportion est donc \(\displaystyle \dfrac{\var{NUMpropAire}{64}}{\var{DENpropAire}{100}}\).
\poids{10} %en pourcentage
%C<
\end{enumerate}
\end{enumerate}

















\end{document}
