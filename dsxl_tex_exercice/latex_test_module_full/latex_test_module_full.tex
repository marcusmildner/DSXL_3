%DEBUT EnTete
\documentclass[10pt,a4paper]{article}
\usepackage[T1]{fontenc}
\usepackage{amsmath}
\usepackage{amsfonts}
\usepackage{amssymb}
\usepackage{graphicx}
\usepackage{calc}
\usepackage{pgf,tikz}
\usepackage{pstricks-add}
\usetikzlibrary{arrows}
\usepackage{mathrsfs}
\usepackage{ifthen}
\usepackage{pdfpages}
\usepackage[left=2cm,right=2cm,top=2cm,bottom=2cm]{geometry}
\usepackage{datatool}
\everymath{\displaystyle}
\usepackage{fancyhdr} 
\usepackage{hyperref}
\usepackage{sagetex}
%FIN EnTete
%DEBUT MesCommandes

\setlength{\parindent}{0em}



% *********************************************************************
% MODIFIER LA LIGNE SELON LE MODE VOULU : **********************
\newcommand{\EnonceCorrection}{C}	% E = mode énoncé   C = mode correction 
% **********************************************************************



%%%%%%%%%%%%% MetaDonnees.tex %%%%%%%%%%%%%%%%%%%%%
\newcommand{\AfficheMetaAuteur}{
% Pour imprimer les métadonnées (à effacer) : 
{\bf Auteur : \AuteurEx    \hfill Nom de l'exercice : \varnom  \hfill  Version :   \VersionEx }

{\bf Date de création : \DateCreationEx \hfill Date de modification : \DateModificationEx }

{\bf Source de l'exercice : \SourceEx }

{\bf Description : \DescriptionEx}

\hrulefill

}
%%%%%%%%%%%%% TitreDuTexte %%%%%%%%%%%%%%%%%%%%%
% #1 :  niveau                   #2 : Titre
\newcommand{\TitreDuTexte}{
{\bf \varnom \large   \hfill \NiveauEx Mathématiques \hfill   $ e^{i\pi}+1=0$ }

\bigskip
\centerline{\bf \large \NomEx}
\bigskip
}

%%%%% Quelques commandes personnalisées %%%%%%%%%%%%%%%%%%
\newcommand{\euro}{\eurologo{}} 
\newcommand{\e}{\mbox{e}}
\newcommand{\R}{\mathbb{R}}
\newcommand{\N}{\mathbb{N}}
\newcommand{\D}{\mathbb{D}}
\newcommand{\Z}{\mathbb{Z}}
\newcommand{\Q}{\mathbb{Q}}
%\newcommand{\C}{\mathbb{C}}
\renewcommand{\mathbf}{}
\renewcommand{\vec}[1]{\overrightarrow{#1}}

%============================================================
% POUR L'ENONCE : Le paramètre est l'énoncé \enonce{Ceci est l'énoncé}
\newcommand{\enonce}[1]{
% \EnonceCorrection = E (énoncé seulement) ou C (énoncé et correction)
\ifthenelse{\equal{\EnonceCorrection}{E}}{#1}{ {\bf \it #1}}{} }

% POUR LA CORRECTION : Le paramètre est la solution \correction{Ceci est la correction}
\newcommand{\correction}[1]{
% \EnonceCorrection = E (énoncé seulement) ou C (énoncé et correction)
\ifthenelse{\equal{\EnonceCorrection}{C}}{ \ \ \newline 
 {\bf SOLUTION : } #1}{}
}

% VARIABLE points par exercices 
\newcounter{pointsexercice}
% Cette variable se remets à 0 à chaque début d'exercice par la commande : \setcounter{pointexercice}{0}

% VARIABLE qui contient le total des points 
\newcounter{pointssujetTOTAL}
\setcounter{pointssujetTOTAL}{0}

% VARIABLE qui contient le numéro de l'exercice : 
\newcounter{numeroexercice}

\newcommand{\Affichepointsexercice}{
% \EnonceCorrection = E (énoncé seulement) ou C (énoncé et correction)
\ifthenelse{\equal{\EnonceCorrection}{C}}{ \ \ \hfill {\bf  Cet exercice comporte  $\thepointsexercice$  points }}{}

}

\newcommand{\AffichepointssujetTOTAL}{
% \EnonceCorrection = E (énoncé seulement) ou C (énoncé et correction)
\ifthenelse{\equal{\EnonceCorrection}{C}}{ \ \ \newline  {\bf Le sujet compte un total de \thepointssujetTOTAL \ points.} }{}
}

\newcommand{\points}[1]{ {\bf [#1 point(s)]}}

\newcommand{\poids}[1]{ {\bf [poids = #1 \%]}}

\newcommand{\EXERCICE}{
% On ajoute 1 au numéro de l'exercice :
\setcounter{numeroexercice}{\thenumeroexercice + 1}
% On remet à zéro les points de l'exercice 
\setcounter{pointsexercice}{0}

\bigskip
{\bf EXERCICE \thenumeroexercice \ (Tous les résultats doivent être justifiés)} 

}

% Cette commande insère une nouvelle page en mode ENONCE 
\newcommand{\NouvellePageModeEnonce}{
% \EnonceCorrection = E (énoncé seulement) ou C (énoncé et correction)
\ifthenelse{\equal{\EnonceCorrection}{E}}{ \newpage }{} }

% Cette commande insère une nouvelle page en mode CORRECTION 
\newcommand{\NouvellePageModeCorrection}{
% \EnonceCorrection = E (énoncé seulement) ou C (énoncé et correction)
\ifthenelse{\equal{\EnonceCorrection}{C}}{ \newpage }{} }


% Cette commande seulement mode ENONCE 
\newcommand{\SeulementModeEnonce}[1]{
% \EnonceCorrection = E (énoncé seulement) ou C (énoncé et correction)
\ifthenelse{\equal{\EnonceCorrection}{E}}{ #1 }{} }

% Cette commande seulement en mode CORRECTION 
\newcommand{\SeulementModeCorrection}[1]{
% \EnonceCorrection = E (énoncé seulement) ou C (énoncé et correction)
\ifthenelse{\equal{\EnonceCorrection}{C}}{ #1 }{} }

% Commande balise pour définir une variable : \var{p}{10} affichera juste 10, mais 
% le programme reconnaitra la variable p 
\newcommand{\var}[2]{\fbox{#2}_{\fbox{\mbox{\bf #1}}}}


%FIN MesCommandes
%DEBUT MetaAuteur
\newcommand{\AuteurEx}{} 
\newcommand{\VersionEx}{} 
\newcommand{\DateCreationEx}{} 
\newcommand{\DateModificationEx}{} 
\newcommand{\SourceEx}{} 
\newcommand{\NotionsEx}{} 
\newcommand{\NiveauEx}{} 
\newcommand{\NomEx}{}
%FIN MetaAuteur

\begin{document}



\EXERCICE
\enonce{ %début énoncé 

Soit ABC un triangle rectangle en B. On sait que AB = 3 et BC = 3
} % fin énoncé 

\begin{description}
\item[1.] \enonce{ %début énoncé 
Calculer la longueur de AC \points{2.0} %en pourcentage
} % fin énoncé 

\correction{ %début énoncé 
 Le triangle ABC est rectangle en B,, donc avec le théorème de Pythagore on a : 
\begin{eqnarray*}
AB^2 + BC^2 &=& AC^2 \\
3^2 + 3^2 &=& AC^2 \\
18 &=&AC^2 \\
& & \mbox{ Donc : } AC = \sqrt{18} 
\end{eqnarray*}
} % fin correction 

\item[2.] \enonce{ %début énoncé 
Calculer l'angle $\widehat{ABC}$ au degré près. } % fin énoncé 
\points{2.0} %en pourcentage
\end{description}

\psset{xunit=1.0cm,yunit=1.0cm,algebraic=true,dimen=middle,dotstyle=o,dotsize=5pt 0,linewidth=2.pt,arrowsize=3pt 2,arrowinset=0.25}
\begin{pspicture*}(-4.3,-3.02)(7.28,6.3)
\multips(0,-3)(0,1.0){10}{\psline[linestyle=dashed,linecap=1,dash=1.5pt 1.5pt,linewidth=0.4pt,linecolor=lightgray]{c-c}(-4.3,0)(7.28,0)}
\multips(-4,0)(1.0,0){12}{\psline[linestyle=dashed,linecap=1,dash=1.5pt 1.5pt,linewidth=0.4pt,linecolor=lightgray]{c-c}(0,-3.02)(0,6.3)}
\psaxes[labelFontSize=\scriptstyle,xAxis=true,yAxis=true,Dx=1.,Dy=1.,ticksize=-2pt 0,subticks=2]{->}(0,0)(-4.3,-3.02)(7.28,6.3)
\psplot[linewidth=2.pt]{-4.3}{7.28}{(-1+-0.4*x)/1.}
%\psplot[linewidth=2.pt]{-4.3}{7.28}{(-2.--1.5*x)/1.}
\begin{scriptsize}
\rput[bl](5.08,5.98){$f$}
\end{scriptsize}
\end{pspicture*}













































\EXERCICE
\enonce{ %début énoncé 

Soit ABC un triangle rectangle en B. On sait que AB = 3 et BC = 3
} % fin énoncé 

\begin{description}
\item[1.] \enonce{ %début énoncé 
Calculer la longueur de AC \points{2.0} %en pourcentage
} % fin énoncé 

\correction{ %début énoncé 
 Le triangle ABC est rectangle en B,, donc avec le théorème de Pythagore on a : 
\begin{eqnarray*}
AB^2 + BC^2 &=& AC^2 \\
3^2 + 3^2 &=& AC^2 \\
18 &=&AC^2 \\
& & \mbox{ Donc : } AC = \sqrt{18} 
\end{eqnarray*}
} % fin correction 

\item[2.] \enonce{ %début énoncé 
Calculer l'angle $\widehat{ABC}$ au degré près. } % fin énoncé 
\points{2.0} %en pourcentage
\end{description}

\psset{xunit=1.0cm,yunit=1.0cm,algebraic=true,dimen=middle,dotstyle=o,dotsize=5pt 0,linewidth=2.pt,arrowsize=3pt 2,arrowinset=0.25}
\begin{pspicture*}(-4.3,-3.02)(7.28,6.3)
\multips(0,-3)(0,1.0){10}{\psline[linestyle=dashed,linecap=1,dash=1.5pt 1.5pt,linewidth=0.4pt,linecolor=lightgray]{c-c}(-4.3,0)(7.28,0)}
\multips(-4,0)(1.0,0){12}{\psline[linestyle=dashed,linecap=1,dash=1.5pt 1.5pt,linewidth=0.4pt,linecolor=lightgray]{c-c}(0,-3.02)(0,6.3)}
\psaxes[labelFontSize=\scriptstyle,xAxis=true,yAxis=true,Dx=1.,Dy=1.,ticksize=-2pt 0,subticks=2]{->}(0,0)(-4.3,-3.02)(7.28,6.3)
\psplot[linewidth=2.pt]{-4.3}{7.28}{(-1+-0.2*x)/1.}
%\psplot[linewidth=2.pt]{-4.3}{7.28}{(-2.--1.5*x)/1.}
\begin{scriptsize}
\rput[bl](5.08,5.98){$f$}
\end{scriptsize}
\end{pspicture*}













































\EXERCICE
\enonce{ %début énoncé 

Soit ABC un triangle rectangle en B. On sait que AB = 3 et BC = 3
} % fin énoncé 

\begin{description}
\item[1.] \enonce{ %début énoncé 
Calculer la longueur de AC \points{2.0} %en pourcentage
} % fin énoncé 

\correction{ %début énoncé 
 Le triangle ABC est rectangle en B,, donc avec le théorème de Pythagore on a : 
\begin{eqnarray*}
AB^2 + BC^2 &=& AC^2 \\
3^2 + 3^2 &=& AC^2 \\
18 &=&AC^2 \\
& & \mbox{ Donc : } AC = \sqrt{18} 
\end{eqnarray*}
} % fin correction 

\item[2.] \enonce{ %début énoncé 
Calculer l'angle $\widehat{ABC}$ au degré près. } % fin énoncé 
\points{2.0} %en pourcentage
\end{description}

\psset{xunit=1.0cm,yunit=1.0cm,algebraic=true,dimen=middle,dotstyle=o,dotsize=5pt 0,linewidth=2.pt,arrowsize=3pt 2,arrowinset=0.25}
\begin{pspicture*}(-4.3,-3.02)(7.28,6.3)
\multips(0,-3)(0,1.0){10}{\psline[linestyle=dashed,linecap=1,dash=1.5pt 1.5pt,linewidth=0.4pt,linecolor=lightgray]{c-c}(-4.3,0)(7.28,0)}
\multips(-4,0)(1.0,0){12}{\psline[linestyle=dashed,linecap=1,dash=1.5pt 1.5pt,linewidth=0.4pt,linecolor=lightgray]{c-c}(0,-3.02)(0,6.3)}
\psaxes[labelFontSize=\scriptstyle,xAxis=true,yAxis=true,Dx=1.,Dy=1.,ticksize=-2pt 0,subticks=2]{->}(0,0)(-4.3,-3.02)(7.28,6.3)
\psplot[linewidth=2.pt]{-4.3}{7.28}{(-1+0.0*x)/1.}
%\psplot[linewidth=2.pt]{-4.3}{7.28}{(-2.--1.5*x)/1.}
\begin{scriptsize}
\rput[bl](5.08,5.98){$f$}
\end{scriptsize}
\end{pspicture*}













































\EXERCICE
\enonce{ %début énoncé 

Soit ABC un triangle rectangle en B. On sait que AB = 3 et BC = 3
} % fin énoncé 

\begin{description}
\item[1.] \enonce{ %début énoncé 
Calculer la longueur de AC \points{2.0} %en pourcentage
} % fin énoncé 

\correction{ %début énoncé 
 Le triangle ABC est rectangle en B,, donc avec le théorème de Pythagore on a : 
\begin{eqnarray*}
AB^2 + BC^2 &=& AC^2 \\
3^2 + 3^2 &=& AC^2 \\
18 &=&AC^2 \\
& & \mbox{ Donc : } AC = \sqrt{18} 
\end{eqnarray*}
} % fin correction 

\item[2.] \enonce{ %début énoncé 
Calculer l'angle $\widehat{ABC}$ au degré près. } % fin énoncé 
\points{2.0} %en pourcentage
\end{description}

\psset{xunit=1.0cm,yunit=1.0cm,algebraic=true,dimen=middle,dotstyle=o,dotsize=5pt 0,linewidth=2.pt,arrowsize=3pt 2,arrowinset=0.25}
\begin{pspicture*}(-4.3,-3.02)(7.28,6.3)
\multips(0,-3)(0,1.0){10}{\psline[linestyle=dashed,linecap=1,dash=1.5pt 1.5pt,linewidth=0.4pt,linecolor=lightgray]{c-c}(-4.3,0)(7.28,0)}
\multips(-4,0)(1.0,0){12}{\psline[linestyle=dashed,linecap=1,dash=1.5pt 1.5pt,linewidth=0.4pt,linecolor=lightgray]{c-c}(0,-3.02)(0,6.3)}
\psaxes[labelFontSize=\scriptstyle,xAxis=true,yAxis=true,Dx=1.,Dy=1.,ticksize=-2pt 0,subticks=2]{->}(0,0)(-4.3,-3.02)(7.28,6.3)
\psplot[linewidth=2.pt]{-4.3}{7.28}{(-1+0.2*x)/1.}
%\psplot[linewidth=2.pt]{-4.3}{7.28}{(-2.--1.5*x)/1.}
\begin{scriptsize}
\rput[bl](5.08,5.98){$f$}
\end{scriptsize}
\end{pspicture*}













































\EXERCICE
\enonce{ %début énoncé 

Soit ABC un triangle rectangle en B. On sait que AB = 3 et BC = 3
} % fin énoncé 

\begin{description}
\item[1.] \enonce{ %début énoncé 
Calculer la longueur de AC \points{2.0} %en pourcentage
} % fin énoncé 

\correction{ %début énoncé 
 Le triangle ABC est rectangle en B,, donc avec le théorème de Pythagore on a : 
\begin{eqnarray*}
AB^2 + BC^2 &=& AC^2 \\
3^2 + 3^2 &=& AC^2 \\
18 &=&AC^2 \\
& & \mbox{ Donc : } AC = \sqrt{18} 
\end{eqnarray*}
} % fin correction 

\item[2.] \enonce{ %début énoncé 
Calculer l'angle $\widehat{ABC}$ au degré près. } % fin énoncé 
\points{2.0} %en pourcentage
\end{description}

\psset{xunit=1.0cm,yunit=1.0cm,algebraic=true,dimen=middle,dotstyle=o,dotsize=5pt 0,linewidth=2.pt,arrowsize=3pt 2,arrowinset=0.25}
\begin{pspicture*}(-4.3,-3.02)(7.28,6.3)
\multips(0,-3)(0,1.0){10}{\psline[linestyle=dashed,linecap=1,dash=1.5pt 1.5pt,linewidth=0.4pt,linecolor=lightgray]{c-c}(-4.3,0)(7.28,0)}
\multips(-4,0)(1.0,0){12}{\psline[linestyle=dashed,linecap=1,dash=1.5pt 1.5pt,linewidth=0.4pt,linecolor=lightgray]{c-c}(0,-3.02)(0,6.3)}
\psaxes[labelFontSize=\scriptstyle,xAxis=true,yAxis=true,Dx=1.,Dy=1.,ticksize=-2pt 0,subticks=2]{->}(0,0)(-4.3,-3.02)(7.28,6.3)
\psplot[linewidth=2.pt]{-4.3}{7.28}{(-1+0.4*x)/1.}
%\psplot[linewidth=2.pt]{-4.3}{7.28}{(-2.--1.5*x)/1.}
\begin{scriptsize}
\rput[bl](5.08,5.98){$f$}
\end{scriptsize}
\end{pspicture*}













































\EXERCICE
\enonce{ %début énoncé 

Soit ABC un triangle rectangle en B. On sait que AB = 3 et BC = 3
} % fin énoncé 

\begin{description}
\item[1.] \enonce{ %début énoncé 
Calculer la longueur de AC \points{2.0} %en pourcentage
} % fin énoncé 

\correction{ %début énoncé 
 Le triangle ABC est rectangle en B,, donc avec le théorème de Pythagore on a : 
\begin{eqnarray*}
AB^2 + BC^2 &=& AC^2 \\
3^2 + 3^2 &=& AC^2 \\
18 &=&AC^2 \\
& & \mbox{ Donc : } AC = \sqrt{18} 
\end{eqnarray*}
} % fin correction 

\item[2.] \enonce{ %début énoncé 
Calculer l'angle $\widehat{ABC}$ au degré près. } % fin énoncé 
\points{2.0} %en pourcentage
\end{description}

\psset{xunit=1.0cm,yunit=1.0cm,algebraic=true,dimen=middle,dotstyle=o,dotsize=5pt 0,linewidth=2.pt,arrowsize=3pt 2,arrowinset=0.25}
\begin{pspicture*}(-4.3,-3.02)(7.28,6.3)
\multips(0,-3)(0,1.0){10}{\psline[linestyle=dashed,linecap=1,dash=1.5pt 1.5pt,linewidth=0.4pt,linecolor=lightgray]{c-c}(-4.3,0)(7.28,0)}
\multips(-4,0)(1.0,0){12}{\psline[linestyle=dashed,linecap=1,dash=1.5pt 1.5pt,linewidth=0.4pt,linecolor=lightgray]{c-c}(0,-3.02)(0,6.3)}
\psaxes[labelFontSize=\scriptstyle,xAxis=true,yAxis=true,Dx=1.,Dy=1.,ticksize=-2pt 0,subticks=2]{->}(0,0)(-4.3,-3.02)(7.28,6.3)
\psplot[linewidth=2.pt]{-4.3}{7.28}{(0+-0.4*x)/1.}
%\psplot[linewidth=2.pt]{-4.3}{7.28}{(-2.--1.5*x)/1.}
\begin{scriptsize}
\rput[bl](5.08,5.98){$f$}
\end{scriptsize}
\end{pspicture*}













































\EXERCICE
\enonce{ %début énoncé 

Soit ABC un triangle rectangle en B. On sait que AB = 3 et BC = 3
} % fin énoncé 

\begin{description}
\item[1.] \enonce{ %début énoncé 
Calculer la longueur de AC \points{2.0} %en pourcentage
} % fin énoncé 

\correction{ %début énoncé 
 Le triangle ABC est rectangle en B,, donc avec le théorème de Pythagore on a : 
\begin{eqnarray*}
AB^2 + BC^2 &=& AC^2 \\
3^2 + 3^2 &=& AC^2 \\
18 &=&AC^2 \\
& & \mbox{ Donc : } AC = \sqrt{18} 
\end{eqnarray*}
} % fin correction 

\item[2.] \enonce{ %début énoncé 
Calculer l'angle $\widehat{ABC}$ au degré près. } % fin énoncé 
\points{2.0} %en pourcentage
\end{description}

\psset{xunit=1.0cm,yunit=1.0cm,algebraic=true,dimen=middle,dotstyle=o,dotsize=5pt 0,linewidth=2.pt,arrowsize=3pt 2,arrowinset=0.25}
\begin{pspicture*}(-4.3,-3.02)(7.28,6.3)
\multips(0,-3)(0,1.0){10}{\psline[linestyle=dashed,linecap=1,dash=1.5pt 1.5pt,linewidth=0.4pt,linecolor=lightgray]{c-c}(-4.3,0)(7.28,0)}
\multips(-4,0)(1.0,0){12}{\psline[linestyle=dashed,linecap=1,dash=1.5pt 1.5pt,linewidth=0.4pt,linecolor=lightgray]{c-c}(0,-3.02)(0,6.3)}
\psaxes[labelFontSize=\scriptstyle,xAxis=true,yAxis=true,Dx=1.,Dy=1.,ticksize=-2pt 0,subticks=2]{->}(0,0)(-4.3,-3.02)(7.28,6.3)
\psplot[linewidth=2.pt]{-4.3}{7.28}{(0+-0.2*x)/1.}
%\psplot[linewidth=2.pt]{-4.3}{7.28}{(-2.--1.5*x)/1.}
\begin{scriptsize}
\rput[bl](5.08,5.98){$f$}
\end{scriptsize}
\end{pspicture*}













































\EXERCICE
\enonce{ %début énoncé 

Soit ABC un triangle rectangle en B. On sait que AB = 3 et BC = 3
} % fin énoncé 

\begin{description}
\item[1.] \enonce{ %début énoncé 
Calculer la longueur de AC \points{2.0} %en pourcentage
} % fin énoncé 

\correction{ %début énoncé 
 Le triangle ABC est rectangle en B,, donc avec le théorème de Pythagore on a : 
\begin{eqnarray*}
AB^2 + BC^2 &=& AC^2 \\
3^2 + 3^2 &=& AC^2 \\
18 &=&AC^2 \\
& & \mbox{ Donc : } AC = \sqrt{18} 
\end{eqnarray*}
} % fin correction 

\item[2.] \enonce{ %début énoncé 
Calculer l'angle $\widehat{ABC}$ au degré près. } % fin énoncé 
\points{2.0} %en pourcentage
\end{description}

\psset{xunit=1.0cm,yunit=1.0cm,algebraic=true,dimen=middle,dotstyle=o,dotsize=5pt 0,linewidth=2.pt,arrowsize=3pt 2,arrowinset=0.25}
\begin{pspicture*}(-4.3,-3.02)(7.28,6.3)
\multips(0,-3)(0,1.0){10}{\psline[linestyle=dashed,linecap=1,dash=1.5pt 1.5pt,linewidth=0.4pt,linecolor=lightgray]{c-c}(-4.3,0)(7.28,0)}
\multips(-4,0)(1.0,0){12}{\psline[linestyle=dashed,linecap=1,dash=1.5pt 1.5pt,linewidth=0.4pt,linecolor=lightgray]{c-c}(0,-3.02)(0,6.3)}
\psaxes[labelFontSize=\scriptstyle,xAxis=true,yAxis=true,Dx=1.,Dy=1.,ticksize=-2pt 0,subticks=2]{->}(0,0)(-4.3,-3.02)(7.28,6.3)
\psplot[linewidth=2.pt]{-4.3}{7.28}{(0+0.0*x)/1.}
%\psplot[linewidth=2.pt]{-4.3}{7.28}{(-2.--1.5*x)/1.}
\begin{scriptsize}
\rput[bl](5.08,5.98){$f$}
\end{scriptsize}
\end{pspicture*}













































\EXERCICE
\enonce{ %début énoncé 

Soit ABC un triangle rectangle en B. On sait que AB = 3 et BC = 3
} % fin énoncé 

\begin{description}
\item[1.] \enonce{ %début énoncé 
Calculer la longueur de AC \points{2.0} %en pourcentage
} % fin énoncé 

\correction{ %début énoncé 
 Le triangle ABC est rectangle en B,, donc avec le théorème de Pythagore on a : 
\begin{eqnarray*}
AB^2 + BC^2 &=& AC^2 \\
3^2 + 3^2 &=& AC^2 \\
18 &=&AC^2 \\
& & \mbox{ Donc : } AC = \sqrt{18} 
\end{eqnarray*}
} % fin correction 

\item[2.] \enonce{ %début énoncé 
Calculer l'angle $\widehat{ABC}$ au degré près. } % fin énoncé 
\points{2.0} %en pourcentage
\end{description}

\psset{xunit=1.0cm,yunit=1.0cm,algebraic=true,dimen=middle,dotstyle=o,dotsize=5pt 0,linewidth=2.pt,arrowsize=3pt 2,arrowinset=0.25}
\begin{pspicture*}(-4.3,-3.02)(7.28,6.3)
\multips(0,-3)(0,1.0){10}{\psline[linestyle=dashed,linecap=1,dash=1.5pt 1.5pt,linewidth=0.4pt,linecolor=lightgray]{c-c}(-4.3,0)(7.28,0)}
\multips(-4,0)(1.0,0){12}{\psline[linestyle=dashed,linecap=1,dash=1.5pt 1.5pt,linewidth=0.4pt,linecolor=lightgray]{c-c}(0,-3.02)(0,6.3)}
\psaxes[labelFontSize=\scriptstyle,xAxis=true,yAxis=true,Dx=1.,Dy=1.,ticksize=-2pt 0,subticks=2]{->}(0,0)(-4.3,-3.02)(7.28,6.3)
\psplot[linewidth=2.pt]{-4.3}{7.28}{(0+0.2*x)/1.}
%\psplot[linewidth=2.pt]{-4.3}{7.28}{(-2.--1.5*x)/1.}
\begin{scriptsize}
\rput[bl](5.08,5.98){$f$}
\end{scriptsize}
\end{pspicture*}













































\EXERCICE
\enonce{ %début énoncé 

Soit ABC un triangle rectangle en B. On sait que AB = 3 et BC = 3
} % fin énoncé 

\begin{description}
\item[1.] \enonce{ %début énoncé 
Calculer la longueur de AC \points{2.0} %en pourcentage
} % fin énoncé 

\correction{ %début énoncé 
 Le triangle ABC est rectangle en B,, donc avec le théorème de Pythagore on a : 
\begin{eqnarray*}
AB^2 + BC^2 &=& AC^2 \\
3^2 + 3^2 &=& AC^2 \\
18 &=&AC^2 \\
& & \mbox{ Donc : } AC = \sqrt{18} 
\end{eqnarray*}
} % fin correction 

\item[2.] \enonce{ %début énoncé 
Calculer l'angle $\widehat{ABC}$ au degré près. } % fin énoncé 
\points{2.0} %en pourcentage
\end{description}

\psset{xunit=1.0cm,yunit=1.0cm,algebraic=true,dimen=middle,dotstyle=o,dotsize=5pt 0,linewidth=2.pt,arrowsize=3pt 2,arrowinset=0.25}
\begin{pspicture*}(-4.3,-3.02)(7.28,6.3)
\multips(0,-3)(0,1.0){10}{\psline[linestyle=dashed,linecap=1,dash=1.5pt 1.5pt,linewidth=0.4pt,linecolor=lightgray]{c-c}(-4.3,0)(7.28,0)}
\multips(-4,0)(1.0,0){12}{\psline[linestyle=dashed,linecap=1,dash=1.5pt 1.5pt,linewidth=0.4pt,linecolor=lightgray]{c-c}(0,-3.02)(0,6.3)}
\psaxes[labelFontSize=\scriptstyle,xAxis=true,yAxis=true,Dx=1.,Dy=1.,ticksize=-2pt 0,subticks=2]{->}(0,0)(-4.3,-3.02)(7.28,6.3)
\psplot[linewidth=2.pt]{-4.3}{7.28}{(0+0.4*x)/1.}
%\psplot[linewidth=2.pt]{-4.3}{7.28}{(-2.--1.5*x)/1.}
\begin{scriptsize}
\rput[bl](5.08,5.98){$f$}
\end{scriptsize}
\end{pspicture*}













































\EXERCICE
\enonce{ %début énoncé 

Soit ABC un triangle rectangle en B. On sait que AB = 3 et BC = 3
} % fin énoncé 

\begin{description}
\item[1.] \enonce{ %début énoncé 
Calculer la longueur de AC \points{2.0} %en pourcentage
} % fin énoncé 

\correction{ %début énoncé 
 Le triangle ABC est rectangle en B,, donc avec le théorème de Pythagore on a : 
\begin{eqnarray*}
AB^2 + BC^2 &=& AC^2 \\
3^2 + 3^2 &=& AC^2 \\
18 &=&AC^2 \\
& & \mbox{ Donc : } AC = \sqrt{18} 
\end{eqnarray*}
} % fin correction 

\item[2.] \enonce{ %début énoncé 
Calculer l'angle $\widehat{ABC}$ au degré près. } % fin énoncé 
\points{2.0} %en pourcentage
\end{description}

\psset{xunit=1.0cm,yunit=1.0cm,algebraic=true,dimen=middle,dotstyle=o,dotsize=5pt 0,linewidth=2.pt,arrowsize=3pt 2,arrowinset=0.25}
\begin{pspicture*}(-4.3,-3.02)(7.28,6.3)
\multips(0,-3)(0,1.0){10}{\psline[linestyle=dashed,linecap=1,dash=1.5pt 1.5pt,linewidth=0.4pt,linecolor=lightgray]{c-c}(-4.3,0)(7.28,0)}
\multips(-4,0)(1.0,0){12}{\psline[linestyle=dashed,linecap=1,dash=1.5pt 1.5pt,linewidth=0.4pt,linecolor=lightgray]{c-c}(0,-3.02)(0,6.3)}
\psaxes[labelFontSize=\scriptstyle,xAxis=true,yAxis=true,Dx=1.,Dy=1.,ticksize=-2pt 0,subticks=2]{->}(0,0)(-4.3,-3.02)(7.28,6.3)
\psplot[linewidth=2.pt]{-4.3}{7.28}{(1+-0.4*x)/1.}
%\psplot[linewidth=2.pt]{-4.3}{7.28}{(-2.--1.5*x)/1.}
\begin{scriptsize}
\rput[bl](5.08,5.98){$f$}
\end{scriptsize}
\end{pspicture*}













































\EXERCICE
\enonce{ %début énoncé 

Soit ABC un triangle rectangle en B. On sait que AB = 3 et BC = 3
} % fin énoncé 

\begin{description}
\item[1.] \enonce{ %début énoncé 
Calculer la longueur de AC \points{2.0} %en pourcentage
} % fin énoncé 

\correction{ %début énoncé 
 Le triangle ABC est rectangle en B,, donc avec le théorème de Pythagore on a : 
\begin{eqnarray*}
AB^2 + BC^2 &=& AC^2 \\
3^2 + 3^2 &=& AC^2 \\
18 &=&AC^2 \\
& & \mbox{ Donc : } AC = \sqrt{18} 
\end{eqnarray*}
} % fin correction 

\item[2.] \enonce{ %début énoncé 
Calculer l'angle $\widehat{ABC}$ au degré près. } % fin énoncé 
\points{2.0} %en pourcentage
\end{description}

\psset{xunit=1.0cm,yunit=1.0cm,algebraic=true,dimen=middle,dotstyle=o,dotsize=5pt 0,linewidth=2.pt,arrowsize=3pt 2,arrowinset=0.25}
\begin{pspicture*}(-4.3,-3.02)(7.28,6.3)
\multips(0,-3)(0,1.0){10}{\psline[linestyle=dashed,linecap=1,dash=1.5pt 1.5pt,linewidth=0.4pt,linecolor=lightgray]{c-c}(-4.3,0)(7.28,0)}
\multips(-4,0)(1.0,0){12}{\psline[linestyle=dashed,linecap=1,dash=1.5pt 1.5pt,linewidth=0.4pt,linecolor=lightgray]{c-c}(0,-3.02)(0,6.3)}
\psaxes[labelFontSize=\scriptstyle,xAxis=true,yAxis=true,Dx=1.,Dy=1.,ticksize=-2pt 0,subticks=2]{->}(0,0)(-4.3,-3.02)(7.28,6.3)
\psplot[linewidth=2.pt]{-4.3}{7.28}{(1+-0.2*x)/1.}
%\psplot[linewidth=2.pt]{-4.3}{7.28}{(-2.--1.5*x)/1.}
\begin{scriptsize}
\rput[bl](5.08,5.98){$f$}
\end{scriptsize}
\end{pspicture*}













































\EXERCICE
\enonce{ %début énoncé 

Soit ABC un triangle rectangle en B. On sait que AB = 3 et BC = 3
} % fin énoncé 

\begin{description}
\item[1.] \enonce{ %début énoncé 
Calculer la longueur de AC \points{2.0} %en pourcentage
} % fin énoncé 

\correction{ %début énoncé 
 Le triangle ABC est rectangle en B,, donc avec le théorème de Pythagore on a : 
\begin{eqnarray*}
AB^2 + BC^2 &=& AC^2 \\
3^2 + 3^2 &=& AC^2 \\
18 &=&AC^2 \\
& & \mbox{ Donc : } AC = \sqrt{18} 
\end{eqnarray*}
} % fin correction 

\item[2.] \enonce{ %début énoncé 
Calculer l'angle $\widehat{ABC}$ au degré près. } % fin énoncé 
\points{2.0} %en pourcentage
\end{description}

\psset{xunit=1.0cm,yunit=1.0cm,algebraic=true,dimen=middle,dotstyle=o,dotsize=5pt 0,linewidth=2.pt,arrowsize=3pt 2,arrowinset=0.25}
\begin{pspicture*}(-4.3,-3.02)(7.28,6.3)
\multips(0,-3)(0,1.0){10}{\psline[linestyle=dashed,linecap=1,dash=1.5pt 1.5pt,linewidth=0.4pt,linecolor=lightgray]{c-c}(-4.3,0)(7.28,0)}
\multips(-4,0)(1.0,0){12}{\psline[linestyle=dashed,linecap=1,dash=1.5pt 1.5pt,linewidth=0.4pt,linecolor=lightgray]{c-c}(0,-3.02)(0,6.3)}
\psaxes[labelFontSize=\scriptstyle,xAxis=true,yAxis=true,Dx=1.,Dy=1.,ticksize=-2pt 0,subticks=2]{->}(0,0)(-4.3,-3.02)(7.28,6.3)
\psplot[linewidth=2.pt]{-4.3}{7.28}{(1+0.0*x)/1.}
%\psplot[linewidth=2.pt]{-4.3}{7.28}{(-2.--1.5*x)/1.}
\begin{scriptsize}
\rput[bl](5.08,5.98){$f$}
\end{scriptsize}
\end{pspicture*}













































\EXERCICE
\enonce{ %début énoncé 

Soit ABC un triangle rectangle en B. On sait que AB = 3 et BC = 3
} % fin énoncé 

\begin{description}
\item[1.] \enonce{ %début énoncé 
Calculer la longueur de AC \points{2.0} %en pourcentage
} % fin énoncé 

\correction{ %début énoncé 
 Le triangle ABC est rectangle en B,, donc avec le théorème de Pythagore on a : 
\begin{eqnarray*}
AB^2 + BC^2 &=& AC^2 \\
3^2 + 3^2 &=& AC^2 \\
18 &=&AC^2 \\
& & \mbox{ Donc : } AC = \sqrt{18} 
\end{eqnarray*}
} % fin correction 

\item[2.] \enonce{ %début énoncé 
Calculer l'angle $\widehat{ABC}$ au degré près. } % fin énoncé 
\points{2.0} %en pourcentage
\end{description}

\psset{xunit=1.0cm,yunit=1.0cm,algebraic=true,dimen=middle,dotstyle=o,dotsize=5pt 0,linewidth=2.pt,arrowsize=3pt 2,arrowinset=0.25}
\begin{pspicture*}(-4.3,-3.02)(7.28,6.3)
\multips(0,-3)(0,1.0){10}{\psline[linestyle=dashed,linecap=1,dash=1.5pt 1.5pt,linewidth=0.4pt,linecolor=lightgray]{c-c}(-4.3,0)(7.28,0)}
\multips(-4,0)(1.0,0){12}{\psline[linestyle=dashed,linecap=1,dash=1.5pt 1.5pt,linewidth=0.4pt,linecolor=lightgray]{c-c}(0,-3.02)(0,6.3)}
\psaxes[labelFontSize=\scriptstyle,xAxis=true,yAxis=true,Dx=1.,Dy=1.,ticksize=-2pt 0,subticks=2]{->}(0,0)(-4.3,-3.02)(7.28,6.3)
\psplot[linewidth=2.pt]{-4.3}{7.28}{(1+0.2*x)/1.}
%\psplot[linewidth=2.pt]{-4.3}{7.28}{(-2.--1.5*x)/1.}
\begin{scriptsize}
\rput[bl](5.08,5.98){$f$}
\end{scriptsize}
\end{pspicture*}













































\EXERCICE
\enonce{ %début énoncé 

Soit ABC un triangle rectangle en B. On sait que AB = 3 et BC = 3
} % fin énoncé 

\begin{description}
\item[1.] \enonce{ %début énoncé 
Calculer la longueur de AC \points{2.0} %en pourcentage
} % fin énoncé 

\correction{ %début énoncé 
 Le triangle ABC est rectangle en B,, donc avec le théorème de Pythagore on a : 
\begin{eqnarray*}
AB^2 + BC^2 &=& AC^2 \\
3^2 + 3^2 &=& AC^2 \\
18 &=&AC^2 \\
& & \mbox{ Donc : } AC = \sqrt{18} 
\end{eqnarray*}
} % fin correction 

\item[2.] \enonce{ %début énoncé 
Calculer l'angle $\widehat{ABC}$ au degré près. } % fin énoncé 
\points{2.0} %en pourcentage
\end{description}

\psset{xunit=1.0cm,yunit=1.0cm,algebraic=true,dimen=middle,dotstyle=o,dotsize=5pt 0,linewidth=2.pt,arrowsize=3pt 2,arrowinset=0.25}
\begin{pspicture*}(-4.3,-3.02)(7.28,6.3)
\multips(0,-3)(0,1.0){10}{\psline[linestyle=dashed,linecap=1,dash=1.5pt 1.5pt,linewidth=0.4pt,linecolor=lightgray]{c-c}(-4.3,0)(7.28,0)}
\multips(-4,0)(1.0,0){12}{\psline[linestyle=dashed,linecap=1,dash=1.5pt 1.5pt,linewidth=0.4pt,linecolor=lightgray]{c-c}(0,-3.02)(0,6.3)}
\psaxes[labelFontSize=\scriptstyle,xAxis=true,yAxis=true,Dx=1.,Dy=1.,ticksize=-2pt 0,subticks=2]{->}(0,0)(-4.3,-3.02)(7.28,6.3)
\psplot[linewidth=2.pt]{-4.3}{7.28}{(1+0.4*x)/1.}
%\psplot[linewidth=2.pt]{-4.3}{7.28}{(-2.--1.5*x)/1.}
\begin{scriptsize}
\rput[bl](5.08,5.98){$f$}
\end{scriptsize}
\end{pspicture*}













































\EXERCICE
\enonce{ %début énoncé 

Soit ABC un triangle rectangle en B. On sait que AB = 4 et BC = 3
} % fin énoncé 

\begin{description}
\item[1.] \enonce{ %début énoncé 
Calculer la longueur de AC \points{2.0} %en pourcentage
} % fin énoncé 

\correction{ %début énoncé 
 Le triangle ABC est rectangle en B,, donc avec le théorème de Pythagore on a : 
\begin{eqnarray*}
AB^2 + BC^2 &=& AC^2 \\
4^2 + 3^2 &=& AC^2 \\
25 &=&AC^2 \\
& & \mbox{ Donc : } AC = \sqrt{25} 
\end{eqnarray*}
} % fin correction 

\item[2.] \enonce{ %début énoncé 
Calculer l'angle $\widehat{ABC}$ au degré près. } % fin énoncé 
\points{2.0} %en pourcentage
\end{description}

\psset{xunit=1.0cm,yunit=1.0cm,algebraic=true,dimen=middle,dotstyle=o,dotsize=5pt 0,linewidth=2.pt,arrowsize=3pt 2,arrowinset=0.25}
\begin{pspicture*}(-4.3,-3.02)(7.28,6.3)
\multips(0,-3)(0,1.0){10}{\psline[linestyle=dashed,linecap=1,dash=1.5pt 1.5pt,linewidth=0.4pt,linecolor=lightgray]{c-c}(-4.3,0)(7.28,0)}
\multips(-4,0)(1.0,0){12}{\psline[linestyle=dashed,linecap=1,dash=1.5pt 1.5pt,linewidth=0.4pt,linecolor=lightgray]{c-c}(0,-3.02)(0,6.3)}
\psaxes[labelFontSize=\scriptstyle,xAxis=true,yAxis=true,Dx=1.,Dy=1.,ticksize=-2pt 0,subticks=2]{->}(0,0)(-4.3,-3.02)(7.28,6.3)
\psplot[linewidth=2.pt]{-4.3}{7.28}{(-1+-0.4*x)/1.}
%\psplot[linewidth=2.pt]{-4.3}{7.28}{(-2.--1.5*x)/1.}
\begin{scriptsize}
\rput[bl](5.08,5.98){$f$}
\end{scriptsize}
\end{pspicture*}













































\EXERCICE
\enonce{ %début énoncé 

Soit ABC un triangle rectangle en B. On sait que AB = 4 et BC = 3
} % fin énoncé 

\begin{description}
\item[1.] \enonce{ %début énoncé 
Calculer la longueur de AC \points{2.0} %en pourcentage
} % fin énoncé 

\correction{ %début énoncé 
 Le triangle ABC est rectangle en B,, donc avec le théorème de Pythagore on a : 
\begin{eqnarray*}
AB^2 + BC^2 &=& AC^2 \\
4^2 + 3^2 &=& AC^2 \\
25 &=&AC^2 \\
& & \mbox{ Donc : } AC = \sqrt{25} 
\end{eqnarray*}
} % fin correction 

\item[2.] \enonce{ %début énoncé 
Calculer l'angle $\widehat{ABC}$ au degré près. } % fin énoncé 
\points{2.0} %en pourcentage
\end{description}

\psset{xunit=1.0cm,yunit=1.0cm,algebraic=true,dimen=middle,dotstyle=o,dotsize=5pt 0,linewidth=2.pt,arrowsize=3pt 2,arrowinset=0.25}
\begin{pspicture*}(-4.3,-3.02)(7.28,6.3)
\multips(0,-3)(0,1.0){10}{\psline[linestyle=dashed,linecap=1,dash=1.5pt 1.5pt,linewidth=0.4pt,linecolor=lightgray]{c-c}(-4.3,0)(7.28,0)}
\multips(-4,0)(1.0,0){12}{\psline[linestyle=dashed,linecap=1,dash=1.5pt 1.5pt,linewidth=0.4pt,linecolor=lightgray]{c-c}(0,-3.02)(0,6.3)}
\psaxes[labelFontSize=\scriptstyle,xAxis=true,yAxis=true,Dx=1.,Dy=1.,ticksize=-2pt 0,subticks=2]{->}(0,0)(-4.3,-3.02)(7.28,6.3)
\psplot[linewidth=2.pt]{-4.3}{7.28}{(-1+-0.2*x)/1.}
%\psplot[linewidth=2.pt]{-4.3}{7.28}{(-2.--1.5*x)/1.}
\begin{scriptsize}
\rput[bl](5.08,5.98){$f$}
\end{scriptsize}
\end{pspicture*}













































\EXERCICE
\enonce{ %début énoncé 

Soit ABC un triangle rectangle en B. On sait que AB = 4 et BC = 3
} % fin énoncé 

\begin{description}
\item[1.] \enonce{ %début énoncé 
Calculer la longueur de AC \points{2.0} %en pourcentage
} % fin énoncé 

\correction{ %début énoncé 
 Le triangle ABC est rectangle en B,, donc avec le théorème de Pythagore on a : 
\begin{eqnarray*}
AB^2 + BC^2 &=& AC^2 \\
4^2 + 3^2 &=& AC^2 \\
25 &=&AC^2 \\
& & \mbox{ Donc : } AC = \sqrt{25} 
\end{eqnarray*}
} % fin correction 

\item[2.] \enonce{ %début énoncé 
Calculer l'angle $\widehat{ABC}$ au degré près. } % fin énoncé 
\points{2.0} %en pourcentage
\end{description}

\psset{xunit=1.0cm,yunit=1.0cm,algebraic=true,dimen=middle,dotstyle=o,dotsize=5pt 0,linewidth=2.pt,arrowsize=3pt 2,arrowinset=0.25}
\begin{pspicture*}(-4.3,-3.02)(7.28,6.3)
\multips(0,-3)(0,1.0){10}{\psline[linestyle=dashed,linecap=1,dash=1.5pt 1.5pt,linewidth=0.4pt,linecolor=lightgray]{c-c}(-4.3,0)(7.28,0)}
\multips(-4,0)(1.0,0){12}{\psline[linestyle=dashed,linecap=1,dash=1.5pt 1.5pt,linewidth=0.4pt,linecolor=lightgray]{c-c}(0,-3.02)(0,6.3)}
\psaxes[labelFontSize=\scriptstyle,xAxis=true,yAxis=true,Dx=1.,Dy=1.,ticksize=-2pt 0,subticks=2]{->}(0,0)(-4.3,-3.02)(7.28,6.3)
\psplot[linewidth=2.pt]{-4.3}{7.28}{(-1+0.0*x)/1.}
%\psplot[linewidth=2.pt]{-4.3}{7.28}{(-2.--1.5*x)/1.}
\begin{scriptsize}
\rput[bl](5.08,5.98){$f$}
\end{scriptsize}
\end{pspicture*}













































\EXERCICE
\enonce{ %début énoncé 

Soit ABC un triangle rectangle en B. On sait que AB = 4 et BC = 3
} % fin énoncé 

\begin{description}
\item[1.] \enonce{ %début énoncé 
Calculer la longueur de AC \points{2.0} %en pourcentage
} % fin énoncé 

\correction{ %début énoncé 
 Le triangle ABC est rectangle en B,, donc avec le théorème de Pythagore on a : 
\begin{eqnarray*}
AB^2 + BC^2 &=& AC^2 \\
4^2 + 3^2 &=& AC^2 \\
25 &=&AC^2 \\
& & \mbox{ Donc : } AC = \sqrt{25} 
\end{eqnarray*}
} % fin correction 

\item[2.] \enonce{ %début énoncé 
Calculer l'angle $\widehat{ABC}$ au degré près. } % fin énoncé 
\points{2.0} %en pourcentage
\end{description}

\psset{xunit=1.0cm,yunit=1.0cm,algebraic=true,dimen=middle,dotstyle=o,dotsize=5pt 0,linewidth=2.pt,arrowsize=3pt 2,arrowinset=0.25}
\begin{pspicture*}(-4.3,-3.02)(7.28,6.3)
\multips(0,-3)(0,1.0){10}{\psline[linestyle=dashed,linecap=1,dash=1.5pt 1.5pt,linewidth=0.4pt,linecolor=lightgray]{c-c}(-4.3,0)(7.28,0)}
\multips(-4,0)(1.0,0){12}{\psline[linestyle=dashed,linecap=1,dash=1.5pt 1.5pt,linewidth=0.4pt,linecolor=lightgray]{c-c}(0,-3.02)(0,6.3)}
\psaxes[labelFontSize=\scriptstyle,xAxis=true,yAxis=true,Dx=1.,Dy=1.,ticksize=-2pt 0,subticks=2]{->}(0,0)(-4.3,-3.02)(7.28,6.3)
\psplot[linewidth=2.pt]{-4.3}{7.28}{(-1+0.2*x)/1.}
%\psplot[linewidth=2.pt]{-4.3}{7.28}{(-2.--1.5*x)/1.}
\begin{scriptsize}
\rput[bl](5.08,5.98){$f$}
\end{scriptsize}
\end{pspicture*}













































\EXERCICE
\enonce{ %début énoncé 

Soit ABC un triangle rectangle en B. On sait que AB = 4 et BC = 3
} % fin énoncé 

\begin{description}
\item[1.] \enonce{ %début énoncé 
Calculer la longueur de AC \points{2.0} %en pourcentage
} % fin énoncé 

\correction{ %début énoncé 
 Le triangle ABC est rectangle en B,, donc avec le théorème de Pythagore on a : 
\begin{eqnarray*}
AB^2 + BC^2 &=& AC^2 \\
4^2 + 3^2 &=& AC^2 \\
25 &=&AC^2 \\
& & \mbox{ Donc : } AC = \sqrt{25} 
\end{eqnarray*}
} % fin correction 

\item[2.] \enonce{ %début énoncé 
Calculer l'angle $\widehat{ABC}$ au degré près. } % fin énoncé 
\points{2.0} %en pourcentage
\end{description}

\psset{xunit=1.0cm,yunit=1.0cm,algebraic=true,dimen=middle,dotstyle=o,dotsize=5pt 0,linewidth=2.pt,arrowsize=3pt 2,arrowinset=0.25}
\begin{pspicture*}(-4.3,-3.02)(7.28,6.3)
\multips(0,-3)(0,1.0){10}{\psline[linestyle=dashed,linecap=1,dash=1.5pt 1.5pt,linewidth=0.4pt,linecolor=lightgray]{c-c}(-4.3,0)(7.28,0)}
\multips(-4,0)(1.0,0){12}{\psline[linestyle=dashed,linecap=1,dash=1.5pt 1.5pt,linewidth=0.4pt,linecolor=lightgray]{c-c}(0,-3.02)(0,6.3)}
\psaxes[labelFontSize=\scriptstyle,xAxis=true,yAxis=true,Dx=1.,Dy=1.,ticksize=-2pt 0,subticks=2]{->}(0,0)(-4.3,-3.02)(7.28,6.3)
\psplot[linewidth=2.pt]{-4.3}{7.28}{(-1+0.4*x)/1.}
%\psplot[linewidth=2.pt]{-4.3}{7.28}{(-2.--1.5*x)/1.}
\begin{scriptsize}
\rput[bl](5.08,5.98){$f$}
\end{scriptsize}
\end{pspicture*}













































\EXERCICE
\enonce{ %début énoncé 

Soit ABC un triangle rectangle en B. On sait que AB = 4 et BC = 3
} % fin énoncé 

\begin{description}
\item[1.] \enonce{ %début énoncé 
Calculer la longueur de AC \points{2.0} %en pourcentage
} % fin énoncé 

\correction{ %début énoncé 
 Le triangle ABC est rectangle en B,, donc avec le théorème de Pythagore on a : 
\begin{eqnarray*}
AB^2 + BC^2 &=& AC^2 \\
4^2 + 3^2 &=& AC^2 \\
25 &=&AC^2 \\
& & \mbox{ Donc : } AC = \sqrt{25} 
\end{eqnarray*}
} % fin correction 

\item[2.] \enonce{ %début énoncé 
Calculer l'angle $\widehat{ABC}$ au degré près. } % fin énoncé 
\points{2.0} %en pourcentage
\end{description}

\psset{xunit=1.0cm,yunit=1.0cm,algebraic=true,dimen=middle,dotstyle=o,dotsize=5pt 0,linewidth=2.pt,arrowsize=3pt 2,arrowinset=0.25}
\begin{pspicture*}(-4.3,-3.02)(7.28,6.3)
\multips(0,-3)(0,1.0){10}{\psline[linestyle=dashed,linecap=1,dash=1.5pt 1.5pt,linewidth=0.4pt,linecolor=lightgray]{c-c}(-4.3,0)(7.28,0)}
\multips(-4,0)(1.0,0){12}{\psline[linestyle=dashed,linecap=1,dash=1.5pt 1.5pt,linewidth=0.4pt,linecolor=lightgray]{c-c}(0,-3.02)(0,6.3)}
\psaxes[labelFontSize=\scriptstyle,xAxis=true,yAxis=true,Dx=1.,Dy=1.,ticksize=-2pt 0,subticks=2]{->}(0,0)(-4.3,-3.02)(7.28,6.3)
\psplot[linewidth=2.pt]{-4.3}{7.28}{(0+-0.4*x)/1.}
%\psplot[linewidth=2.pt]{-4.3}{7.28}{(-2.--1.5*x)/1.}
\begin{scriptsize}
\rput[bl](5.08,5.98){$f$}
\end{scriptsize}
\end{pspicture*}













































\EXERCICE
\enonce{ %début énoncé 

Soit ABC un triangle rectangle en B. On sait que AB = 4 et BC = 3
} % fin énoncé 

\begin{description}
\item[1.] \enonce{ %début énoncé 
Calculer la longueur de AC \points{2.0} %en pourcentage
} % fin énoncé 

\correction{ %début énoncé 
 Le triangle ABC est rectangle en B,, donc avec le théorème de Pythagore on a : 
\begin{eqnarray*}
AB^2 + BC^2 &=& AC^2 \\
4^2 + 3^2 &=& AC^2 \\
25 &=&AC^2 \\
& & \mbox{ Donc : } AC = \sqrt{25} 
\end{eqnarray*}
} % fin correction 

\item[2.] \enonce{ %début énoncé 
Calculer l'angle $\widehat{ABC}$ au degré près. } % fin énoncé 
\points{2.0} %en pourcentage
\end{description}

\psset{xunit=1.0cm,yunit=1.0cm,algebraic=true,dimen=middle,dotstyle=o,dotsize=5pt 0,linewidth=2.pt,arrowsize=3pt 2,arrowinset=0.25}
\begin{pspicture*}(-4.3,-3.02)(7.28,6.3)
\multips(0,-3)(0,1.0){10}{\psline[linestyle=dashed,linecap=1,dash=1.5pt 1.5pt,linewidth=0.4pt,linecolor=lightgray]{c-c}(-4.3,0)(7.28,0)}
\multips(-4,0)(1.0,0){12}{\psline[linestyle=dashed,linecap=1,dash=1.5pt 1.5pt,linewidth=0.4pt,linecolor=lightgray]{c-c}(0,-3.02)(0,6.3)}
\psaxes[labelFontSize=\scriptstyle,xAxis=true,yAxis=true,Dx=1.,Dy=1.,ticksize=-2pt 0,subticks=2]{->}(0,0)(-4.3,-3.02)(7.28,6.3)
\psplot[linewidth=2.pt]{-4.3}{7.28}{(0+-0.2*x)/1.}
%\psplot[linewidth=2.pt]{-4.3}{7.28}{(-2.--1.5*x)/1.}
\begin{scriptsize}
\rput[bl](5.08,5.98){$f$}
\end{scriptsize}
\end{pspicture*}













































\EXERCICE
\enonce{ %début énoncé 

Soit ABC un triangle rectangle en B. On sait que AB = 4 et BC = 3
} % fin énoncé 

\begin{description}
\item[1.] \enonce{ %début énoncé 
Calculer la longueur de AC \points{2.0} %en pourcentage
} % fin énoncé 

\correction{ %début énoncé 
 Le triangle ABC est rectangle en B,, donc avec le théorème de Pythagore on a : 
\begin{eqnarray*}
AB^2 + BC^2 &=& AC^2 \\
4^2 + 3^2 &=& AC^2 \\
25 &=&AC^2 \\
& & \mbox{ Donc : } AC = \sqrt{25} 
\end{eqnarray*}
} % fin correction 

\item[2.] \enonce{ %début énoncé 
Calculer l'angle $\widehat{ABC}$ au degré près. } % fin énoncé 
\points{2.0} %en pourcentage
\end{description}

\psset{xunit=1.0cm,yunit=1.0cm,algebraic=true,dimen=middle,dotstyle=o,dotsize=5pt 0,linewidth=2.pt,arrowsize=3pt 2,arrowinset=0.25}
\begin{pspicture*}(-4.3,-3.02)(7.28,6.3)
\multips(0,-3)(0,1.0){10}{\psline[linestyle=dashed,linecap=1,dash=1.5pt 1.5pt,linewidth=0.4pt,linecolor=lightgray]{c-c}(-4.3,0)(7.28,0)}
\multips(-4,0)(1.0,0){12}{\psline[linestyle=dashed,linecap=1,dash=1.5pt 1.5pt,linewidth=0.4pt,linecolor=lightgray]{c-c}(0,-3.02)(0,6.3)}
\psaxes[labelFontSize=\scriptstyle,xAxis=true,yAxis=true,Dx=1.,Dy=1.,ticksize=-2pt 0,subticks=2]{->}(0,0)(-4.3,-3.02)(7.28,6.3)
\psplot[linewidth=2.pt]{-4.3}{7.28}{(0+0.0*x)/1.}
%\psplot[linewidth=2.pt]{-4.3}{7.28}{(-2.--1.5*x)/1.}
\begin{scriptsize}
\rput[bl](5.08,5.98){$f$}
\end{scriptsize}
\end{pspicture*}













































\EXERCICE
\enonce{ %début énoncé 

Soit ABC un triangle rectangle en B. On sait que AB = 4 et BC = 3
} % fin énoncé 

\begin{description}
\item[1.] \enonce{ %début énoncé 
Calculer la longueur de AC \points{2.0} %en pourcentage
} % fin énoncé 

\correction{ %début énoncé 
 Le triangle ABC est rectangle en B,, donc avec le théorème de Pythagore on a : 
\begin{eqnarray*}
AB^2 + BC^2 &=& AC^2 \\
4^2 + 3^2 &=& AC^2 \\
25 &=&AC^2 \\
& & \mbox{ Donc : } AC = \sqrt{25} 
\end{eqnarray*}
} % fin correction 

\item[2.] \enonce{ %début énoncé 
Calculer l'angle $\widehat{ABC}$ au degré près. } % fin énoncé 
\points{2.0} %en pourcentage
\end{description}

\psset{xunit=1.0cm,yunit=1.0cm,algebraic=true,dimen=middle,dotstyle=o,dotsize=5pt 0,linewidth=2.pt,arrowsize=3pt 2,arrowinset=0.25}
\begin{pspicture*}(-4.3,-3.02)(7.28,6.3)
\multips(0,-3)(0,1.0){10}{\psline[linestyle=dashed,linecap=1,dash=1.5pt 1.5pt,linewidth=0.4pt,linecolor=lightgray]{c-c}(-4.3,0)(7.28,0)}
\multips(-4,0)(1.0,0){12}{\psline[linestyle=dashed,linecap=1,dash=1.5pt 1.5pt,linewidth=0.4pt,linecolor=lightgray]{c-c}(0,-3.02)(0,6.3)}
\psaxes[labelFontSize=\scriptstyle,xAxis=true,yAxis=true,Dx=1.,Dy=1.,ticksize=-2pt 0,subticks=2]{->}(0,0)(-4.3,-3.02)(7.28,6.3)
\psplot[linewidth=2.pt]{-4.3}{7.28}{(0+0.2*x)/1.}
%\psplot[linewidth=2.pt]{-4.3}{7.28}{(-2.--1.5*x)/1.}
\begin{scriptsize}
\rput[bl](5.08,5.98){$f$}
\end{scriptsize}
\end{pspicture*}













































\EXERCICE
\enonce{ %début énoncé 

Soit ABC un triangle rectangle en B. On sait que AB = 4 et BC = 3
} % fin énoncé 

\begin{description}
\item[1.] \enonce{ %début énoncé 
Calculer la longueur de AC \points{2.0} %en pourcentage
} % fin énoncé 

\correction{ %début énoncé 
 Le triangle ABC est rectangle en B,, donc avec le théorème de Pythagore on a : 
\begin{eqnarray*}
AB^2 + BC^2 &=& AC^2 \\
4^2 + 3^2 &=& AC^2 \\
25 &=&AC^2 \\
& & \mbox{ Donc : } AC = \sqrt{25} 
\end{eqnarray*}
} % fin correction 

\item[2.] \enonce{ %début énoncé 
Calculer l'angle $\widehat{ABC}$ au degré près. } % fin énoncé 
\points{2.0} %en pourcentage
\end{description}

\psset{xunit=1.0cm,yunit=1.0cm,algebraic=true,dimen=middle,dotstyle=o,dotsize=5pt 0,linewidth=2.pt,arrowsize=3pt 2,arrowinset=0.25}
\begin{pspicture*}(-4.3,-3.02)(7.28,6.3)
\multips(0,-3)(0,1.0){10}{\psline[linestyle=dashed,linecap=1,dash=1.5pt 1.5pt,linewidth=0.4pt,linecolor=lightgray]{c-c}(-4.3,0)(7.28,0)}
\multips(-4,0)(1.0,0){12}{\psline[linestyle=dashed,linecap=1,dash=1.5pt 1.5pt,linewidth=0.4pt,linecolor=lightgray]{c-c}(0,-3.02)(0,6.3)}
\psaxes[labelFontSize=\scriptstyle,xAxis=true,yAxis=true,Dx=1.,Dy=1.,ticksize=-2pt 0,subticks=2]{->}(0,0)(-4.3,-3.02)(7.28,6.3)
\psplot[linewidth=2.pt]{-4.3}{7.28}{(0+0.4*x)/1.}
%\psplot[linewidth=2.pt]{-4.3}{7.28}{(-2.--1.5*x)/1.}
\begin{scriptsize}
\rput[bl](5.08,5.98){$f$}
\end{scriptsize}
\end{pspicture*}













































\EXERCICE
\enonce{ %début énoncé 

Soit ABC un triangle rectangle en B. On sait que AB = 4 et BC = 3
} % fin énoncé 

\begin{description}
\item[1.] \enonce{ %début énoncé 
Calculer la longueur de AC \points{2.0} %en pourcentage
} % fin énoncé 

\correction{ %début énoncé 
 Le triangle ABC est rectangle en B,, donc avec le théorème de Pythagore on a : 
\begin{eqnarray*}
AB^2 + BC^2 &=& AC^2 \\
4^2 + 3^2 &=& AC^2 \\
25 &=&AC^2 \\
& & \mbox{ Donc : } AC = \sqrt{25} 
\end{eqnarray*}
} % fin correction 

\item[2.] \enonce{ %début énoncé 
Calculer l'angle $\widehat{ABC}$ au degré près. } % fin énoncé 
\points{2.0} %en pourcentage
\end{description}

\psset{xunit=1.0cm,yunit=1.0cm,algebraic=true,dimen=middle,dotstyle=o,dotsize=5pt 0,linewidth=2.pt,arrowsize=3pt 2,arrowinset=0.25}
\begin{pspicture*}(-4.3,-3.02)(7.28,6.3)
\multips(0,-3)(0,1.0){10}{\psline[linestyle=dashed,linecap=1,dash=1.5pt 1.5pt,linewidth=0.4pt,linecolor=lightgray]{c-c}(-4.3,0)(7.28,0)}
\multips(-4,0)(1.0,0){12}{\psline[linestyle=dashed,linecap=1,dash=1.5pt 1.5pt,linewidth=0.4pt,linecolor=lightgray]{c-c}(0,-3.02)(0,6.3)}
\psaxes[labelFontSize=\scriptstyle,xAxis=true,yAxis=true,Dx=1.,Dy=1.,ticksize=-2pt 0,subticks=2]{->}(0,0)(-4.3,-3.02)(7.28,6.3)
\psplot[linewidth=2.pt]{-4.3}{7.28}{(1+-0.4*x)/1.}
%\psplot[linewidth=2.pt]{-4.3}{7.28}{(-2.--1.5*x)/1.}
\begin{scriptsize}
\rput[bl](5.08,5.98){$f$}
\end{scriptsize}
\end{pspicture*}













































\EXERCICE
\enonce{ %début énoncé 

Soit ABC un triangle rectangle en B. On sait que AB = 4 et BC = 3
} % fin énoncé 

\begin{description}
\item[1.] \enonce{ %début énoncé 
Calculer la longueur de AC \points{2.0} %en pourcentage
} % fin énoncé 

\correction{ %début énoncé 
 Le triangle ABC est rectangle en B,, donc avec le théorème de Pythagore on a : 
\begin{eqnarray*}
AB^2 + BC^2 &=& AC^2 \\
4^2 + 3^2 &=& AC^2 \\
25 &=&AC^2 \\
& & \mbox{ Donc : } AC = \sqrt{25} 
\end{eqnarray*}
} % fin correction 

\item[2.] \enonce{ %début énoncé 
Calculer l'angle $\widehat{ABC}$ au degré près. } % fin énoncé 
\points{2.0} %en pourcentage
\end{description}

\psset{xunit=1.0cm,yunit=1.0cm,algebraic=true,dimen=middle,dotstyle=o,dotsize=5pt 0,linewidth=2.pt,arrowsize=3pt 2,arrowinset=0.25}
\begin{pspicture*}(-4.3,-3.02)(7.28,6.3)
\multips(0,-3)(0,1.0){10}{\psline[linestyle=dashed,linecap=1,dash=1.5pt 1.5pt,linewidth=0.4pt,linecolor=lightgray]{c-c}(-4.3,0)(7.28,0)}
\multips(-4,0)(1.0,0){12}{\psline[linestyle=dashed,linecap=1,dash=1.5pt 1.5pt,linewidth=0.4pt,linecolor=lightgray]{c-c}(0,-3.02)(0,6.3)}
\psaxes[labelFontSize=\scriptstyle,xAxis=true,yAxis=true,Dx=1.,Dy=1.,ticksize=-2pt 0,subticks=2]{->}(0,0)(-4.3,-3.02)(7.28,6.3)
\psplot[linewidth=2.pt]{-4.3}{7.28}{(1+-0.2*x)/1.}
%\psplot[linewidth=2.pt]{-4.3}{7.28}{(-2.--1.5*x)/1.}
\begin{scriptsize}
\rput[bl](5.08,5.98){$f$}
\end{scriptsize}
\end{pspicture*}













































\EXERCICE
\enonce{ %début énoncé 

Soit ABC un triangle rectangle en B. On sait que AB = 4 et BC = 3
} % fin énoncé 

\begin{description}
\item[1.] \enonce{ %début énoncé 
Calculer la longueur de AC \points{2.0} %en pourcentage
} % fin énoncé 

\correction{ %début énoncé 
 Le triangle ABC est rectangle en B,, donc avec le théorème de Pythagore on a : 
\begin{eqnarray*}
AB^2 + BC^2 &=& AC^2 \\
4^2 + 3^2 &=& AC^2 \\
25 &=&AC^2 \\
& & \mbox{ Donc : } AC = \sqrt{25} 
\end{eqnarray*}
} % fin correction 

\item[2.] \enonce{ %début énoncé 
Calculer l'angle $\widehat{ABC}$ au degré près. } % fin énoncé 
\points{2.0} %en pourcentage
\end{description}

\psset{xunit=1.0cm,yunit=1.0cm,algebraic=true,dimen=middle,dotstyle=o,dotsize=5pt 0,linewidth=2.pt,arrowsize=3pt 2,arrowinset=0.25}
\begin{pspicture*}(-4.3,-3.02)(7.28,6.3)
\multips(0,-3)(0,1.0){10}{\psline[linestyle=dashed,linecap=1,dash=1.5pt 1.5pt,linewidth=0.4pt,linecolor=lightgray]{c-c}(-4.3,0)(7.28,0)}
\multips(-4,0)(1.0,0){12}{\psline[linestyle=dashed,linecap=1,dash=1.5pt 1.5pt,linewidth=0.4pt,linecolor=lightgray]{c-c}(0,-3.02)(0,6.3)}
\psaxes[labelFontSize=\scriptstyle,xAxis=true,yAxis=true,Dx=1.,Dy=1.,ticksize=-2pt 0,subticks=2]{->}(0,0)(-4.3,-3.02)(7.28,6.3)
\psplot[linewidth=2.pt]{-4.3}{7.28}{(1+0.0*x)/1.}
%\psplot[linewidth=2.pt]{-4.3}{7.28}{(-2.--1.5*x)/1.}
\begin{scriptsize}
\rput[bl](5.08,5.98){$f$}
\end{scriptsize}
\end{pspicture*}













































\EXERCICE
\enonce{ %début énoncé 

Soit ABC un triangle rectangle en B. On sait que AB = 4 et BC = 3
} % fin énoncé 

\begin{description}
\item[1.] \enonce{ %début énoncé 
Calculer la longueur de AC \points{2.0} %en pourcentage
} % fin énoncé 

\correction{ %début énoncé 
 Le triangle ABC est rectangle en B,, donc avec le théorème de Pythagore on a : 
\begin{eqnarray*}
AB^2 + BC^2 &=& AC^2 \\
4^2 + 3^2 &=& AC^2 \\
25 &=&AC^2 \\
& & \mbox{ Donc : } AC = \sqrt{25} 
\end{eqnarray*}
} % fin correction 

\item[2.] \enonce{ %début énoncé 
Calculer l'angle $\widehat{ABC}$ au degré près. } % fin énoncé 
\points{2.0} %en pourcentage
\end{description}

\psset{xunit=1.0cm,yunit=1.0cm,algebraic=true,dimen=middle,dotstyle=o,dotsize=5pt 0,linewidth=2.pt,arrowsize=3pt 2,arrowinset=0.25}
\begin{pspicture*}(-4.3,-3.02)(7.28,6.3)
\multips(0,-3)(0,1.0){10}{\psline[linestyle=dashed,linecap=1,dash=1.5pt 1.5pt,linewidth=0.4pt,linecolor=lightgray]{c-c}(-4.3,0)(7.28,0)}
\multips(-4,0)(1.0,0){12}{\psline[linestyle=dashed,linecap=1,dash=1.5pt 1.5pt,linewidth=0.4pt,linecolor=lightgray]{c-c}(0,-3.02)(0,6.3)}
\psaxes[labelFontSize=\scriptstyle,xAxis=true,yAxis=true,Dx=1.,Dy=1.,ticksize=-2pt 0,subticks=2]{->}(0,0)(-4.3,-3.02)(7.28,6.3)
\psplot[linewidth=2.pt]{-4.3}{7.28}{(1+0.2*x)/1.}
%\psplot[linewidth=2.pt]{-4.3}{7.28}{(-2.--1.5*x)/1.}
\begin{scriptsize}
\rput[bl](5.08,5.98){$f$}
\end{scriptsize}
\end{pspicture*}













































\EXERCICE
\enonce{ %début énoncé 

Soit ABC un triangle rectangle en B. On sait que AB = 4 et BC = 3
} % fin énoncé 

\begin{description}
\item[1.] \enonce{ %début énoncé 
Calculer la longueur de AC \points{2.0} %en pourcentage
} % fin énoncé 

\correction{ %début énoncé 
 Le triangle ABC est rectangle en B,, donc avec le théorème de Pythagore on a : 
\begin{eqnarray*}
AB^2 + BC^2 &=& AC^2 \\
4^2 + 3^2 &=& AC^2 \\
25 &=&AC^2 \\
& & \mbox{ Donc : } AC = \sqrt{25} 
\end{eqnarray*}
} % fin correction 

\item[2.] \enonce{ %début énoncé 
Calculer l'angle $\widehat{ABC}$ au degré près. } % fin énoncé 
\points{2.0} %en pourcentage
\end{description}

\psset{xunit=1.0cm,yunit=1.0cm,algebraic=true,dimen=middle,dotstyle=o,dotsize=5pt 0,linewidth=2.pt,arrowsize=3pt 2,arrowinset=0.25}
\begin{pspicture*}(-4.3,-3.02)(7.28,6.3)
\multips(0,-3)(0,1.0){10}{\psline[linestyle=dashed,linecap=1,dash=1.5pt 1.5pt,linewidth=0.4pt,linecolor=lightgray]{c-c}(-4.3,0)(7.28,0)}
\multips(-4,0)(1.0,0){12}{\psline[linestyle=dashed,linecap=1,dash=1.5pt 1.5pt,linewidth=0.4pt,linecolor=lightgray]{c-c}(0,-3.02)(0,6.3)}
\psaxes[labelFontSize=\scriptstyle,xAxis=true,yAxis=true,Dx=1.,Dy=1.,ticksize=-2pt 0,subticks=2]{->}(0,0)(-4.3,-3.02)(7.28,6.3)
\psplot[linewidth=2.pt]{-4.3}{7.28}{(1+0.4*x)/1.}
%\psplot[linewidth=2.pt]{-4.3}{7.28}{(-2.--1.5*x)/1.}
\begin{scriptsize}
\rput[bl](5.08,5.98){$f$}
\end{scriptsize}
\end{pspicture*}













































\EXERCICE
\enonce{ %début énoncé 

Soit ABC un triangle rectangle en B. On sait que AB = 4 et BC = 4
} % fin énoncé 

\begin{description}
\item[1.] \enonce{ %début énoncé 
Calculer la longueur de AC \points{2.0} %en pourcentage
} % fin énoncé 

\correction{ %début énoncé 
 Le triangle ABC est rectangle en B,, donc avec le théorème de Pythagore on a : 
\begin{eqnarray*}
AB^2 + BC^2 &=& AC^2 \\
4^2 + 4^2 &=& AC^2 \\
32 &=&AC^2 \\
& & \mbox{ Donc : } AC = \sqrt{32} 
\end{eqnarray*}
} % fin correction 

\item[2.] \enonce{ %début énoncé 
Calculer l'angle $\widehat{ABC}$ au degré près. } % fin énoncé 
\points{2.0} %en pourcentage
\end{description}

\psset{xunit=1.0cm,yunit=1.0cm,algebraic=true,dimen=middle,dotstyle=o,dotsize=5pt 0,linewidth=2.pt,arrowsize=3pt 2,arrowinset=0.25}
\begin{pspicture*}(-4.3,-3.02)(7.28,6.3)
\multips(0,-3)(0,1.0){10}{\psline[linestyle=dashed,linecap=1,dash=1.5pt 1.5pt,linewidth=0.4pt,linecolor=lightgray]{c-c}(-4.3,0)(7.28,0)}
\multips(-4,0)(1.0,0){12}{\psline[linestyle=dashed,linecap=1,dash=1.5pt 1.5pt,linewidth=0.4pt,linecolor=lightgray]{c-c}(0,-3.02)(0,6.3)}
\psaxes[labelFontSize=\scriptstyle,xAxis=true,yAxis=true,Dx=1.,Dy=1.,ticksize=-2pt 0,subticks=2]{->}(0,0)(-4.3,-3.02)(7.28,6.3)
\psplot[linewidth=2.pt]{-4.3}{7.28}{(-1+-0.4*x)/1.}
%\psplot[linewidth=2.pt]{-4.3}{7.28}{(-2.--1.5*x)/1.}
\begin{scriptsize}
\rput[bl](5.08,5.98){$f$}
\end{scriptsize}
\end{pspicture*}













































\EXERCICE
\enonce{ %début énoncé 

Soit ABC un triangle rectangle en B. On sait que AB = 4 et BC = 4
} % fin énoncé 

\begin{description}
\item[1.] \enonce{ %début énoncé 
Calculer la longueur de AC \points{2.0} %en pourcentage
} % fin énoncé 

\correction{ %début énoncé 
 Le triangle ABC est rectangle en B,, donc avec le théorème de Pythagore on a : 
\begin{eqnarray*}
AB^2 + BC^2 &=& AC^2 \\
4^2 + 4^2 &=& AC^2 \\
32 &=&AC^2 \\
& & \mbox{ Donc : } AC = \sqrt{32} 
\end{eqnarray*}
} % fin correction 

\item[2.] \enonce{ %début énoncé 
Calculer l'angle $\widehat{ABC}$ au degré près. } % fin énoncé 
\points{2.0} %en pourcentage
\end{description}

\psset{xunit=1.0cm,yunit=1.0cm,algebraic=true,dimen=middle,dotstyle=o,dotsize=5pt 0,linewidth=2.pt,arrowsize=3pt 2,arrowinset=0.25}
\begin{pspicture*}(-4.3,-3.02)(7.28,6.3)
\multips(0,-3)(0,1.0){10}{\psline[linestyle=dashed,linecap=1,dash=1.5pt 1.5pt,linewidth=0.4pt,linecolor=lightgray]{c-c}(-4.3,0)(7.28,0)}
\multips(-4,0)(1.0,0){12}{\psline[linestyle=dashed,linecap=1,dash=1.5pt 1.5pt,linewidth=0.4pt,linecolor=lightgray]{c-c}(0,-3.02)(0,6.3)}
\psaxes[labelFontSize=\scriptstyle,xAxis=true,yAxis=true,Dx=1.,Dy=1.,ticksize=-2pt 0,subticks=2]{->}(0,0)(-4.3,-3.02)(7.28,6.3)
\psplot[linewidth=2.pt]{-4.3}{7.28}{(-1+-0.2*x)/1.}
%\psplot[linewidth=2.pt]{-4.3}{7.28}{(-2.--1.5*x)/1.}
\begin{scriptsize}
\rput[bl](5.08,5.98){$f$}
\end{scriptsize}
\end{pspicture*}













































\EXERCICE
\enonce{ %début énoncé 

Soit ABC un triangle rectangle en B. On sait que AB = 4 et BC = 4
} % fin énoncé 

\begin{description}
\item[1.] \enonce{ %début énoncé 
Calculer la longueur de AC \points{2.0} %en pourcentage
} % fin énoncé 

\correction{ %début énoncé 
 Le triangle ABC est rectangle en B,, donc avec le théorème de Pythagore on a : 
\begin{eqnarray*}
AB^2 + BC^2 &=& AC^2 \\
4^2 + 4^2 &=& AC^2 \\
32 &=&AC^2 \\
& & \mbox{ Donc : } AC = \sqrt{32} 
\end{eqnarray*}
} % fin correction 

\item[2.] \enonce{ %début énoncé 
Calculer l'angle $\widehat{ABC}$ au degré près. } % fin énoncé 
\points{2.0} %en pourcentage
\end{description}

\psset{xunit=1.0cm,yunit=1.0cm,algebraic=true,dimen=middle,dotstyle=o,dotsize=5pt 0,linewidth=2.pt,arrowsize=3pt 2,arrowinset=0.25}
\begin{pspicture*}(-4.3,-3.02)(7.28,6.3)
\multips(0,-3)(0,1.0){10}{\psline[linestyle=dashed,linecap=1,dash=1.5pt 1.5pt,linewidth=0.4pt,linecolor=lightgray]{c-c}(-4.3,0)(7.28,0)}
\multips(-4,0)(1.0,0){12}{\psline[linestyle=dashed,linecap=1,dash=1.5pt 1.5pt,linewidth=0.4pt,linecolor=lightgray]{c-c}(0,-3.02)(0,6.3)}
\psaxes[labelFontSize=\scriptstyle,xAxis=true,yAxis=true,Dx=1.,Dy=1.,ticksize=-2pt 0,subticks=2]{->}(0,0)(-4.3,-3.02)(7.28,6.3)
\psplot[linewidth=2.pt]{-4.3}{7.28}{(-1+0.0*x)/1.}
%\psplot[linewidth=2.pt]{-4.3}{7.28}{(-2.--1.5*x)/1.}
\begin{scriptsize}
\rput[bl](5.08,5.98){$f$}
\end{scriptsize}
\end{pspicture*}













































\EXERCICE
\enonce{ %début énoncé 

Soit ABC un triangle rectangle en B. On sait que AB = 4 et BC = 4
} % fin énoncé 

\begin{description}
\item[1.] \enonce{ %début énoncé 
Calculer la longueur de AC \points{2.0} %en pourcentage
} % fin énoncé 

\correction{ %début énoncé 
 Le triangle ABC est rectangle en B,, donc avec le théorème de Pythagore on a : 
\begin{eqnarray*}
AB^2 + BC^2 &=& AC^2 \\
4^2 + 4^2 &=& AC^2 \\
32 &=&AC^2 \\
& & \mbox{ Donc : } AC = \sqrt{32} 
\end{eqnarray*}
} % fin correction 

\item[2.] \enonce{ %début énoncé 
Calculer l'angle $\widehat{ABC}$ au degré près. } % fin énoncé 
\points{2.0} %en pourcentage
\end{description}

\psset{xunit=1.0cm,yunit=1.0cm,algebraic=true,dimen=middle,dotstyle=o,dotsize=5pt 0,linewidth=2.pt,arrowsize=3pt 2,arrowinset=0.25}
\begin{pspicture*}(-4.3,-3.02)(7.28,6.3)
\multips(0,-3)(0,1.0){10}{\psline[linestyle=dashed,linecap=1,dash=1.5pt 1.5pt,linewidth=0.4pt,linecolor=lightgray]{c-c}(-4.3,0)(7.28,0)}
\multips(-4,0)(1.0,0){12}{\psline[linestyle=dashed,linecap=1,dash=1.5pt 1.5pt,linewidth=0.4pt,linecolor=lightgray]{c-c}(0,-3.02)(0,6.3)}
\psaxes[labelFontSize=\scriptstyle,xAxis=true,yAxis=true,Dx=1.,Dy=1.,ticksize=-2pt 0,subticks=2]{->}(0,0)(-4.3,-3.02)(7.28,6.3)
\psplot[linewidth=2.pt]{-4.3}{7.28}{(-1+0.2*x)/1.}
%\psplot[linewidth=2.pt]{-4.3}{7.28}{(-2.--1.5*x)/1.}
\begin{scriptsize}
\rput[bl](5.08,5.98){$f$}
\end{scriptsize}
\end{pspicture*}













































\EXERCICE
\enonce{ %début énoncé 

Soit ABC un triangle rectangle en B. On sait que AB = 4 et BC = 4
} % fin énoncé 

\begin{description}
\item[1.] \enonce{ %début énoncé 
Calculer la longueur de AC \points{2.0} %en pourcentage
} % fin énoncé 

\correction{ %début énoncé 
 Le triangle ABC est rectangle en B,, donc avec le théorème de Pythagore on a : 
\begin{eqnarray*}
AB^2 + BC^2 &=& AC^2 \\
4^2 + 4^2 &=& AC^2 \\
32 &=&AC^2 \\
& & \mbox{ Donc : } AC = \sqrt{32} 
\end{eqnarray*}
} % fin correction 

\item[2.] \enonce{ %début énoncé 
Calculer l'angle $\widehat{ABC}$ au degré près. } % fin énoncé 
\points{2.0} %en pourcentage
\end{description}

\psset{xunit=1.0cm,yunit=1.0cm,algebraic=true,dimen=middle,dotstyle=o,dotsize=5pt 0,linewidth=2.pt,arrowsize=3pt 2,arrowinset=0.25}
\begin{pspicture*}(-4.3,-3.02)(7.28,6.3)
\multips(0,-3)(0,1.0){10}{\psline[linestyle=dashed,linecap=1,dash=1.5pt 1.5pt,linewidth=0.4pt,linecolor=lightgray]{c-c}(-4.3,0)(7.28,0)}
\multips(-4,0)(1.0,0){12}{\psline[linestyle=dashed,linecap=1,dash=1.5pt 1.5pt,linewidth=0.4pt,linecolor=lightgray]{c-c}(0,-3.02)(0,6.3)}
\psaxes[labelFontSize=\scriptstyle,xAxis=true,yAxis=true,Dx=1.,Dy=1.,ticksize=-2pt 0,subticks=2]{->}(0,0)(-4.3,-3.02)(7.28,6.3)
\psplot[linewidth=2.pt]{-4.3}{7.28}{(-1+0.4*x)/1.}
%\psplot[linewidth=2.pt]{-4.3}{7.28}{(-2.--1.5*x)/1.}
\begin{scriptsize}
\rput[bl](5.08,5.98){$f$}
\end{scriptsize}
\end{pspicture*}













































\EXERCICE
\enonce{ %début énoncé 

Soit ABC un triangle rectangle en B. On sait que AB = 4 et BC = 4
} % fin énoncé 

\begin{description}
\item[1.] \enonce{ %début énoncé 
Calculer la longueur de AC \points{2.0} %en pourcentage
} % fin énoncé 

\correction{ %début énoncé 
 Le triangle ABC est rectangle en B,, donc avec le théorème de Pythagore on a : 
\begin{eqnarray*}
AB^2 + BC^2 &=& AC^2 \\
4^2 + 4^2 &=& AC^2 \\
32 &=&AC^2 \\
& & \mbox{ Donc : } AC = \sqrt{32} 
\end{eqnarray*}
} % fin correction 

\item[2.] \enonce{ %début énoncé 
Calculer l'angle $\widehat{ABC}$ au degré près. } % fin énoncé 
\points{2.0} %en pourcentage
\end{description}

\psset{xunit=1.0cm,yunit=1.0cm,algebraic=true,dimen=middle,dotstyle=o,dotsize=5pt 0,linewidth=2.pt,arrowsize=3pt 2,arrowinset=0.25}
\begin{pspicture*}(-4.3,-3.02)(7.28,6.3)
\multips(0,-3)(0,1.0){10}{\psline[linestyle=dashed,linecap=1,dash=1.5pt 1.5pt,linewidth=0.4pt,linecolor=lightgray]{c-c}(-4.3,0)(7.28,0)}
\multips(-4,0)(1.0,0){12}{\psline[linestyle=dashed,linecap=1,dash=1.5pt 1.5pt,linewidth=0.4pt,linecolor=lightgray]{c-c}(0,-3.02)(0,6.3)}
\psaxes[labelFontSize=\scriptstyle,xAxis=true,yAxis=true,Dx=1.,Dy=1.,ticksize=-2pt 0,subticks=2]{->}(0,0)(-4.3,-3.02)(7.28,6.3)
\psplot[linewidth=2.pt]{-4.3}{7.28}{(0+-0.4*x)/1.}
%\psplot[linewidth=2.pt]{-4.3}{7.28}{(-2.--1.5*x)/1.}
\begin{scriptsize}
\rput[bl](5.08,5.98){$f$}
\end{scriptsize}
\end{pspicture*}













































\EXERCICE
\enonce{ %début énoncé 

Soit ABC un triangle rectangle en B. On sait que AB = 4 et BC = 4
} % fin énoncé 

\begin{description}
\item[1.] \enonce{ %début énoncé 
Calculer la longueur de AC \points{2.0} %en pourcentage
} % fin énoncé 

\correction{ %début énoncé 
 Le triangle ABC est rectangle en B,, donc avec le théorème de Pythagore on a : 
\begin{eqnarray*}
AB^2 + BC^2 &=& AC^2 \\
4^2 + 4^2 &=& AC^2 \\
32 &=&AC^2 \\
& & \mbox{ Donc : } AC = \sqrt{32} 
\end{eqnarray*}
} % fin correction 

\item[2.] \enonce{ %début énoncé 
Calculer l'angle $\widehat{ABC}$ au degré près. } % fin énoncé 
\points{2.0} %en pourcentage
\end{description}

\psset{xunit=1.0cm,yunit=1.0cm,algebraic=true,dimen=middle,dotstyle=o,dotsize=5pt 0,linewidth=2.pt,arrowsize=3pt 2,arrowinset=0.25}
\begin{pspicture*}(-4.3,-3.02)(7.28,6.3)
\multips(0,-3)(0,1.0){10}{\psline[linestyle=dashed,linecap=1,dash=1.5pt 1.5pt,linewidth=0.4pt,linecolor=lightgray]{c-c}(-4.3,0)(7.28,0)}
\multips(-4,0)(1.0,0){12}{\psline[linestyle=dashed,linecap=1,dash=1.5pt 1.5pt,linewidth=0.4pt,linecolor=lightgray]{c-c}(0,-3.02)(0,6.3)}
\psaxes[labelFontSize=\scriptstyle,xAxis=true,yAxis=true,Dx=1.,Dy=1.,ticksize=-2pt 0,subticks=2]{->}(0,0)(-4.3,-3.02)(7.28,6.3)
\psplot[linewidth=2.pt]{-4.3}{7.28}{(0+-0.2*x)/1.}
%\psplot[linewidth=2.pt]{-4.3}{7.28}{(-2.--1.5*x)/1.}
\begin{scriptsize}
\rput[bl](5.08,5.98){$f$}
\end{scriptsize}
\end{pspicture*}













































\EXERCICE
\enonce{ %début énoncé 

Soit ABC un triangle rectangle en B. On sait que AB = 4 et BC = 4
} % fin énoncé 

\begin{description}
\item[1.] \enonce{ %début énoncé 
Calculer la longueur de AC \points{2.0} %en pourcentage
} % fin énoncé 

\correction{ %début énoncé 
 Le triangle ABC est rectangle en B,, donc avec le théorème de Pythagore on a : 
\begin{eqnarray*}
AB^2 + BC^2 &=& AC^2 \\
4^2 + 4^2 &=& AC^2 \\
32 &=&AC^2 \\
& & \mbox{ Donc : } AC = \sqrt{32} 
\end{eqnarray*}
} % fin correction 

\item[2.] \enonce{ %début énoncé 
Calculer l'angle $\widehat{ABC}$ au degré près. } % fin énoncé 
\points{2.0} %en pourcentage
\end{description}

\psset{xunit=1.0cm,yunit=1.0cm,algebraic=true,dimen=middle,dotstyle=o,dotsize=5pt 0,linewidth=2.pt,arrowsize=3pt 2,arrowinset=0.25}
\begin{pspicture*}(-4.3,-3.02)(7.28,6.3)
\multips(0,-3)(0,1.0){10}{\psline[linestyle=dashed,linecap=1,dash=1.5pt 1.5pt,linewidth=0.4pt,linecolor=lightgray]{c-c}(-4.3,0)(7.28,0)}
\multips(-4,0)(1.0,0){12}{\psline[linestyle=dashed,linecap=1,dash=1.5pt 1.5pt,linewidth=0.4pt,linecolor=lightgray]{c-c}(0,-3.02)(0,6.3)}
\psaxes[labelFontSize=\scriptstyle,xAxis=true,yAxis=true,Dx=1.,Dy=1.,ticksize=-2pt 0,subticks=2]{->}(0,0)(-4.3,-3.02)(7.28,6.3)
\psplot[linewidth=2.pt]{-4.3}{7.28}{(0+0.0*x)/1.}
%\psplot[linewidth=2.pt]{-4.3}{7.28}{(-2.--1.5*x)/1.}
\begin{scriptsize}
\rput[bl](5.08,5.98){$f$}
\end{scriptsize}
\end{pspicture*}













































\EXERCICE
\enonce{ %début énoncé 

Soit ABC un triangle rectangle en B. On sait que AB = 4 et BC = 4
} % fin énoncé 

\begin{description}
\item[1.] \enonce{ %début énoncé 
Calculer la longueur de AC \points{2.0} %en pourcentage
} % fin énoncé 

\correction{ %début énoncé 
 Le triangle ABC est rectangle en B,, donc avec le théorème de Pythagore on a : 
\begin{eqnarray*}
AB^2 + BC^2 &=& AC^2 \\
4^2 + 4^2 &=& AC^2 \\
32 &=&AC^2 \\
& & \mbox{ Donc : } AC = \sqrt{32} 
\end{eqnarray*}
} % fin correction 

\item[2.] \enonce{ %début énoncé 
Calculer l'angle $\widehat{ABC}$ au degré près. } % fin énoncé 
\points{2.0} %en pourcentage
\end{description}

\psset{xunit=1.0cm,yunit=1.0cm,algebraic=true,dimen=middle,dotstyle=o,dotsize=5pt 0,linewidth=2.pt,arrowsize=3pt 2,arrowinset=0.25}
\begin{pspicture*}(-4.3,-3.02)(7.28,6.3)
\multips(0,-3)(0,1.0){10}{\psline[linestyle=dashed,linecap=1,dash=1.5pt 1.5pt,linewidth=0.4pt,linecolor=lightgray]{c-c}(-4.3,0)(7.28,0)}
\multips(-4,0)(1.0,0){12}{\psline[linestyle=dashed,linecap=1,dash=1.5pt 1.5pt,linewidth=0.4pt,linecolor=lightgray]{c-c}(0,-3.02)(0,6.3)}
\psaxes[labelFontSize=\scriptstyle,xAxis=true,yAxis=true,Dx=1.,Dy=1.,ticksize=-2pt 0,subticks=2]{->}(0,0)(-4.3,-3.02)(7.28,6.3)
\psplot[linewidth=2.pt]{-4.3}{7.28}{(0+0.2*x)/1.}
%\psplot[linewidth=2.pt]{-4.3}{7.28}{(-2.--1.5*x)/1.}
\begin{scriptsize}
\rput[bl](5.08,5.98){$f$}
\end{scriptsize}
\end{pspicture*}













































\EXERCICE
\enonce{ %début énoncé 

Soit ABC un triangle rectangle en B. On sait que AB = 4 et BC = 4
} % fin énoncé 

\begin{description}
\item[1.] \enonce{ %début énoncé 
Calculer la longueur de AC \points{2.0} %en pourcentage
} % fin énoncé 

\correction{ %début énoncé 
 Le triangle ABC est rectangle en B,, donc avec le théorème de Pythagore on a : 
\begin{eqnarray*}
AB^2 + BC^2 &=& AC^2 \\
4^2 + 4^2 &=& AC^2 \\
32 &=&AC^2 \\
& & \mbox{ Donc : } AC = \sqrt{32} 
\end{eqnarray*}
} % fin correction 

\item[2.] \enonce{ %début énoncé 
Calculer l'angle $\widehat{ABC}$ au degré près. } % fin énoncé 
\points{2.0} %en pourcentage
\end{description}

\psset{xunit=1.0cm,yunit=1.0cm,algebraic=true,dimen=middle,dotstyle=o,dotsize=5pt 0,linewidth=2.pt,arrowsize=3pt 2,arrowinset=0.25}
\begin{pspicture*}(-4.3,-3.02)(7.28,6.3)
\multips(0,-3)(0,1.0){10}{\psline[linestyle=dashed,linecap=1,dash=1.5pt 1.5pt,linewidth=0.4pt,linecolor=lightgray]{c-c}(-4.3,0)(7.28,0)}
\multips(-4,0)(1.0,0){12}{\psline[linestyle=dashed,linecap=1,dash=1.5pt 1.5pt,linewidth=0.4pt,linecolor=lightgray]{c-c}(0,-3.02)(0,6.3)}
\psaxes[labelFontSize=\scriptstyle,xAxis=true,yAxis=true,Dx=1.,Dy=1.,ticksize=-2pt 0,subticks=2]{->}(0,0)(-4.3,-3.02)(7.28,6.3)
\psplot[linewidth=2.pt]{-4.3}{7.28}{(0+0.4*x)/1.}
%\psplot[linewidth=2.pt]{-4.3}{7.28}{(-2.--1.5*x)/1.}
\begin{scriptsize}
\rput[bl](5.08,5.98){$f$}
\end{scriptsize}
\end{pspicture*}













































\EXERCICE
\enonce{ %début énoncé 

Soit ABC un triangle rectangle en B. On sait que AB = 4 et BC = 4
} % fin énoncé 

\begin{description}
\item[1.] \enonce{ %début énoncé 
Calculer la longueur de AC \points{2.0} %en pourcentage
} % fin énoncé 

\correction{ %début énoncé 
 Le triangle ABC est rectangle en B,, donc avec le théorème de Pythagore on a : 
\begin{eqnarray*}
AB^2 + BC^2 &=& AC^2 \\
4^2 + 4^2 &=& AC^2 \\
32 &=&AC^2 \\
& & \mbox{ Donc : } AC = \sqrt{32} 
\end{eqnarray*}
} % fin correction 

\item[2.] \enonce{ %début énoncé 
Calculer l'angle $\widehat{ABC}$ au degré près. } % fin énoncé 
\points{2.0} %en pourcentage
\end{description}

\psset{xunit=1.0cm,yunit=1.0cm,algebraic=true,dimen=middle,dotstyle=o,dotsize=5pt 0,linewidth=2.pt,arrowsize=3pt 2,arrowinset=0.25}
\begin{pspicture*}(-4.3,-3.02)(7.28,6.3)
\multips(0,-3)(0,1.0){10}{\psline[linestyle=dashed,linecap=1,dash=1.5pt 1.5pt,linewidth=0.4pt,linecolor=lightgray]{c-c}(-4.3,0)(7.28,0)}
\multips(-4,0)(1.0,0){12}{\psline[linestyle=dashed,linecap=1,dash=1.5pt 1.5pt,linewidth=0.4pt,linecolor=lightgray]{c-c}(0,-3.02)(0,6.3)}
\psaxes[labelFontSize=\scriptstyle,xAxis=true,yAxis=true,Dx=1.,Dy=1.,ticksize=-2pt 0,subticks=2]{->}(0,0)(-4.3,-3.02)(7.28,6.3)
\psplot[linewidth=2.pt]{-4.3}{7.28}{(1+-0.4*x)/1.}
%\psplot[linewidth=2.pt]{-4.3}{7.28}{(-2.--1.5*x)/1.}
\begin{scriptsize}
\rput[bl](5.08,5.98){$f$}
\end{scriptsize}
\end{pspicture*}













































\EXERCICE
\enonce{ %début énoncé 

Soit ABC un triangle rectangle en B. On sait que AB = 4 et BC = 4
} % fin énoncé 

\begin{description}
\item[1.] \enonce{ %début énoncé 
Calculer la longueur de AC \points{2.0} %en pourcentage
} % fin énoncé 

\correction{ %début énoncé 
 Le triangle ABC est rectangle en B,, donc avec le théorème de Pythagore on a : 
\begin{eqnarray*}
AB^2 + BC^2 &=& AC^2 \\
4^2 + 4^2 &=& AC^2 \\
32 &=&AC^2 \\
& & \mbox{ Donc : } AC = \sqrt{32} 
\end{eqnarray*}
} % fin correction 

\item[2.] \enonce{ %début énoncé 
Calculer l'angle $\widehat{ABC}$ au degré près. } % fin énoncé 
\points{2.0} %en pourcentage
\end{description}

\psset{xunit=1.0cm,yunit=1.0cm,algebraic=true,dimen=middle,dotstyle=o,dotsize=5pt 0,linewidth=2.pt,arrowsize=3pt 2,arrowinset=0.25}
\begin{pspicture*}(-4.3,-3.02)(7.28,6.3)
\multips(0,-3)(0,1.0){10}{\psline[linestyle=dashed,linecap=1,dash=1.5pt 1.5pt,linewidth=0.4pt,linecolor=lightgray]{c-c}(-4.3,0)(7.28,0)}
\multips(-4,0)(1.0,0){12}{\psline[linestyle=dashed,linecap=1,dash=1.5pt 1.5pt,linewidth=0.4pt,linecolor=lightgray]{c-c}(0,-3.02)(0,6.3)}
\psaxes[labelFontSize=\scriptstyle,xAxis=true,yAxis=true,Dx=1.,Dy=1.,ticksize=-2pt 0,subticks=2]{->}(0,0)(-4.3,-3.02)(7.28,6.3)
\psplot[linewidth=2.pt]{-4.3}{7.28}{(1+-0.2*x)/1.}
%\psplot[linewidth=2.pt]{-4.3}{7.28}{(-2.--1.5*x)/1.}
\begin{scriptsize}
\rput[bl](5.08,5.98){$f$}
\end{scriptsize}
\end{pspicture*}













































\EXERCICE
\enonce{ %début énoncé 

Soit ABC un triangle rectangle en B. On sait que AB = 4 et BC = 4
} % fin énoncé 

\begin{description}
\item[1.] \enonce{ %début énoncé 
Calculer la longueur de AC \points{2.0} %en pourcentage
} % fin énoncé 

\correction{ %début énoncé 
 Le triangle ABC est rectangle en B,, donc avec le théorème de Pythagore on a : 
\begin{eqnarray*}
AB^2 + BC^2 &=& AC^2 \\
4^2 + 4^2 &=& AC^2 \\
32 &=&AC^2 \\
& & \mbox{ Donc : } AC = \sqrt{32} 
\end{eqnarray*}
} % fin correction 

\item[2.] \enonce{ %début énoncé 
Calculer l'angle $\widehat{ABC}$ au degré près. } % fin énoncé 
\points{2.0} %en pourcentage
\end{description}

\psset{xunit=1.0cm,yunit=1.0cm,algebraic=true,dimen=middle,dotstyle=o,dotsize=5pt 0,linewidth=2.pt,arrowsize=3pt 2,arrowinset=0.25}
\begin{pspicture*}(-4.3,-3.02)(7.28,6.3)
\multips(0,-3)(0,1.0){10}{\psline[linestyle=dashed,linecap=1,dash=1.5pt 1.5pt,linewidth=0.4pt,linecolor=lightgray]{c-c}(-4.3,0)(7.28,0)}
\multips(-4,0)(1.0,0){12}{\psline[linestyle=dashed,linecap=1,dash=1.5pt 1.5pt,linewidth=0.4pt,linecolor=lightgray]{c-c}(0,-3.02)(0,6.3)}
\psaxes[labelFontSize=\scriptstyle,xAxis=true,yAxis=true,Dx=1.,Dy=1.,ticksize=-2pt 0,subticks=2]{->}(0,0)(-4.3,-3.02)(7.28,6.3)
\psplot[linewidth=2.pt]{-4.3}{7.28}{(1+0.0*x)/1.}
%\psplot[linewidth=2.pt]{-4.3}{7.28}{(-2.--1.5*x)/1.}
\begin{scriptsize}
\rput[bl](5.08,5.98){$f$}
\end{scriptsize}
\end{pspicture*}













































\EXERCICE
\enonce{ %début énoncé 

Soit ABC un triangle rectangle en B. On sait que AB = 4 et BC = 4
} % fin énoncé 

\begin{description}
\item[1.] \enonce{ %début énoncé 
Calculer la longueur de AC \points{2.0} %en pourcentage
} % fin énoncé 

\correction{ %début énoncé 
 Le triangle ABC est rectangle en B,, donc avec le théorème de Pythagore on a : 
\begin{eqnarray*}
AB^2 + BC^2 &=& AC^2 \\
4^2 + 4^2 &=& AC^2 \\
32 &=&AC^2 \\
& & \mbox{ Donc : } AC = \sqrt{32} 
\end{eqnarray*}
} % fin correction 

\item[2.] \enonce{ %début énoncé 
Calculer l'angle $\widehat{ABC}$ au degré près. } % fin énoncé 
\points{2.0} %en pourcentage
\end{description}

\psset{xunit=1.0cm,yunit=1.0cm,algebraic=true,dimen=middle,dotstyle=o,dotsize=5pt 0,linewidth=2.pt,arrowsize=3pt 2,arrowinset=0.25}
\begin{pspicture*}(-4.3,-3.02)(7.28,6.3)
\multips(0,-3)(0,1.0){10}{\psline[linestyle=dashed,linecap=1,dash=1.5pt 1.5pt,linewidth=0.4pt,linecolor=lightgray]{c-c}(-4.3,0)(7.28,0)}
\multips(-4,0)(1.0,0){12}{\psline[linestyle=dashed,linecap=1,dash=1.5pt 1.5pt,linewidth=0.4pt,linecolor=lightgray]{c-c}(0,-3.02)(0,6.3)}
\psaxes[labelFontSize=\scriptstyle,xAxis=true,yAxis=true,Dx=1.,Dy=1.,ticksize=-2pt 0,subticks=2]{->}(0,0)(-4.3,-3.02)(7.28,6.3)
\psplot[linewidth=2.pt]{-4.3}{7.28}{(1+0.2*x)/1.}
%\psplot[linewidth=2.pt]{-4.3}{7.28}{(-2.--1.5*x)/1.}
\begin{scriptsize}
\rput[bl](5.08,5.98){$f$}
\end{scriptsize}
\end{pspicture*}













































\EXERCICE
\enonce{ %début énoncé 

Soit ABC un triangle rectangle en B. On sait que AB = 4 et BC = 4
} % fin énoncé 

\begin{description}
\item[1.] \enonce{ %début énoncé 
Calculer la longueur de AC \points{2.0} %en pourcentage
} % fin énoncé 

\correction{ %début énoncé 
 Le triangle ABC est rectangle en B,, donc avec le théorème de Pythagore on a : 
\begin{eqnarray*}
AB^2 + BC^2 &=& AC^2 \\
4^2 + 4^2 &=& AC^2 \\
32 &=&AC^2 \\
& & \mbox{ Donc : } AC = \sqrt{32} 
\end{eqnarray*}
} % fin correction 

\item[2.] \enonce{ %début énoncé 
Calculer l'angle $\widehat{ABC}$ au degré près. } % fin énoncé 
\points{2.0} %en pourcentage
\end{description}

\psset{xunit=1.0cm,yunit=1.0cm,algebraic=true,dimen=middle,dotstyle=o,dotsize=5pt 0,linewidth=2.pt,arrowsize=3pt 2,arrowinset=0.25}
\begin{pspicture*}(-4.3,-3.02)(7.28,6.3)
\multips(0,-3)(0,1.0){10}{\psline[linestyle=dashed,linecap=1,dash=1.5pt 1.5pt,linewidth=0.4pt,linecolor=lightgray]{c-c}(-4.3,0)(7.28,0)}
\multips(-4,0)(1.0,0){12}{\psline[linestyle=dashed,linecap=1,dash=1.5pt 1.5pt,linewidth=0.4pt,linecolor=lightgray]{c-c}(0,-3.02)(0,6.3)}
\psaxes[labelFontSize=\scriptstyle,xAxis=true,yAxis=true,Dx=1.,Dy=1.,ticksize=-2pt 0,subticks=2]{->}(0,0)(-4.3,-3.02)(7.28,6.3)
\psplot[linewidth=2.pt]{-4.3}{7.28}{(1+0.4*x)/1.}
%\psplot[linewidth=2.pt]{-4.3}{7.28}{(-2.--1.5*x)/1.}
\begin{scriptsize}
\rput[bl](5.08,5.98){$f$}
\end{scriptsize}
\end{pspicture*}













































\EXERCICE
\enonce{ %début énoncé 

Soit ABC un triangle rectangle en B. On sait que AB = 5 et BC = 3
} % fin énoncé 

\begin{description}
\item[1.] \enonce{ %début énoncé 
Calculer la longueur de AC \points{2.0} %en pourcentage
} % fin énoncé 

\correction{ %début énoncé 
 Le triangle ABC est rectangle en B,, donc avec le théorème de Pythagore on a : 
\begin{eqnarray*}
AB^2 + BC^2 &=& AC^2 \\
5^2 + 3^2 &=& AC^2 \\
34 &=&AC^2 \\
& & \mbox{ Donc : } AC = \sqrt{34} 
\end{eqnarray*}
} % fin correction 

\item[2.] \enonce{ %début énoncé 
Calculer l'angle $\widehat{ABC}$ au degré près. } % fin énoncé 
\points{2.0} %en pourcentage
\end{description}

\psset{xunit=1.0cm,yunit=1.0cm,algebraic=true,dimen=middle,dotstyle=o,dotsize=5pt 0,linewidth=2.pt,arrowsize=3pt 2,arrowinset=0.25}
\begin{pspicture*}(-4.3,-3.02)(7.28,6.3)
\multips(0,-3)(0,1.0){10}{\psline[linestyle=dashed,linecap=1,dash=1.5pt 1.5pt,linewidth=0.4pt,linecolor=lightgray]{c-c}(-4.3,0)(7.28,0)}
\multips(-4,0)(1.0,0){12}{\psline[linestyle=dashed,linecap=1,dash=1.5pt 1.5pt,linewidth=0.4pt,linecolor=lightgray]{c-c}(0,-3.02)(0,6.3)}
\psaxes[labelFontSize=\scriptstyle,xAxis=true,yAxis=true,Dx=1.,Dy=1.,ticksize=-2pt 0,subticks=2]{->}(0,0)(-4.3,-3.02)(7.28,6.3)
\psplot[linewidth=2.pt]{-4.3}{7.28}{(-1+-0.4*x)/1.}
%\psplot[linewidth=2.pt]{-4.3}{7.28}{(-2.--1.5*x)/1.}
\begin{scriptsize}
\rput[bl](5.08,5.98){$f$}
\end{scriptsize}
\end{pspicture*}













































\EXERCICE
\enonce{ %début énoncé 

Soit ABC un triangle rectangle en B. On sait que AB = 5 et BC = 3
} % fin énoncé 

\begin{description}
\item[1.] \enonce{ %début énoncé 
Calculer la longueur de AC \points{2.0} %en pourcentage
} % fin énoncé 

\correction{ %début énoncé 
 Le triangle ABC est rectangle en B,, donc avec le théorème de Pythagore on a : 
\begin{eqnarray*}
AB^2 + BC^2 &=& AC^2 \\
5^2 + 3^2 &=& AC^2 \\
34 &=&AC^2 \\
& & \mbox{ Donc : } AC = \sqrt{34} 
\end{eqnarray*}
} % fin correction 

\item[2.] \enonce{ %début énoncé 
Calculer l'angle $\widehat{ABC}$ au degré près. } % fin énoncé 
\points{2.0} %en pourcentage
\end{description}

\psset{xunit=1.0cm,yunit=1.0cm,algebraic=true,dimen=middle,dotstyle=o,dotsize=5pt 0,linewidth=2.pt,arrowsize=3pt 2,arrowinset=0.25}
\begin{pspicture*}(-4.3,-3.02)(7.28,6.3)
\multips(0,-3)(0,1.0){10}{\psline[linestyle=dashed,linecap=1,dash=1.5pt 1.5pt,linewidth=0.4pt,linecolor=lightgray]{c-c}(-4.3,0)(7.28,0)}
\multips(-4,0)(1.0,0){12}{\psline[linestyle=dashed,linecap=1,dash=1.5pt 1.5pt,linewidth=0.4pt,linecolor=lightgray]{c-c}(0,-3.02)(0,6.3)}
\psaxes[labelFontSize=\scriptstyle,xAxis=true,yAxis=true,Dx=1.,Dy=1.,ticksize=-2pt 0,subticks=2]{->}(0,0)(-4.3,-3.02)(7.28,6.3)
\psplot[linewidth=2.pt]{-4.3}{7.28}{(-1+-0.2*x)/1.}
%\psplot[linewidth=2.pt]{-4.3}{7.28}{(-2.--1.5*x)/1.}
\begin{scriptsize}
\rput[bl](5.08,5.98){$f$}
\end{scriptsize}
\end{pspicture*}













































\EXERCICE
\enonce{ %début énoncé 

Soit ABC un triangle rectangle en B. On sait que AB = 5 et BC = 3
} % fin énoncé 

\begin{description}
\item[1.] \enonce{ %début énoncé 
Calculer la longueur de AC \points{2.0} %en pourcentage
} % fin énoncé 

\correction{ %début énoncé 
 Le triangle ABC est rectangle en B,, donc avec le théorème de Pythagore on a : 
\begin{eqnarray*}
AB^2 + BC^2 &=& AC^2 \\
5^2 + 3^2 &=& AC^2 \\
34 &=&AC^2 \\
& & \mbox{ Donc : } AC = \sqrt{34} 
\end{eqnarray*}
} % fin correction 

\item[2.] \enonce{ %début énoncé 
Calculer l'angle $\widehat{ABC}$ au degré près. } % fin énoncé 
\points{2.0} %en pourcentage
\end{description}

\psset{xunit=1.0cm,yunit=1.0cm,algebraic=true,dimen=middle,dotstyle=o,dotsize=5pt 0,linewidth=2.pt,arrowsize=3pt 2,arrowinset=0.25}
\begin{pspicture*}(-4.3,-3.02)(7.28,6.3)
\multips(0,-3)(0,1.0){10}{\psline[linestyle=dashed,linecap=1,dash=1.5pt 1.5pt,linewidth=0.4pt,linecolor=lightgray]{c-c}(-4.3,0)(7.28,0)}
\multips(-4,0)(1.0,0){12}{\psline[linestyle=dashed,linecap=1,dash=1.5pt 1.5pt,linewidth=0.4pt,linecolor=lightgray]{c-c}(0,-3.02)(0,6.3)}
\psaxes[labelFontSize=\scriptstyle,xAxis=true,yAxis=true,Dx=1.,Dy=1.,ticksize=-2pt 0,subticks=2]{->}(0,0)(-4.3,-3.02)(7.28,6.3)
\psplot[linewidth=2.pt]{-4.3}{7.28}{(-1+0.0*x)/1.}
%\psplot[linewidth=2.pt]{-4.3}{7.28}{(-2.--1.5*x)/1.}
\begin{scriptsize}
\rput[bl](5.08,5.98){$f$}
\end{scriptsize}
\end{pspicture*}













































\EXERCICE
\enonce{ %début énoncé 

Soit ABC un triangle rectangle en B. On sait que AB = 5 et BC = 3
} % fin énoncé 

\begin{description}
\item[1.] \enonce{ %début énoncé 
Calculer la longueur de AC \points{2.0} %en pourcentage
} % fin énoncé 

\correction{ %début énoncé 
 Le triangle ABC est rectangle en B,, donc avec le théorème de Pythagore on a : 
\begin{eqnarray*}
AB^2 + BC^2 &=& AC^2 \\
5^2 + 3^2 &=& AC^2 \\
34 &=&AC^2 \\
& & \mbox{ Donc : } AC = \sqrt{34} 
\end{eqnarray*}
} % fin correction 

\item[2.] \enonce{ %début énoncé 
Calculer l'angle $\widehat{ABC}$ au degré près. } % fin énoncé 
\points{2.0} %en pourcentage
\end{description}

\psset{xunit=1.0cm,yunit=1.0cm,algebraic=true,dimen=middle,dotstyle=o,dotsize=5pt 0,linewidth=2.pt,arrowsize=3pt 2,arrowinset=0.25}
\begin{pspicture*}(-4.3,-3.02)(7.28,6.3)
\multips(0,-3)(0,1.0){10}{\psline[linestyle=dashed,linecap=1,dash=1.5pt 1.5pt,linewidth=0.4pt,linecolor=lightgray]{c-c}(-4.3,0)(7.28,0)}
\multips(-4,0)(1.0,0){12}{\psline[linestyle=dashed,linecap=1,dash=1.5pt 1.5pt,linewidth=0.4pt,linecolor=lightgray]{c-c}(0,-3.02)(0,6.3)}
\psaxes[labelFontSize=\scriptstyle,xAxis=true,yAxis=true,Dx=1.,Dy=1.,ticksize=-2pt 0,subticks=2]{->}(0,0)(-4.3,-3.02)(7.28,6.3)
\psplot[linewidth=2.pt]{-4.3}{7.28}{(-1+0.2*x)/1.}
%\psplot[linewidth=2.pt]{-4.3}{7.28}{(-2.--1.5*x)/1.}
\begin{scriptsize}
\rput[bl](5.08,5.98){$f$}
\end{scriptsize}
\end{pspicture*}













































\EXERCICE
\enonce{ %début énoncé 

Soit ABC un triangle rectangle en B. On sait que AB = 5 et BC = 3
} % fin énoncé 

\begin{description}
\item[1.] \enonce{ %début énoncé 
Calculer la longueur de AC \points{2.0} %en pourcentage
} % fin énoncé 

\correction{ %début énoncé 
 Le triangle ABC est rectangle en B,, donc avec le théorème de Pythagore on a : 
\begin{eqnarray*}
AB^2 + BC^2 &=& AC^2 \\
5^2 + 3^2 &=& AC^2 \\
34 &=&AC^2 \\
& & \mbox{ Donc : } AC = \sqrt{34} 
\end{eqnarray*}
} % fin correction 

\item[2.] \enonce{ %début énoncé 
Calculer l'angle $\widehat{ABC}$ au degré près. } % fin énoncé 
\points{2.0} %en pourcentage
\end{description}

\psset{xunit=1.0cm,yunit=1.0cm,algebraic=true,dimen=middle,dotstyle=o,dotsize=5pt 0,linewidth=2.pt,arrowsize=3pt 2,arrowinset=0.25}
\begin{pspicture*}(-4.3,-3.02)(7.28,6.3)
\multips(0,-3)(0,1.0){10}{\psline[linestyle=dashed,linecap=1,dash=1.5pt 1.5pt,linewidth=0.4pt,linecolor=lightgray]{c-c}(-4.3,0)(7.28,0)}
\multips(-4,0)(1.0,0){12}{\psline[linestyle=dashed,linecap=1,dash=1.5pt 1.5pt,linewidth=0.4pt,linecolor=lightgray]{c-c}(0,-3.02)(0,6.3)}
\psaxes[labelFontSize=\scriptstyle,xAxis=true,yAxis=true,Dx=1.,Dy=1.,ticksize=-2pt 0,subticks=2]{->}(0,0)(-4.3,-3.02)(7.28,6.3)
\psplot[linewidth=2.pt]{-4.3}{7.28}{(-1+0.4*x)/1.}
%\psplot[linewidth=2.pt]{-4.3}{7.28}{(-2.--1.5*x)/1.}
\begin{scriptsize}
\rput[bl](5.08,5.98){$f$}
\end{scriptsize}
\end{pspicture*}













































\EXERCICE
\enonce{ %début énoncé 

Soit ABC un triangle rectangle en B. On sait que AB = 5 et BC = 3
} % fin énoncé 

\begin{description}
\item[1.] \enonce{ %début énoncé 
Calculer la longueur de AC \points{2.0} %en pourcentage
} % fin énoncé 

\correction{ %début énoncé 
 Le triangle ABC est rectangle en B,, donc avec le théorème de Pythagore on a : 
\begin{eqnarray*}
AB^2 + BC^2 &=& AC^2 \\
5^2 + 3^2 &=& AC^2 \\
34 &=&AC^2 \\
& & \mbox{ Donc : } AC = \sqrt{34} 
\end{eqnarray*}
} % fin correction 

\item[2.] \enonce{ %début énoncé 
Calculer l'angle $\widehat{ABC}$ au degré près. } % fin énoncé 
\points{2.0} %en pourcentage
\end{description}

\psset{xunit=1.0cm,yunit=1.0cm,algebraic=true,dimen=middle,dotstyle=o,dotsize=5pt 0,linewidth=2.pt,arrowsize=3pt 2,arrowinset=0.25}
\begin{pspicture*}(-4.3,-3.02)(7.28,6.3)
\multips(0,-3)(0,1.0){10}{\psline[linestyle=dashed,linecap=1,dash=1.5pt 1.5pt,linewidth=0.4pt,linecolor=lightgray]{c-c}(-4.3,0)(7.28,0)}
\multips(-4,0)(1.0,0){12}{\psline[linestyle=dashed,linecap=1,dash=1.5pt 1.5pt,linewidth=0.4pt,linecolor=lightgray]{c-c}(0,-3.02)(0,6.3)}
\psaxes[labelFontSize=\scriptstyle,xAxis=true,yAxis=true,Dx=1.,Dy=1.,ticksize=-2pt 0,subticks=2]{->}(0,0)(-4.3,-3.02)(7.28,6.3)
\psplot[linewidth=2.pt]{-4.3}{7.28}{(0+-0.4*x)/1.}
%\psplot[linewidth=2.pt]{-4.3}{7.28}{(-2.--1.5*x)/1.}
\begin{scriptsize}
\rput[bl](5.08,5.98){$f$}
\end{scriptsize}
\end{pspicture*}













































\EXERCICE
\enonce{ %début énoncé 

Soit ABC un triangle rectangle en B. On sait que AB = 5 et BC = 3
} % fin énoncé 

\begin{description}
\item[1.] \enonce{ %début énoncé 
Calculer la longueur de AC \points{2.0} %en pourcentage
} % fin énoncé 

\correction{ %début énoncé 
 Le triangle ABC est rectangle en B,, donc avec le théorème de Pythagore on a : 
\begin{eqnarray*}
AB^2 + BC^2 &=& AC^2 \\
5^2 + 3^2 &=& AC^2 \\
34 &=&AC^2 \\
& & \mbox{ Donc : } AC = \sqrt{34} 
\end{eqnarray*}
} % fin correction 

\item[2.] \enonce{ %début énoncé 
Calculer l'angle $\widehat{ABC}$ au degré près. } % fin énoncé 
\points{2.0} %en pourcentage
\end{description}

\psset{xunit=1.0cm,yunit=1.0cm,algebraic=true,dimen=middle,dotstyle=o,dotsize=5pt 0,linewidth=2.pt,arrowsize=3pt 2,arrowinset=0.25}
\begin{pspicture*}(-4.3,-3.02)(7.28,6.3)
\multips(0,-3)(0,1.0){10}{\psline[linestyle=dashed,linecap=1,dash=1.5pt 1.5pt,linewidth=0.4pt,linecolor=lightgray]{c-c}(-4.3,0)(7.28,0)}
\multips(-4,0)(1.0,0){12}{\psline[linestyle=dashed,linecap=1,dash=1.5pt 1.5pt,linewidth=0.4pt,linecolor=lightgray]{c-c}(0,-3.02)(0,6.3)}
\psaxes[labelFontSize=\scriptstyle,xAxis=true,yAxis=true,Dx=1.,Dy=1.,ticksize=-2pt 0,subticks=2]{->}(0,0)(-4.3,-3.02)(7.28,6.3)
\psplot[linewidth=2.pt]{-4.3}{7.28}{(0+-0.2*x)/1.}
%\psplot[linewidth=2.pt]{-4.3}{7.28}{(-2.--1.5*x)/1.}
\begin{scriptsize}
\rput[bl](5.08,5.98){$f$}
\end{scriptsize}
\end{pspicture*}













































\EXERCICE
\enonce{ %début énoncé 

Soit ABC un triangle rectangle en B. On sait que AB = 5 et BC = 3
} % fin énoncé 

\begin{description}
\item[1.] \enonce{ %début énoncé 
Calculer la longueur de AC \points{2.0} %en pourcentage
} % fin énoncé 

\correction{ %début énoncé 
 Le triangle ABC est rectangle en B,, donc avec le théorème de Pythagore on a : 
\begin{eqnarray*}
AB^2 + BC^2 &=& AC^2 \\
5^2 + 3^2 &=& AC^2 \\
34 &=&AC^2 \\
& & \mbox{ Donc : } AC = \sqrt{34} 
\end{eqnarray*}
} % fin correction 

\item[2.] \enonce{ %début énoncé 
Calculer l'angle $\widehat{ABC}$ au degré près. } % fin énoncé 
\points{2.0} %en pourcentage
\end{description}

\psset{xunit=1.0cm,yunit=1.0cm,algebraic=true,dimen=middle,dotstyle=o,dotsize=5pt 0,linewidth=2.pt,arrowsize=3pt 2,arrowinset=0.25}
\begin{pspicture*}(-4.3,-3.02)(7.28,6.3)
\multips(0,-3)(0,1.0){10}{\psline[linestyle=dashed,linecap=1,dash=1.5pt 1.5pt,linewidth=0.4pt,linecolor=lightgray]{c-c}(-4.3,0)(7.28,0)}
\multips(-4,0)(1.0,0){12}{\psline[linestyle=dashed,linecap=1,dash=1.5pt 1.5pt,linewidth=0.4pt,linecolor=lightgray]{c-c}(0,-3.02)(0,6.3)}
\psaxes[labelFontSize=\scriptstyle,xAxis=true,yAxis=true,Dx=1.,Dy=1.,ticksize=-2pt 0,subticks=2]{->}(0,0)(-4.3,-3.02)(7.28,6.3)
\psplot[linewidth=2.pt]{-4.3}{7.28}{(0+0.0*x)/1.}
%\psplot[linewidth=2.pt]{-4.3}{7.28}{(-2.--1.5*x)/1.}
\begin{scriptsize}
\rput[bl](5.08,5.98){$f$}
\end{scriptsize}
\end{pspicture*}













































\EXERCICE
\enonce{ %début énoncé 

Soit ABC un triangle rectangle en B. On sait que AB = 5 et BC = 3
} % fin énoncé 

\begin{description}
\item[1.] \enonce{ %début énoncé 
Calculer la longueur de AC \points{2.0} %en pourcentage
} % fin énoncé 

\correction{ %début énoncé 
 Le triangle ABC est rectangle en B,, donc avec le théorème de Pythagore on a : 
\begin{eqnarray*}
AB^2 + BC^2 &=& AC^2 \\
5^2 + 3^2 &=& AC^2 \\
34 &=&AC^2 \\
& & \mbox{ Donc : } AC = \sqrt{34} 
\end{eqnarray*}
} % fin correction 

\item[2.] \enonce{ %début énoncé 
Calculer l'angle $\widehat{ABC}$ au degré près. } % fin énoncé 
\points{2.0} %en pourcentage
\end{description}

\psset{xunit=1.0cm,yunit=1.0cm,algebraic=true,dimen=middle,dotstyle=o,dotsize=5pt 0,linewidth=2.pt,arrowsize=3pt 2,arrowinset=0.25}
\begin{pspicture*}(-4.3,-3.02)(7.28,6.3)
\multips(0,-3)(0,1.0){10}{\psline[linestyle=dashed,linecap=1,dash=1.5pt 1.5pt,linewidth=0.4pt,linecolor=lightgray]{c-c}(-4.3,0)(7.28,0)}
\multips(-4,0)(1.0,0){12}{\psline[linestyle=dashed,linecap=1,dash=1.5pt 1.5pt,linewidth=0.4pt,linecolor=lightgray]{c-c}(0,-3.02)(0,6.3)}
\psaxes[labelFontSize=\scriptstyle,xAxis=true,yAxis=true,Dx=1.,Dy=1.,ticksize=-2pt 0,subticks=2]{->}(0,0)(-4.3,-3.02)(7.28,6.3)
\psplot[linewidth=2.pt]{-4.3}{7.28}{(0+0.2*x)/1.}
%\psplot[linewidth=2.pt]{-4.3}{7.28}{(-2.--1.5*x)/1.}
\begin{scriptsize}
\rput[bl](5.08,5.98){$f$}
\end{scriptsize}
\end{pspicture*}













































\EXERCICE
\enonce{ %début énoncé 

Soit ABC un triangle rectangle en B. On sait que AB = 5 et BC = 3
} % fin énoncé 

\begin{description}
\item[1.] \enonce{ %début énoncé 
Calculer la longueur de AC \points{2.0} %en pourcentage
} % fin énoncé 

\correction{ %début énoncé 
 Le triangle ABC est rectangle en B,, donc avec le théorème de Pythagore on a : 
\begin{eqnarray*}
AB^2 + BC^2 &=& AC^2 \\
5^2 + 3^2 &=& AC^2 \\
34 &=&AC^2 \\
& & \mbox{ Donc : } AC = \sqrt{34} 
\end{eqnarray*}
} % fin correction 

\item[2.] \enonce{ %début énoncé 
Calculer l'angle $\widehat{ABC}$ au degré près. } % fin énoncé 
\points{2.0} %en pourcentage
\end{description}

\psset{xunit=1.0cm,yunit=1.0cm,algebraic=true,dimen=middle,dotstyle=o,dotsize=5pt 0,linewidth=2.pt,arrowsize=3pt 2,arrowinset=0.25}
\begin{pspicture*}(-4.3,-3.02)(7.28,6.3)
\multips(0,-3)(0,1.0){10}{\psline[linestyle=dashed,linecap=1,dash=1.5pt 1.5pt,linewidth=0.4pt,linecolor=lightgray]{c-c}(-4.3,0)(7.28,0)}
\multips(-4,0)(1.0,0){12}{\psline[linestyle=dashed,linecap=1,dash=1.5pt 1.5pt,linewidth=0.4pt,linecolor=lightgray]{c-c}(0,-3.02)(0,6.3)}
\psaxes[labelFontSize=\scriptstyle,xAxis=true,yAxis=true,Dx=1.,Dy=1.,ticksize=-2pt 0,subticks=2]{->}(0,0)(-4.3,-3.02)(7.28,6.3)
\psplot[linewidth=2.pt]{-4.3}{7.28}{(0+0.4*x)/1.}
%\psplot[linewidth=2.pt]{-4.3}{7.28}{(-2.--1.5*x)/1.}
\begin{scriptsize}
\rput[bl](5.08,5.98){$f$}
\end{scriptsize}
\end{pspicture*}













































\EXERCICE
\enonce{ %début énoncé 

Soit ABC un triangle rectangle en B. On sait que AB = 5 et BC = 3
} % fin énoncé 

\begin{description}
\item[1.] \enonce{ %début énoncé 
Calculer la longueur de AC \points{2.0} %en pourcentage
} % fin énoncé 

\correction{ %début énoncé 
 Le triangle ABC est rectangle en B,, donc avec le théorème de Pythagore on a : 
\begin{eqnarray*}
AB^2 + BC^2 &=& AC^2 \\
5^2 + 3^2 &=& AC^2 \\
34 &=&AC^2 \\
& & \mbox{ Donc : } AC = \sqrt{34} 
\end{eqnarray*}
} % fin correction 

\item[2.] \enonce{ %début énoncé 
Calculer l'angle $\widehat{ABC}$ au degré près. } % fin énoncé 
\points{2.0} %en pourcentage
\end{description}

\psset{xunit=1.0cm,yunit=1.0cm,algebraic=true,dimen=middle,dotstyle=o,dotsize=5pt 0,linewidth=2.pt,arrowsize=3pt 2,arrowinset=0.25}
\begin{pspicture*}(-4.3,-3.02)(7.28,6.3)
\multips(0,-3)(0,1.0){10}{\psline[linestyle=dashed,linecap=1,dash=1.5pt 1.5pt,linewidth=0.4pt,linecolor=lightgray]{c-c}(-4.3,0)(7.28,0)}
\multips(-4,0)(1.0,0){12}{\psline[linestyle=dashed,linecap=1,dash=1.5pt 1.5pt,linewidth=0.4pt,linecolor=lightgray]{c-c}(0,-3.02)(0,6.3)}
\psaxes[labelFontSize=\scriptstyle,xAxis=true,yAxis=true,Dx=1.,Dy=1.,ticksize=-2pt 0,subticks=2]{->}(0,0)(-4.3,-3.02)(7.28,6.3)
\psplot[linewidth=2.pt]{-4.3}{7.28}{(1+-0.4*x)/1.}
%\psplot[linewidth=2.pt]{-4.3}{7.28}{(-2.--1.5*x)/1.}
\begin{scriptsize}
\rput[bl](5.08,5.98){$f$}
\end{scriptsize}
\end{pspicture*}













































\EXERCICE
\enonce{ %début énoncé 

Soit ABC un triangle rectangle en B. On sait que AB = 5 et BC = 3
} % fin énoncé 

\begin{description}
\item[1.] \enonce{ %début énoncé 
Calculer la longueur de AC \points{2.0} %en pourcentage
} % fin énoncé 

\correction{ %début énoncé 
 Le triangle ABC est rectangle en B,, donc avec le théorème de Pythagore on a : 
\begin{eqnarray*}
AB^2 + BC^2 &=& AC^2 \\
5^2 + 3^2 &=& AC^2 \\
34 &=&AC^2 \\
& & \mbox{ Donc : } AC = \sqrt{34} 
\end{eqnarray*}
} % fin correction 

\item[2.] \enonce{ %début énoncé 
Calculer l'angle $\widehat{ABC}$ au degré près. } % fin énoncé 
\points{2.0} %en pourcentage
\end{description}

\psset{xunit=1.0cm,yunit=1.0cm,algebraic=true,dimen=middle,dotstyle=o,dotsize=5pt 0,linewidth=2.pt,arrowsize=3pt 2,arrowinset=0.25}
\begin{pspicture*}(-4.3,-3.02)(7.28,6.3)
\multips(0,-3)(0,1.0){10}{\psline[linestyle=dashed,linecap=1,dash=1.5pt 1.5pt,linewidth=0.4pt,linecolor=lightgray]{c-c}(-4.3,0)(7.28,0)}
\multips(-4,0)(1.0,0){12}{\psline[linestyle=dashed,linecap=1,dash=1.5pt 1.5pt,linewidth=0.4pt,linecolor=lightgray]{c-c}(0,-3.02)(0,6.3)}
\psaxes[labelFontSize=\scriptstyle,xAxis=true,yAxis=true,Dx=1.,Dy=1.,ticksize=-2pt 0,subticks=2]{->}(0,0)(-4.3,-3.02)(7.28,6.3)
\psplot[linewidth=2.pt]{-4.3}{7.28}{(1+-0.2*x)/1.}
%\psplot[linewidth=2.pt]{-4.3}{7.28}{(-2.--1.5*x)/1.}
\begin{scriptsize}
\rput[bl](5.08,5.98){$f$}
\end{scriptsize}
\end{pspicture*}













































\EXERCICE
\enonce{ %début énoncé 

Soit ABC un triangle rectangle en B. On sait que AB = 5 et BC = 3
} % fin énoncé 

\begin{description}
\item[1.] \enonce{ %début énoncé 
Calculer la longueur de AC \points{2.0} %en pourcentage
} % fin énoncé 

\correction{ %début énoncé 
 Le triangle ABC est rectangle en B,, donc avec le théorème de Pythagore on a : 
\begin{eqnarray*}
AB^2 + BC^2 &=& AC^2 \\
5^2 + 3^2 &=& AC^2 \\
34 &=&AC^2 \\
& & \mbox{ Donc : } AC = \sqrt{34} 
\end{eqnarray*}
} % fin correction 

\item[2.] \enonce{ %début énoncé 
Calculer l'angle $\widehat{ABC}$ au degré près. } % fin énoncé 
\points{2.0} %en pourcentage
\end{description}

\psset{xunit=1.0cm,yunit=1.0cm,algebraic=true,dimen=middle,dotstyle=o,dotsize=5pt 0,linewidth=2.pt,arrowsize=3pt 2,arrowinset=0.25}
\begin{pspicture*}(-4.3,-3.02)(7.28,6.3)
\multips(0,-3)(0,1.0){10}{\psline[linestyle=dashed,linecap=1,dash=1.5pt 1.5pt,linewidth=0.4pt,linecolor=lightgray]{c-c}(-4.3,0)(7.28,0)}
\multips(-4,0)(1.0,0){12}{\psline[linestyle=dashed,linecap=1,dash=1.5pt 1.5pt,linewidth=0.4pt,linecolor=lightgray]{c-c}(0,-3.02)(0,6.3)}
\psaxes[labelFontSize=\scriptstyle,xAxis=true,yAxis=true,Dx=1.,Dy=1.,ticksize=-2pt 0,subticks=2]{->}(0,0)(-4.3,-3.02)(7.28,6.3)
\psplot[linewidth=2.pt]{-4.3}{7.28}{(1+0.0*x)/1.}
%\psplot[linewidth=2.pt]{-4.3}{7.28}{(-2.--1.5*x)/1.}
\begin{scriptsize}
\rput[bl](5.08,5.98){$f$}
\end{scriptsize}
\end{pspicture*}













































\EXERCICE
\enonce{ %début énoncé 

Soit ABC un triangle rectangle en B. On sait que AB = 5 et BC = 3
} % fin énoncé 

\begin{description}
\item[1.] \enonce{ %début énoncé 
Calculer la longueur de AC \points{2.0} %en pourcentage
} % fin énoncé 

\correction{ %début énoncé 
 Le triangle ABC est rectangle en B,, donc avec le théorème de Pythagore on a : 
\begin{eqnarray*}
AB^2 + BC^2 &=& AC^2 \\
5^2 + 3^2 &=& AC^2 \\
34 &=&AC^2 \\
& & \mbox{ Donc : } AC = \sqrt{34} 
\end{eqnarray*}
} % fin correction 

\item[2.] \enonce{ %début énoncé 
Calculer l'angle $\widehat{ABC}$ au degré près. } % fin énoncé 
\points{2.0} %en pourcentage
\end{description}

\psset{xunit=1.0cm,yunit=1.0cm,algebraic=true,dimen=middle,dotstyle=o,dotsize=5pt 0,linewidth=2.pt,arrowsize=3pt 2,arrowinset=0.25}
\begin{pspicture*}(-4.3,-3.02)(7.28,6.3)
\multips(0,-3)(0,1.0){10}{\psline[linestyle=dashed,linecap=1,dash=1.5pt 1.5pt,linewidth=0.4pt,linecolor=lightgray]{c-c}(-4.3,0)(7.28,0)}
\multips(-4,0)(1.0,0){12}{\psline[linestyle=dashed,linecap=1,dash=1.5pt 1.5pt,linewidth=0.4pt,linecolor=lightgray]{c-c}(0,-3.02)(0,6.3)}
\psaxes[labelFontSize=\scriptstyle,xAxis=true,yAxis=true,Dx=1.,Dy=1.,ticksize=-2pt 0,subticks=2]{->}(0,0)(-4.3,-3.02)(7.28,6.3)
\psplot[linewidth=2.pt]{-4.3}{7.28}{(1+0.2*x)/1.}
%\psplot[linewidth=2.pt]{-4.3}{7.28}{(-2.--1.5*x)/1.}
\begin{scriptsize}
\rput[bl](5.08,5.98){$f$}
\end{scriptsize}
\end{pspicture*}













































\EXERCICE
\enonce{ %début énoncé 

Soit ABC un triangle rectangle en B. On sait que AB = 5 et BC = 3
} % fin énoncé 

\begin{description}
\item[1.] \enonce{ %début énoncé 
Calculer la longueur de AC \points{2.0} %en pourcentage
} % fin énoncé 

\correction{ %début énoncé 
 Le triangle ABC est rectangle en B,, donc avec le théorème de Pythagore on a : 
\begin{eqnarray*}
AB^2 + BC^2 &=& AC^2 \\
5^2 + 3^2 &=& AC^2 \\
34 &=&AC^2 \\
& & \mbox{ Donc : } AC = \sqrt{34} 
\end{eqnarray*}
} % fin correction 

\item[2.] \enonce{ %début énoncé 
Calculer l'angle $\widehat{ABC}$ au degré près. } % fin énoncé 
\points{2.0} %en pourcentage
\end{description}

\psset{xunit=1.0cm,yunit=1.0cm,algebraic=true,dimen=middle,dotstyle=o,dotsize=5pt 0,linewidth=2.pt,arrowsize=3pt 2,arrowinset=0.25}
\begin{pspicture*}(-4.3,-3.02)(7.28,6.3)
\multips(0,-3)(0,1.0){10}{\psline[linestyle=dashed,linecap=1,dash=1.5pt 1.5pt,linewidth=0.4pt,linecolor=lightgray]{c-c}(-4.3,0)(7.28,0)}
\multips(-4,0)(1.0,0){12}{\psline[linestyle=dashed,linecap=1,dash=1.5pt 1.5pt,linewidth=0.4pt,linecolor=lightgray]{c-c}(0,-3.02)(0,6.3)}
\psaxes[labelFontSize=\scriptstyle,xAxis=true,yAxis=true,Dx=1.,Dy=1.,ticksize=-2pt 0,subticks=2]{->}(0,0)(-4.3,-3.02)(7.28,6.3)
\psplot[linewidth=2.pt]{-4.3}{7.28}{(1+0.4*x)/1.}
%\psplot[linewidth=2.pt]{-4.3}{7.28}{(-2.--1.5*x)/1.}
\begin{scriptsize}
\rput[bl](5.08,5.98){$f$}
\end{scriptsize}
\end{pspicture*}













































\EXERCICE
\enonce{ %début énoncé 

Soit ABC un triangle rectangle en B. On sait que AB = 5 et BC = 4
} % fin énoncé 

\begin{description}
\item[1.] \enonce{ %début énoncé 
Calculer la longueur de AC \points{2.0} %en pourcentage
} % fin énoncé 

\correction{ %début énoncé 
 Le triangle ABC est rectangle en B,, donc avec le théorème de Pythagore on a : 
\begin{eqnarray*}
AB^2 + BC^2 &=& AC^2 \\
5^2 + 4^2 &=& AC^2 \\
41 &=&AC^2 \\
& & \mbox{ Donc : } AC = \sqrt{41} 
\end{eqnarray*}
} % fin correction 

\item[2.] \enonce{ %début énoncé 
Calculer l'angle $\widehat{ABC}$ au degré près. } % fin énoncé 
\points{2.0} %en pourcentage
\end{description}

\psset{xunit=1.0cm,yunit=1.0cm,algebraic=true,dimen=middle,dotstyle=o,dotsize=5pt 0,linewidth=2.pt,arrowsize=3pt 2,arrowinset=0.25}
\begin{pspicture*}(-4.3,-3.02)(7.28,6.3)
\multips(0,-3)(0,1.0){10}{\psline[linestyle=dashed,linecap=1,dash=1.5pt 1.5pt,linewidth=0.4pt,linecolor=lightgray]{c-c}(-4.3,0)(7.28,0)}
\multips(-4,0)(1.0,0){12}{\psline[linestyle=dashed,linecap=1,dash=1.5pt 1.5pt,linewidth=0.4pt,linecolor=lightgray]{c-c}(0,-3.02)(0,6.3)}
\psaxes[labelFontSize=\scriptstyle,xAxis=true,yAxis=true,Dx=1.,Dy=1.,ticksize=-2pt 0,subticks=2]{->}(0,0)(-4.3,-3.02)(7.28,6.3)
\psplot[linewidth=2.pt]{-4.3}{7.28}{(-1+-0.4*x)/1.}
%\psplot[linewidth=2.pt]{-4.3}{7.28}{(-2.--1.5*x)/1.}
\begin{scriptsize}
\rput[bl](5.08,5.98){$f$}
\end{scriptsize}
\end{pspicture*}













































\EXERCICE
\enonce{ %début énoncé 

Soit ABC un triangle rectangle en B. On sait que AB = 5 et BC = 4
} % fin énoncé 

\begin{description}
\item[1.] \enonce{ %début énoncé 
Calculer la longueur de AC \points{2.0} %en pourcentage
} % fin énoncé 

\correction{ %début énoncé 
 Le triangle ABC est rectangle en B,, donc avec le théorème de Pythagore on a : 
\begin{eqnarray*}
AB^2 + BC^2 &=& AC^2 \\
5^2 + 4^2 &=& AC^2 \\
41 &=&AC^2 \\
& & \mbox{ Donc : } AC = \sqrt{41} 
\end{eqnarray*}
} % fin correction 

\item[2.] \enonce{ %début énoncé 
Calculer l'angle $\widehat{ABC}$ au degré près. } % fin énoncé 
\points{2.0} %en pourcentage
\end{description}

\psset{xunit=1.0cm,yunit=1.0cm,algebraic=true,dimen=middle,dotstyle=o,dotsize=5pt 0,linewidth=2.pt,arrowsize=3pt 2,arrowinset=0.25}
\begin{pspicture*}(-4.3,-3.02)(7.28,6.3)
\multips(0,-3)(0,1.0){10}{\psline[linestyle=dashed,linecap=1,dash=1.5pt 1.5pt,linewidth=0.4pt,linecolor=lightgray]{c-c}(-4.3,0)(7.28,0)}
\multips(-4,0)(1.0,0){12}{\psline[linestyle=dashed,linecap=1,dash=1.5pt 1.5pt,linewidth=0.4pt,linecolor=lightgray]{c-c}(0,-3.02)(0,6.3)}
\psaxes[labelFontSize=\scriptstyle,xAxis=true,yAxis=true,Dx=1.,Dy=1.,ticksize=-2pt 0,subticks=2]{->}(0,0)(-4.3,-3.02)(7.28,6.3)
\psplot[linewidth=2.pt]{-4.3}{7.28}{(-1+-0.2*x)/1.}
%\psplot[linewidth=2.pt]{-4.3}{7.28}{(-2.--1.5*x)/1.}
\begin{scriptsize}
\rput[bl](5.08,5.98){$f$}
\end{scriptsize}
\end{pspicture*}













































\EXERCICE
\enonce{ %début énoncé 

Soit ABC un triangle rectangle en B. On sait que AB = 5 et BC = 4
} % fin énoncé 

\begin{description}
\item[1.] \enonce{ %début énoncé 
Calculer la longueur de AC \points{2.0} %en pourcentage
} % fin énoncé 

\correction{ %début énoncé 
 Le triangle ABC est rectangle en B,, donc avec le théorème de Pythagore on a : 
\begin{eqnarray*}
AB^2 + BC^2 &=& AC^2 \\
5^2 + 4^2 &=& AC^2 \\
41 &=&AC^2 \\
& & \mbox{ Donc : } AC = \sqrt{41} 
\end{eqnarray*}
} % fin correction 

\item[2.] \enonce{ %début énoncé 
Calculer l'angle $\widehat{ABC}$ au degré près. } % fin énoncé 
\points{2.0} %en pourcentage
\end{description}

\psset{xunit=1.0cm,yunit=1.0cm,algebraic=true,dimen=middle,dotstyle=o,dotsize=5pt 0,linewidth=2.pt,arrowsize=3pt 2,arrowinset=0.25}
\begin{pspicture*}(-4.3,-3.02)(7.28,6.3)
\multips(0,-3)(0,1.0){10}{\psline[linestyle=dashed,linecap=1,dash=1.5pt 1.5pt,linewidth=0.4pt,linecolor=lightgray]{c-c}(-4.3,0)(7.28,0)}
\multips(-4,0)(1.0,0){12}{\psline[linestyle=dashed,linecap=1,dash=1.5pt 1.5pt,linewidth=0.4pt,linecolor=lightgray]{c-c}(0,-3.02)(0,6.3)}
\psaxes[labelFontSize=\scriptstyle,xAxis=true,yAxis=true,Dx=1.,Dy=1.,ticksize=-2pt 0,subticks=2]{->}(0,0)(-4.3,-3.02)(7.28,6.3)
\psplot[linewidth=2.pt]{-4.3}{7.28}{(-1+0.0*x)/1.}
%\psplot[linewidth=2.pt]{-4.3}{7.28}{(-2.--1.5*x)/1.}
\begin{scriptsize}
\rput[bl](5.08,5.98){$f$}
\end{scriptsize}
\end{pspicture*}













































\EXERCICE
\enonce{ %début énoncé 

Soit ABC un triangle rectangle en B. On sait que AB = 5 et BC = 4
} % fin énoncé 

\begin{description}
\item[1.] \enonce{ %début énoncé 
Calculer la longueur de AC \points{2.0} %en pourcentage
} % fin énoncé 

\correction{ %début énoncé 
 Le triangle ABC est rectangle en B,, donc avec le théorème de Pythagore on a : 
\begin{eqnarray*}
AB^2 + BC^2 &=& AC^2 \\
5^2 + 4^2 &=& AC^2 \\
41 &=&AC^2 \\
& & \mbox{ Donc : } AC = \sqrt{41} 
\end{eqnarray*}
} % fin correction 

\item[2.] \enonce{ %début énoncé 
Calculer l'angle $\widehat{ABC}$ au degré près. } % fin énoncé 
\points{2.0} %en pourcentage
\end{description}

\psset{xunit=1.0cm,yunit=1.0cm,algebraic=true,dimen=middle,dotstyle=o,dotsize=5pt 0,linewidth=2.pt,arrowsize=3pt 2,arrowinset=0.25}
\begin{pspicture*}(-4.3,-3.02)(7.28,6.3)
\multips(0,-3)(0,1.0){10}{\psline[linestyle=dashed,linecap=1,dash=1.5pt 1.5pt,linewidth=0.4pt,linecolor=lightgray]{c-c}(-4.3,0)(7.28,0)}
\multips(-4,0)(1.0,0){12}{\psline[linestyle=dashed,linecap=1,dash=1.5pt 1.5pt,linewidth=0.4pt,linecolor=lightgray]{c-c}(0,-3.02)(0,6.3)}
\psaxes[labelFontSize=\scriptstyle,xAxis=true,yAxis=true,Dx=1.,Dy=1.,ticksize=-2pt 0,subticks=2]{->}(0,0)(-4.3,-3.02)(7.28,6.3)
\psplot[linewidth=2.pt]{-4.3}{7.28}{(-1+0.2*x)/1.}
%\psplot[linewidth=2.pt]{-4.3}{7.28}{(-2.--1.5*x)/1.}
\begin{scriptsize}
\rput[bl](5.08,5.98){$f$}
\end{scriptsize}
\end{pspicture*}













































\EXERCICE
\enonce{ %début énoncé 

Soit ABC un triangle rectangle en B. On sait que AB = 5 et BC = 4
} % fin énoncé 

\begin{description}
\item[1.] \enonce{ %début énoncé 
Calculer la longueur de AC \points{2.0} %en pourcentage
} % fin énoncé 

\correction{ %début énoncé 
 Le triangle ABC est rectangle en B,, donc avec le théorème de Pythagore on a : 
\begin{eqnarray*}
AB^2 + BC^2 &=& AC^2 \\
5^2 + 4^2 &=& AC^2 \\
41 &=&AC^2 \\
& & \mbox{ Donc : } AC = \sqrt{41} 
\end{eqnarray*}
} % fin correction 

\item[2.] \enonce{ %début énoncé 
Calculer l'angle $\widehat{ABC}$ au degré près. } % fin énoncé 
\points{2.0} %en pourcentage
\end{description}

\psset{xunit=1.0cm,yunit=1.0cm,algebraic=true,dimen=middle,dotstyle=o,dotsize=5pt 0,linewidth=2.pt,arrowsize=3pt 2,arrowinset=0.25}
\begin{pspicture*}(-4.3,-3.02)(7.28,6.3)
\multips(0,-3)(0,1.0){10}{\psline[linestyle=dashed,linecap=1,dash=1.5pt 1.5pt,linewidth=0.4pt,linecolor=lightgray]{c-c}(-4.3,0)(7.28,0)}
\multips(-4,0)(1.0,0){12}{\psline[linestyle=dashed,linecap=1,dash=1.5pt 1.5pt,linewidth=0.4pt,linecolor=lightgray]{c-c}(0,-3.02)(0,6.3)}
\psaxes[labelFontSize=\scriptstyle,xAxis=true,yAxis=true,Dx=1.,Dy=1.,ticksize=-2pt 0,subticks=2]{->}(0,0)(-4.3,-3.02)(7.28,6.3)
\psplot[linewidth=2.pt]{-4.3}{7.28}{(-1+0.4*x)/1.}
%\psplot[linewidth=2.pt]{-4.3}{7.28}{(-2.--1.5*x)/1.}
\begin{scriptsize}
\rput[bl](5.08,5.98){$f$}
\end{scriptsize}
\end{pspicture*}













































\EXERCICE
\enonce{ %début énoncé 

Soit ABC un triangle rectangle en B. On sait que AB = 5 et BC = 4
} % fin énoncé 

\begin{description}
\item[1.] \enonce{ %début énoncé 
Calculer la longueur de AC \points{2.0} %en pourcentage
} % fin énoncé 

\correction{ %début énoncé 
 Le triangle ABC est rectangle en B,, donc avec le théorème de Pythagore on a : 
\begin{eqnarray*}
AB^2 + BC^2 &=& AC^2 \\
5^2 + 4^2 &=& AC^2 \\
41 &=&AC^2 \\
& & \mbox{ Donc : } AC = \sqrt{41} 
\end{eqnarray*}
} % fin correction 

\item[2.] \enonce{ %début énoncé 
Calculer l'angle $\widehat{ABC}$ au degré près. } % fin énoncé 
\points{2.0} %en pourcentage
\end{description}

\psset{xunit=1.0cm,yunit=1.0cm,algebraic=true,dimen=middle,dotstyle=o,dotsize=5pt 0,linewidth=2.pt,arrowsize=3pt 2,arrowinset=0.25}
\begin{pspicture*}(-4.3,-3.02)(7.28,6.3)
\multips(0,-3)(0,1.0){10}{\psline[linestyle=dashed,linecap=1,dash=1.5pt 1.5pt,linewidth=0.4pt,linecolor=lightgray]{c-c}(-4.3,0)(7.28,0)}
\multips(-4,0)(1.0,0){12}{\psline[linestyle=dashed,linecap=1,dash=1.5pt 1.5pt,linewidth=0.4pt,linecolor=lightgray]{c-c}(0,-3.02)(0,6.3)}
\psaxes[labelFontSize=\scriptstyle,xAxis=true,yAxis=true,Dx=1.,Dy=1.,ticksize=-2pt 0,subticks=2]{->}(0,0)(-4.3,-3.02)(7.28,6.3)
\psplot[linewidth=2.pt]{-4.3}{7.28}{(0+-0.4*x)/1.}
%\psplot[linewidth=2.pt]{-4.3}{7.28}{(-2.--1.5*x)/1.}
\begin{scriptsize}
\rput[bl](5.08,5.98){$f$}
\end{scriptsize}
\end{pspicture*}













































\EXERCICE
\enonce{ %début énoncé 

Soit ABC un triangle rectangle en B. On sait que AB = 5 et BC = 4
} % fin énoncé 

\begin{description}
\item[1.] \enonce{ %début énoncé 
Calculer la longueur de AC \points{2.0} %en pourcentage
} % fin énoncé 

\correction{ %début énoncé 
 Le triangle ABC est rectangle en B,, donc avec le théorème de Pythagore on a : 
\begin{eqnarray*}
AB^2 + BC^2 &=& AC^2 \\
5^2 + 4^2 &=& AC^2 \\
41 &=&AC^2 \\
& & \mbox{ Donc : } AC = \sqrt{41} 
\end{eqnarray*}
} % fin correction 

\item[2.] \enonce{ %début énoncé 
Calculer l'angle $\widehat{ABC}$ au degré près. } % fin énoncé 
\points{2.0} %en pourcentage
\end{description}

\psset{xunit=1.0cm,yunit=1.0cm,algebraic=true,dimen=middle,dotstyle=o,dotsize=5pt 0,linewidth=2.pt,arrowsize=3pt 2,arrowinset=0.25}
\begin{pspicture*}(-4.3,-3.02)(7.28,6.3)
\multips(0,-3)(0,1.0){10}{\psline[linestyle=dashed,linecap=1,dash=1.5pt 1.5pt,linewidth=0.4pt,linecolor=lightgray]{c-c}(-4.3,0)(7.28,0)}
\multips(-4,0)(1.0,0){12}{\psline[linestyle=dashed,linecap=1,dash=1.5pt 1.5pt,linewidth=0.4pt,linecolor=lightgray]{c-c}(0,-3.02)(0,6.3)}
\psaxes[labelFontSize=\scriptstyle,xAxis=true,yAxis=true,Dx=1.,Dy=1.,ticksize=-2pt 0,subticks=2]{->}(0,0)(-4.3,-3.02)(7.28,6.3)
\psplot[linewidth=2.pt]{-4.3}{7.28}{(0+-0.2*x)/1.}
%\psplot[linewidth=2.pt]{-4.3}{7.28}{(-2.--1.5*x)/1.}
\begin{scriptsize}
\rput[bl](5.08,5.98){$f$}
\end{scriptsize}
\end{pspicture*}













































\EXERCICE
\enonce{ %début énoncé 

Soit ABC un triangle rectangle en B. On sait que AB = 5 et BC = 4
} % fin énoncé 

\begin{description}
\item[1.] \enonce{ %début énoncé 
Calculer la longueur de AC \points{2.0} %en pourcentage
} % fin énoncé 

\correction{ %début énoncé 
 Le triangle ABC est rectangle en B,, donc avec le théorème de Pythagore on a : 
\begin{eqnarray*}
AB^2 + BC^2 &=& AC^2 \\
5^2 + 4^2 &=& AC^2 \\
41 &=&AC^2 \\
& & \mbox{ Donc : } AC = \sqrt{41} 
\end{eqnarray*}
} % fin correction 

\item[2.] \enonce{ %début énoncé 
Calculer l'angle $\widehat{ABC}$ au degré près. } % fin énoncé 
\points{2.0} %en pourcentage
\end{description}

\psset{xunit=1.0cm,yunit=1.0cm,algebraic=true,dimen=middle,dotstyle=o,dotsize=5pt 0,linewidth=2.pt,arrowsize=3pt 2,arrowinset=0.25}
\begin{pspicture*}(-4.3,-3.02)(7.28,6.3)
\multips(0,-3)(0,1.0){10}{\psline[linestyle=dashed,linecap=1,dash=1.5pt 1.5pt,linewidth=0.4pt,linecolor=lightgray]{c-c}(-4.3,0)(7.28,0)}
\multips(-4,0)(1.0,0){12}{\psline[linestyle=dashed,linecap=1,dash=1.5pt 1.5pt,linewidth=0.4pt,linecolor=lightgray]{c-c}(0,-3.02)(0,6.3)}
\psaxes[labelFontSize=\scriptstyle,xAxis=true,yAxis=true,Dx=1.,Dy=1.,ticksize=-2pt 0,subticks=2]{->}(0,0)(-4.3,-3.02)(7.28,6.3)
\psplot[linewidth=2.pt]{-4.3}{7.28}{(0+0.0*x)/1.}
%\psplot[linewidth=2.pt]{-4.3}{7.28}{(-2.--1.5*x)/1.}
\begin{scriptsize}
\rput[bl](5.08,5.98){$f$}
\end{scriptsize}
\end{pspicture*}













































\EXERCICE
\enonce{ %début énoncé 

Soit ABC un triangle rectangle en B. On sait que AB = 5 et BC = 4
} % fin énoncé 

\begin{description}
\item[1.] \enonce{ %début énoncé 
Calculer la longueur de AC \points{2.0} %en pourcentage
} % fin énoncé 

\correction{ %début énoncé 
 Le triangle ABC est rectangle en B,, donc avec le théorème de Pythagore on a : 
\begin{eqnarray*}
AB^2 + BC^2 &=& AC^2 \\
5^2 + 4^2 &=& AC^2 \\
41 &=&AC^2 \\
& & \mbox{ Donc : } AC = \sqrt{41} 
\end{eqnarray*}
} % fin correction 

\item[2.] \enonce{ %début énoncé 
Calculer l'angle $\widehat{ABC}$ au degré près. } % fin énoncé 
\points{2.0} %en pourcentage
\end{description}

\psset{xunit=1.0cm,yunit=1.0cm,algebraic=true,dimen=middle,dotstyle=o,dotsize=5pt 0,linewidth=2.pt,arrowsize=3pt 2,arrowinset=0.25}
\begin{pspicture*}(-4.3,-3.02)(7.28,6.3)
\multips(0,-3)(0,1.0){10}{\psline[linestyle=dashed,linecap=1,dash=1.5pt 1.5pt,linewidth=0.4pt,linecolor=lightgray]{c-c}(-4.3,0)(7.28,0)}
\multips(-4,0)(1.0,0){12}{\psline[linestyle=dashed,linecap=1,dash=1.5pt 1.5pt,linewidth=0.4pt,linecolor=lightgray]{c-c}(0,-3.02)(0,6.3)}
\psaxes[labelFontSize=\scriptstyle,xAxis=true,yAxis=true,Dx=1.,Dy=1.,ticksize=-2pt 0,subticks=2]{->}(0,0)(-4.3,-3.02)(7.28,6.3)
\psplot[linewidth=2.pt]{-4.3}{7.28}{(0+0.2*x)/1.}
%\psplot[linewidth=2.pt]{-4.3}{7.28}{(-2.--1.5*x)/1.}
\begin{scriptsize}
\rput[bl](5.08,5.98){$f$}
\end{scriptsize}
\end{pspicture*}













































\EXERCICE
\enonce{ %début énoncé 

Soit ABC un triangle rectangle en B. On sait que AB = 5 et BC = 4
} % fin énoncé 

\begin{description}
\item[1.] \enonce{ %début énoncé 
Calculer la longueur de AC \points{2.0} %en pourcentage
} % fin énoncé 

\correction{ %début énoncé 
 Le triangle ABC est rectangle en B,, donc avec le théorème de Pythagore on a : 
\begin{eqnarray*}
AB^2 + BC^2 &=& AC^2 \\
5^2 + 4^2 &=& AC^2 \\
41 &=&AC^2 \\
& & \mbox{ Donc : } AC = \sqrt{41} 
\end{eqnarray*}
} % fin correction 

\item[2.] \enonce{ %début énoncé 
Calculer l'angle $\widehat{ABC}$ au degré près. } % fin énoncé 
\points{2.0} %en pourcentage
\end{description}

\psset{xunit=1.0cm,yunit=1.0cm,algebraic=true,dimen=middle,dotstyle=o,dotsize=5pt 0,linewidth=2.pt,arrowsize=3pt 2,arrowinset=0.25}
\begin{pspicture*}(-4.3,-3.02)(7.28,6.3)
\multips(0,-3)(0,1.0){10}{\psline[linestyle=dashed,linecap=1,dash=1.5pt 1.5pt,linewidth=0.4pt,linecolor=lightgray]{c-c}(-4.3,0)(7.28,0)}
\multips(-4,0)(1.0,0){12}{\psline[linestyle=dashed,linecap=1,dash=1.5pt 1.5pt,linewidth=0.4pt,linecolor=lightgray]{c-c}(0,-3.02)(0,6.3)}
\psaxes[labelFontSize=\scriptstyle,xAxis=true,yAxis=true,Dx=1.,Dy=1.,ticksize=-2pt 0,subticks=2]{->}(0,0)(-4.3,-3.02)(7.28,6.3)
\psplot[linewidth=2.pt]{-4.3}{7.28}{(0+0.4*x)/1.}
%\psplot[linewidth=2.pt]{-4.3}{7.28}{(-2.--1.5*x)/1.}
\begin{scriptsize}
\rput[bl](5.08,5.98){$f$}
\end{scriptsize}
\end{pspicture*}













































\EXERCICE
\enonce{ %début énoncé 

Soit ABC un triangle rectangle en B. On sait que AB = 5 et BC = 4
} % fin énoncé 

\begin{description}
\item[1.] \enonce{ %début énoncé 
Calculer la longueur de AC \points{2.0} %en pourcentage
} % fin énoncé 

\correction{ %début énoncé 
 Le triangle ABC est rectangle en B,, donc avec le théorème de Pythagore on a : 
\begin{eqnarray*}
AB^2 + BC^2 &=& AC^2 \\
5^2 + 4^2 &=& AC^2 \\
41 &=&AC^2 \\
& & \mbox{ Donc : } AC = \sqrt{41} 
\end{eqnarray*}
} % fin correction 

\item[2.] \enonce{ %début énoncé 
Calculer l'angle $\widehat{ABC}$ au degré près. } % fin énoncé 
\points{2.0} %en pourcentage
\end{description}

\psset{xunit=1.0cm,yunit=1.0cm,algebraic=true,dimen=middle,dotstyle=o,dotsize=5pt 0,linewidth=2.pt,arrowsize=3pt 2,arrowinset=0.25}
\begin{pspicture*}(-4.3,-3.02)(7.28,6.3)
\multips(0,-3)(0,1.0){10}{\psline[linestyle=dashed,linecap=1,dash=1.5pt 1.5pt,linewidth=0.4pt,linecolor=lightgray]{c-c}(-4.3,0)(7.28,0)}
\multips(-4,0)(1.0,0){12}{\psline[linestyle=dashed,linecap=1,dash=1.5pt 1.5pt,linewidth=0.4pt,linecolor=lightgray]{c-c}(0,-3.02)(0,6.3)}
\psaxes[labelFontSize=\scriptstyle,xAxis=true,yAxis=true,Dx=1.,Dy=1.,ticksize=-2pt 0,subticks=2]{->}(0,0)(-4.3,-3.02)(7.28,6.3)
\psplot[linewidth=2.pt]{-4.3}{7.28}{(1+-0.4*x)/1.}
%\psplot[linewidth=2.pt]{-4.3}{7.28}{(-2.--1.5*x)/1.}
\begin{scriptsize}
\rput[bl](5.08,5.98){$f$}
\end{scriptsize}
\end{pspicture*}













































\EXERCICE
\enonce{ %début énoncé 

Soit ABC un triangle rectangle en B. On sait que AB = 5 et BC = 4
} % fin énoncé 

\begin{description}
\item[1.] \enonce{ %début énoncé 
Calculer la longueur de AC \points{2.0} %en pourcentage
} % fin énoncé 

\correction{ %début énoncé 
 Le triangle ABC est rectangle en B,, donc avec le théorème de Pythagore on a : 
\begin{eqnarray*}
AB^2 + BC^2 &=& AC^2 \\
5^2 + 4^2 &=& AC^2 \\
41 &=&AC^2 \\
& & \mbox{ Donc : } AC = \sqrt{41} 
\end{eqnarray*}
} % fin correction 

\item[2.] \enonce{ %début énoncé 
Calculer l'angle $\widehat{ABC}$ au degré près. } % fin énoncé 
\points{2.0} %en pourcentage
\end{description}

\psset{xunit=1.0cm,yunit=1.0cm,algebraic=true,dimen=middle,dotstyle=o,dotsize=5pt 0,linewidth=2.pt,arrowsize=3pt 2,arrowinset=0.25}
\begin{pspicture*}(-4.3,-3.02)(7.28,6.3)
\multips(0,-3)(0,1.0){10}{\psline[linestyle=dashed,linecap=1,dash=1.5pt 1.5pt,linewidth=0.4pt,linecolor=lightgray]{c-c}(-4.3,0)(7.28,0)}
\multips(-4,0)(1.0,0){12}{\psline[linestyle=dashed,linecap=1,dash=1.5pt 1.5pt,linewidth=0.4pt,linecolor=lightgray]{c-c}(0,-3.02)(0,6.3)}
\psaxes[labelFontSize=\scriptstyle,xAxis=true,yAxis=true,Dx=1.,Dy=1.,ticksize=-2pt 0,subticks=2]{->}(0,0)(-4.3,-3.02)(7.28,6.3)
\psplot[linewidth=2.pt]{-4.3}{7.28}{(1+-0.2*x)/1.}
%\psplot[linewidth=2.pt]{-4.3}{7.28}{(-2.--1.5*x)/1.}
\begin{scriptsize}
\rput[bl](5.08,5.98){$f$}
\end{scriptsize}
\end{pspicture*}













































\EXERCICE
\enonce{ %début énoncé 

Soit ABC un triangle rectangle en B. On sait que AB = 5 et BC = 4
} % fin énoncé 

\begin{description}
\item[1.] \enonce{ %début énoncé 
Calculer la longueur de AC \points{2.0} %en pourcentage
} % fin énoncé 

\correction{ %début énoncé 
 Le triangle ABC est rectangle en B,, donc avec le théorème de Pythagore on a : 
\begin{eqnarray*}
AB^2 + BC^2 &=& AC^2 \\
5^2 + 4^2 &=& AC^2 \\
41 &=&AC^2 \\
& & \mbox{ Donc : } AC = \sqrt{41} 
\end{eqnarray*}
} % fin correction 

\item[2.] \enonce{ %début énoncé 
Calculer l'angle $\widehat{ABC}$ au degré près. } % fin énoncé 
\points{2.0} %en pourcentage
\end{description}

\psset{xunit=1.0cm,yunit=1.0cm,algebraic=true,dimen=middle,dotstyle=o,dotsize=5pt 0,linewidth=2.pt,arrowsize=3pt 2,arrowinset=0.25}
\begin{pspicture*}(-4.3,-3.02)(7.28,6.3)
\multips(0,-3)(0,1.0){10}{\psline[linestyle=dashed,linecap=1,dash=1.5pt 1.5pt,linewidth=0.4pt,linecolor=lightgray]{c-c}(-4.3,0)(7.28,0)}
\multips(-4,0)(1.0,0){12}{\psline[linestyle=dashed,linecap=1,dash=1.5pt 1.5pt,linewidth=0.4pt,linecolor=lightgray]{c-c}(0,-3.02)(0,6.3)}
\psaxes[labelFontSize=\scriptstyle,xAxis=true,yAxis=true,Dx=1.,Dy=1.,ticksize=-2pt 0,subticks=2]{->}(0,0)(-4.3,-3.02)(7.28,6.3)
\psplot[linewidth=2.pt]{-4.3}{7.28}{(1+0.0*x)/1.}
%\psplot[linewidth=2.pt]{-4.3}{7.28}{(-2.--1.5*x)/1.}
\begin{scriptsize}
\rput[bl](5.08,5.98){$f$}
\end{scriptsize}
\end{pspicture*}













































\EXERCICE
\enonce{ %début énoncé 

Soit ABC un triangle rectangle en B. On sait que AB = 5 et BC = 4
} % fin énoncé 

\begin{description}
\item[1.] \enonce{ %début énoncé 
Calculer la longueur de AC \points{2.0} %en pourcentage
} % fin énoncé 

\correction{ %début énoncé 
 Le triangle ABC est rectangle en B,, donc avec le théorème de Pythagore on a : 
\begin{eqnarray*}
AB^2 + BC^2 &=& AC^2 \\
5^2 + 4^2 &=& AC^2 \\
41 &=&AC^2 \\
& & \mbox{ Donc : } AC = \sqrt{41} 
\end{eqnarray*}
} % fin correction 

\item[2.] \enonce{ %début énoncé 
Calculer l'angle $\widehat{ABC}$ au degré près. } % fin énoncé 
\points{2.0} %en pourcentage
\end{description}

\psset{xunit=1.0cm,yunit=1.0cm,algebraic=true,dimen=middle,dotstyle=o,dotsize=5pt 0,linewidth=2.pt,arrowsize=3pt 2,arrowinset=0.25}
\begin{pspicture*}(-4.3,-3.02)(7.28,6.3)
\multips(0,-3)(0,1.0){10}{\psline[linestyle=dashed,linecap=1,dash=1.5pt 1.5pt,linewidth=0.4pt,linecolor=lightgray]{c-c}(-4.3,0)(7.28,0)}
\multips(-4,0)(1.0,0){12}{\psline[linestyle=dashed,linecap=1,dash=1.5pt 1.5pt,linewidth=0.4pt,linecolor=lightgray]{c-c}(0,-3.02)(0,6.3)}
\psaxes[labelFontSize=\scriptstyle,xAxis=true,yAxis=true,Dx=1.,Dy=1.,ticksize=-2pt 0,subticks=2]{->}(0,0)(-4.3,-3.02)(7.28,6.3)
\psplot[linewidth=2.pt]{-4.3}{7.28}{(1+0.2*x)/1.}
%\psplot[linewidth=2.pt]{-4.3}{7.28}{(-2.--1.5*x)/1.}
\begin{scriptsize}
\rput[bl](5.08,5.98){$f$}
\end{scriptsize}
\end{pspicture*}













































\EXERCICE
\enonce{ %début énoncé 

Soit ABC un triangle rectangle en B. On sait que AB = 5 et BC = 4
} % fin énoncé 

\begin{description}
\item[1.] \enonce{ %début énoncé 
Calculer la longueur de AC \points{2.0} %en pourcentage
} % fin énoncé 

\correction{ %début énoncé 
 Le triangle ABC est rectangle en B,, donc avec le théorème de Pythagore on a : 
\begin{eqnarray*}
AB^2 + BC^2 &=& AC^2 \\
5^2 + 4^2 &=& AC^2 \\
41 &=&AC^2 \\
& & \mbox{ Donc : } AC = \sqrt{41} 
\end{eqnarray*}
} % fin correction 

\item[2.] \enonce{ %début énoncé 
Calculer l'angle $\widehat{ABC}$ au degré près. } % fin énoncé 
\points{2.0} %en pourcentage
\end{description}

\psset{xunit=1.0cm,yunit=1.0cm,algebraic=true,dimen=middle,dotstyle=o,dotsize=5pt 0,linewidth=2.pt,arrowsize=3pt 2,arrowinset=0.25}
\begin{pspicture*}(-4.3,-3.02)(7.28,6.3)
\multips(0,-3)(0,1.0){10}{\psline[linestyle=dashed,linecap=1,dash=1.5pt 1.5pt,linewidth=0.4pt,linecolor=lightgray]{c-c}(-4.3,0)(7.28,0)}
\multips(-4,0)(1.0,0){12}{\psline[linestyle=dashed,linecap=1,dash=1.5pt 1.5pt,linewidth=0.4pt,linecolor=lightgray]{c-c}(0,-3.02)(0,6.3)}
\psaxes[labelFontSize=\scriptstyle,xAxis=true,yAxis=true,Dx=1.,Dy=1.,ticksize=-2pt 0,subticks=2]{->}(0,0)(-4.3,-3.02)(7.28,6.3)
\psplot[linewidth=2.pt]{-4.3}{7.28}{(1+0.4*x)/1.}
%\psplot[linewidth=2.pt]{-4.3}{7.28}{(-2.--1.5*x)/1.}
\begin{scriptsize}
\rput[bl](5.08,5.98){$f$}
\end{scriptsize}
\end{pspicture*}













































\EXERCICE
\enonce{ %début énoncé 

Soit ABC un triangle rectangle en B. On sait que AB = 5 et BC = 5
} % fin énoncé 

\begin{description}
\item[1.] \enonce{ %début énoncé 
Calculer la longueur de AC \points{2.0} %en pourcentage
} % fin énoncé 

\correction{ %début énoncé 
 Le triangle ABC est rectangle en B,, donc avec le théorème de Pythagore on a : 
\begin{eqnarray*}
AB^2 + BC^2 &=& AC^2 \\
5^2 + 5^2 &=& AC^2 \\
50 &=&AC^2 \\
& & \mbox{ Donc : } AC = \sqrt{50} 
\end{eqnarray*}
} % fin correction 

\item[2.] \enonce{ %début énoncé 
Calculer l'angle $\widehat{ABC}$ au degré près. } % fin énoncé 
\points{2.0} %en pourcentage
\end{description}

\psset{xunit=1.0cm,yunit=1.0cm,algebraic=true,dimen=middle,dotstyle=o,dotsize=5pt 0,linewidth=2.pt,arrowsize=3pt 2,arrowinset=0.25}
\begin{pspicture*}(-4.3,-3.02)(7.28,6.3)
\multips(0,-3)(0,1.0){10}{\psline[linestyle=dashed,linecap=1,dash=1.5pt 1.5pt,linewidth=0.4pt,linecolor=lightgray]{c-c}(-4.3,0)(7.28,0)}
\multips(-4,0)(1.0,0){12}{\psline[linestyle=dashed,linecap=1,dash=1.5pt 1.5pt,linewidth=0.4pt,linecolor=lightgray]{c-c}(0,-3.02)(0,6.3)}
\psaxes[labelFontSize=\scriptstyle,xAxis=true,yAxis=true,Dx=1.,Dy=1.,ticksize=-2pt 0,subticks=2]{->}(0,0)(-4.3,-3.02)(7.28,6.3)
\psplot[linewidth=2.pt]{-4.3}{7.28}{(-1+-0.4*x)/1.}
%\psplot[linewidth=2.pt]{-4.3}{7.28}{(-2.--1.5*x)/1.}
\begin{scriptsize}
\rput[bl](5.08,5.98){$f$}
\end{scriptsize}
\end{pspicture*}













































\EXERCICE
\enonce{ %début énoncé 

Soit ABC un triangle rectangle en B. On sait que AB = 5 et BC = 5
} % fin énoncé 

\begin{description}
\item[1.] \enonce{ %début énoncé 
Calculer la longueur de AC \points{2.0} %en pourcentage
} % fin énoncé 

\correction{ %début énoncé 
 Le triangle ABC est rectangle en B,, donc avec le théorème de Pythagore on a : 
\begin{eqnarray*}
AB^2 + BC^2 &=& AC^2 \\
5^2 + 5^2 &=& AC^2 \\
50 &=&AC^2 \\
& & \mbox{ Donc : } AC = \sqrt{50} 
\end{eqnarray*}
} % fin correction 

\item[2.] \enonce{ %début énoncé 
Calculer l'angle $\widehat{ABC}$ au degré près. } % fin énoncé 
\points{2.0} %en pourcentage
\end{description}

\psset{xunit=1.0cm,yunit=1.0cm,algebraic=true,dimen=middle,dotstyle=o,dotsize=5pt 0,linewidth=2.pt,arrowsize=3pt 2,arrowinset=0.25}
\begin{pspicture*}(-4.3,-3.02)(7.28,6.3)
\multips(0,-3)(0,1.0){10}{\psline[linestyle=dashed,linecap=1,dash=1.5pt 1.5pt,linewidth=0.4pt,linecolor=lightgray]{c-c}(-4.3,0)(7.28,0)}
\multips(-4,0)(1.0,0){12}{\psline[linestyle=dashed,linecap=1,dash=1.5pt 1.5pt,linewidth=0.4pt,linecolor=lightgray]{c-c}(0,-3.02)(0,6.3)}
\psaxes[labelFontSize=\scriptstyle,xAxis=true,yAxis=true,Dx=1.,Dy=1.,ticksize=-2pt 0,subticks=2]{->}(0,0)(-4.3,-3.02)(7.28,6.3)
\psplot[linewidth=2.pt]{-4.3}{7.28}{(-1+-0.2*x)/1.}
%\psplot[linewidth=2.pt]{-4.3}{7.28}{(-2.--1.5*x)/1.}
\begin{scriptsize}
\rput[bl](5.08,5.98){$f$}
\end{scriptsize}
\end{pspicture*}













































\EXERCICE
\enonce{ %début énoncé 

Soit ABC un triangle rectangle en B. On sait que AB = 5 et BC = 5
} % fin énoncé 

\begin{description}
\item[1.] \enonce{ %début énoncé 
Calculer la longueur de AC \points{2.0} %en pourcentage
} % fin énoncé 

\correction{ %début énoncé 
 Le triangle ABC est rectangle en B,, donc avec le théorème de Pythagore on a : 
\begin{eqnarray*}
AB^2 + BC^2 &=& AC^2 \\
5^2 + 5^2 &=& AC^2 \\
50 &=&AC^2 \\
& & \mbox{ Donc : } AC = \sqrt{50} 
\end{eqnarray*}
} % fin correction 

\item[2.] \enonce{ %début énoncé 
Calculer l'angle $\widehat{ABC}$ au degré près. } % fin énoncé 
\points{2.0} %en pourcentage
\end{description}

\psset{xunit=1.0cm,yunit=1.0cm,algebraic=true,dimen=middle,dotstyle=o,dotsize=5pt 0,linewidth=2.pt,arrowsize=3pt 2,arrowinset=0.25}
\begin{pspicture*}(-4.3,-3.02)(7.28,6.3)
\multips(0,-3)(0,1.0){10}{\psline[linestyle=dashed,linecap=1,dash=1.5pt 1.5pt,linewidth=0.4pt,linecolor=lightgray]{c-c}(-4.3,0)(7.28,0)}
\multips(-4,0)(1.0,0){12}{\psline[linestyle=dashed,linecap=1,dash=1.5pt 1.5pt,linewidth=0.4pt,linecolor=lightgray]{c-c}(0,-3.02)(0,6.3)}
\psaxes[labelFontSize=\scriptstyle,xAxis=true,yAxis=true,Dx=1.,Dy=1.,ticksize=-2pt 0,subticks=2]{->}(0,0)(-4.3,-3.02)(7.28,6.3)
\psplot[linewidth=2.pt]{-4.3}{7.28}{(-1+0.0*x)/1.}
%\psplot[linewidth=2.pt]{-4.3}{7.28}{(-2.--1.5*x)/1.}
\begin{scriptsize}
\rput[bl](5.08,5.98){$f$}
\end{scriptsize}
\end{pspicture*}













































\EXERCICE
\enonce{ %début énoncé 

Soit ABC un triangle rectangle en B. On sait que AB = 5 et BC = 5
} % fin énoncé 

\begin{description}
\item[1.] \enonce{ %début énoncé 
Calculer la longueur de AC \points{2.0} %en pourcentage
} % fin énoncé 

\correction{ %début énoncé 
 Le triangle ABC est rectangle en B,, donc avec le théorème de Pythagore on a : 
\begin{eqnarray*}
AB^2 + BC^2 &=& AC^2 \\
5^2 + 5^2 &=& AC^2 \\
50 &=&AC^2 \\
& & \mbox{ Donc : } AC = \sqrt{50} 
\end{eqnarray*}
} % fin correction 

\item[2.] \enonce{ %début énoncé 
Calculer l'angle $\widehat{ABC}$ au degré près. } % fin énoncé 
\points{2.0} %en pourcentage
\end{description}

\psset{xunit=1.0cm,yunit=1.0cm,algebraic=true,dimen=middle,dotstyle=o,dotsize=5pt 0,linewidth=2.pt,arrowsize=3pt 2,arrowinset=0.25}
\begin{pspicture*}(-4.3,-3.02)(7.28,6.3)
\multips(0,-3)(0,1.0){10}{\psline[linestyle=dashed,linecap=1,dash=1.5pt 1.5pt,linewidth=0.4pt,linecolor=lightgray]{c-c}(-4.3,0)(7.28,0)}
\multips(-4,0)(1.0,0){12}{\psline[linestyle=dashed,linecap=1,dash=1.5pt 1.5pt,linewidth=0.4pt,linecolor=lightgray]{c-c}(0,-3.02)(0,6.3)}
\psaxes[labelFontSize=\scriptstyle,xAxis=true,yAxis=true,Dx=1.,Dy=1.,ticksize=-2pt 0,subticks=2]{->}(0,0)(-4.3,-3.02)(7.28,6.3)
\psplot[linewidth=2.pt]{-4.3}{7.28}{(-1+0.2*x)/1.}
%\psplot[linewidth=2.pt]{-4.3}{7.28}{(-2.--1.5*x)/1.}
\begin{scriptsize}
\rput[bl](5.08,5.98){$f$}
\end{scriptsize}
\end{pspicture*}













































\EXERCICE
\enonce{ %début énoncé 

Soit ABC un triangle rectangle en B. On sait que AB = 5 et BC = 5
} % fin énoncé 

\begin{description}
\item[1.] \enonce{ %début énoncé 
Calculer la longueur de AC \points{2.0} %en pourcentage
} % fin énoncé 

\correction{ %début énoncé 
 Le triangle ABC est rectangle en B,, donc avec le théorème de Pythagore on a : 
\begin{eqnarray*}
AB^2 + BC^2 &=& AC^2 \\
5^2 + 5^2 &=& AC^2 \\
50 &=&AC^2 \\
& & \mbox{ Donc : } AC = \sqrt{50} 
\end{eqnarray*}
} % fin correction 

\item[2.] \enonce{ %début énoncé 
Calculer l'angle $\widehat{ABC}$ au degré près. } % fin énoncé 
\points{2.0} %en pourcentage
\end{description}

\psset{xunit=1.0cm,yunit=1.0cm,algebraic=true,dimen=middle,dotstyle=o,dotsize=5pt 0,linewidth=2.pt,arrowsize=3pt 2,arrowinset=0.25}
\begin{pspicture*}(-4.3,-3.02)(7.28,6.3)
\multips(0,-3)(0,1.0){10}{\psline[linestyle=dashed,linecap=1,dash=1.5pt 1.5pt,linewidth=0.4pt,linecolor=lightgray]{c-c}(-4.3,0)(7.28,0)}
\multips(-4,0)(1.0,0){12}{\psline[linestyle=dashed,linecap=1,dash=1.5pt 1.5pt,linewidth=0.4pt,linecolor=lightgray]{c-c}(0,-3.02)(0,6.3)}
\psaxes[labelFontSize=\scriptstyle,xAxis=true,yAxis=true,Dx=1.,Dy=1.,ticksize=-2pt 0,subticks=2]{->}(0,0)(-4.3,-3.02)(7.28,6.3)
\psplot[linewidth=2.pt]{-4.3}{7.28}{(-1+0.4*x)/1.}
%\psplot[linewidth=2.pt]{-4.3}{7.28}{(-2.--1.5*x)/1.}
\begin{scriptsize}
\rput[bl](5.08,5.98){$f$}
\end{scriptsize}
\end{pspicture*}













































\EXERCICE
\enonce{ %début énoncé 

Soit ABC un triangle rectangle en B. On sait que AB = 5 et BC = 5
} % fin énoncé 

\begin{description}
\item[1.] \enonce{ %début énoncé 
Calculer la longueur de AC \points{2.0} %en pourcentage
} % fin énoncé 

\correction{ %début énoncé 
 Le triangle ABC est rectangle en B,, donc avec le théorème de Pythagore on a : 
\begin{eqnarray*}
AB^2 + BC^2 &=& AC^2 \\
5^2 + 5^2 &=& AC^2 \\
50 &=&AC^2 \\
& & \mbox{ Donc : } AC = \sqrt{50} 
\end{eqnarray*}
} % fin correction 

\item[2.] \enonce{ %début énoncé 
Calculer l'angle $\widehat{ABC}$ au degré près. } % fin énoncé 
\points{2.0} %en pourcentage
\end{description}

\psset{xunit=1.0cm,yunit=1.0cm,algebraic=true,dimen=middle,dotstyle=o,dotsize=5pt 0,linewidth=2.pt,arrowsize=3pt 2,arrowinset=0.25}
\begin{pspicture*}(-4.3,-3.02)(7.28,6.3)
\multips(0,-3)(0,1.0){10}{\psline[linestyle=dashed,linecap=1,dash=1.5pt 1.5pt,linewidth=0.4pt,linecolor=lightgray]{c-c}(-4.3,0)(7.28,0)}
\multips(-4,0)(1.0,0){12}{\psline[linestyle=dashed,linecap=1,dash=1.5pt 1.5pt,linewidth=0.4pt,linecolor=lightgray]{c-c}(0,-3.02)(0,6.3)}
\psaxes[labelFontSize=\scriptstyle,xAxis=true,yAxis=true,Dx=1.,Dy=1.,ticksize=-2pt 0,subticks=2]{->}(0,0)(-4.3,-3.02)(7.28,6.3)
\psplot[linewidth=2.pt]{-4.3}{7.28}{(0+-0.4*x)/1.}
%\psplot[linewidth=2.pt]{-4.3}{7.28}{(-2.--1.5*x)/1.}
\begin{scriptsize}
\rput[bl](5.08,5.98){$f$}
\end{scriptsize}
\end{pspicture*}













































\EXERCICE
\enonce{ %début énoncé 

Soit ABC un triangle rectangle en B. On sait que AB = 5 et BC = 5
} % fin énoncé 

\begin{description}
\item[1.] \enonce{ %début énoncé 
Calculer la longueur de AC \points{2.0} %en pourcentage
} % fin énoncé 

\correction{ %début énoncé 
 Le triangle ABC est rectangle en B,, donc avec le théorème de Pythagore on a : 
\begin{eqnarray*}
AB^2 + BC^2 &=& AC^2 \\
5^2 + 5^2 &=& AC^2 \\
50 &=&AC^2 \\
& & \mbox{ Donc : } AC = \sqrt{50} 
\end{eqnarray*}
} % fin correction 

\item[2.] \enonce{ %début énoncé 
Calculer l'angle $\widehat{ABC}$ au degré près. } % fin énoncé 
\points{2.0} %en pourcentage
\end{description}

\psset{xunit=1.0cm,yunit=1.0cm,algebraic=true,dimen=middle,dotstyle=o,dotsize=5pt 0,linewidth=2.pt,arrowsize=3pt 2,arrowinset=0.25}
\begin{pspicture*}(-4.3,-3.02)(7.28,6.3)
\multips(0,-3)(0,1.0){10}{\psline[linestyle=dashed,linecap=1,dash=1.5pt 1.5pt,linewidth=0.4pt,linecolor=lightgray]{c-c}(-4.3,0)(7.28,0)}
\multips(-4,0)(1.0,0){12}{\psline[linestyle=dashed,linecap=1,dash=1.5pt 1.5pt,linewidth=0.4pt,linecolor=lightgray]{c-c}(0,-3.02)(0,6.3)}
\psaxes[labelFontSize=\scriptstyle,xAxis=true,yAxis=true,Dx=1.,Dy=1.,ticksize=-2pt 0,subticks=2]{->}(0,0)(-4.3,-3.02)(7.28,6.3)
\psplot[linewidth=2.pt]{-4.3}{7.28}{(0+-0.2*x)/1.}
%\psplot[linewidth=2.pt]{-4.3}{7.28}{(-2.--1.5*x)/1.}
\begin{scriptsize}
\rput[bl](5.08,5.98){$f$}
\end{scriptsize}
\end{pspicture*}













































\EXERCICE
\enonce{ %début énoncé 

Soit ABC un triangle rectangle en B. On sait que AB = 5 et BC = 5
} % fin énoncé 

\begin{description}
\item[1.] \enonce{ %début énoncé 
Calculer la longueur de AC \points{2.0} %en pourcentage
} % fin énoncé 

\correction{ %début énoncé 
 Le triangle ABC est rectangle en B,, donc avec le théorème de Pythagore on a : 
\begin{eqnarray*}
AB^2 + BC^2 &=& AC^2 \\
5^2 + 5^2 &=& AC^2 \\
50 &=&AC^2 \\
& & \mbox{ Donc : } AC = \sqrt{50} 
\end{eqnarray*}
} % fin correction 

\item[2.] \enonce{ %début énoncé 
Calculer l'angle $\widehat{ABC}$ au degré près. } % fin énoncé 
\points{2.0} %en pourcentage
\end{description}

\psset{xunit=1.0cm,yunit=1.0cm,algebraic=true,dimen=middle,dotstyle=o,dotsize=5pt 0,linewidth=2.pt,arrowsize=3pt 2,arrowinset=0.25}
\begin{pspicture*}(-4.3,-3.02)(7.28,6.3)
\multips(0,-3)(0,1.0){10}{\psline[linestyle=dashed,linecap=1,dash=1.5pt 1.5pt,linewidth=0.4pt,linecolor=lightgray]{c-c}(-4.3,0)(7.28,0)}
\multips(-4,0)(1.0,0){12}{\psline[linestyle=dashed,linecap=1,dash=1.5pt 1.5pt,linewidth=0.4pt,linecolor=lightgray]{c-c}(0,-3.02)(0,6.3)}
\psaxes[labelFontSize=\scriptstyle,xAxis=true,yAxis=true,Dx=1.,Dy=1.,ticksize=-2pt 0,subticks=2]{->}(0,0)(-4.3,-3.02)(7.28,6.3)
\psplot[linewidth=2.pt]{-4.3}{7.28}{(0+0.0*x)/1.}
%\psplot[linewidth=2.pt]{-4.3}{7.28}{(-2.--1.5*x)/1.}
\begin{scriptsize}
\rput[bl](5.08,5.98){$f$}
\end{scriptsize}
\end{pspicture*}













































\EXERCICE
\enonce{ %début énoncé 

Soit ABC un triangle rectangle en B. On sait que AB = 5 et BC = 5
} % fin énoncé 

\begin{description}
\item[1.] \enonce{ %début énoncé 
Calculer la longueur de AC \points{2.0} %en pourcentage
} % fin énoncé 

\correction{ %début énoncé 
 Le triangle ABC est rectangle en B,, donc avec le théorème de Pythagore on a : 
\begin{eqnarray*}
AB^2 + BC^2 &=& AC^2 \\
5^2 + 5^2 &=& AC^2 \\
50 &=&AC^2 \\
& & \mbox{ Donc : } AC = \sqrt{50} 
\end{eqnarray*}
} % fin correction 

\item[2.] \enonce{ %début énoncé 
Calculer l'angle $\widehat{ABC}$ au degré près. } % fin énoncé 
\points{2.0} %en pourcentage
\end{description}

\psset{xunit=1.0cm,yunit=1.0cm,algebraic=true,dimen=middle,dotstyle=o,dotsize=5pt 0,linewidth=2.pt,arrowsize=3pt 2,arrowinset=0.25}
\begin{pspicture*}(-4.3,-3.02)(7.28,6.3)
\multips(0,-3)(0,1.0){10}{\psline[linestyle=dashed,linecap=1,dash=1.5pt 1.5pt,linewidth=0.4pt,linecolor=lightgray]{c-c}(-4.3,0)(7.28,0)}
\multips(-4,0)(1.0,0){12}{\psline[linestyle=dashed,linecap=1,dash=1.5pt 1.5pt,linewidth=0.4pt,linecolor=lightgray]{c-c}(0,-3.02)(0,6.3)}
\psaxes[labelFontSize=\scriptstyle,xAxis=true,yAxis=true,Dx=1.,Dy=1.,ticksize=-2pt 0,subticks=2]{->}(0,0)(-4.3,-3.02)(7.28,6.3)
\psplot[linewidth=2.pt]{-4.3}{7.28}{(0+0.2*x)/1.}
%\psplot[linewidth=2.pt]{-4.3}{7.28}{(-2.--1.5*x)/1.}
\begin{scriptsize}
\rput[bl](5.08,5.98){$f$}
\end{scriptsize}
\end{pspicture*}













































\EXERCICE
\enonce{ %début énoncé 

Soit ABC un triangle rectangle en B. On sait que AB = 5 et BC = 5
} % fin énoncé 

\begin{description}
\item[1.] \enonce{ %début énoncé 
Calculer la longueur de AC \points{2.0} %en pourcentage
} % fin énoncé 

\correction{ %début énoncé 
 Le triangle ABC est rectangle en B,, donc avec le théorème de Pythagore on a : 
\begin{eqnarray*}
AB^2 + BC^2 &=& AC^2 \\
5^2 + 5^2 &=& AC^2 \\
50 &=&AC^2 \\
& & \mbox{ Donc : } AC = \sqrt{50} 
\end{eqnarray*}
} % fin correction 

\item[2.] \enonce{ %début énoncé 
Calculer l'angle $\widehat{ABC}$ au degré près. } % fin énoncé 
\points{2.0} %en pourcentage
\end{description}

\psset{xunit=1.0cm,yunit=1.0cm,algebraic=true,dimen=middle,dotstyle=o,dotsize=5pt 0,linewidth=2.pt,arrowsize=3pt 2,arrowinset=0.25}
\begin{pspicture*}(-4.3,-3.02)(7.28,6.3)
\multips(0,-3)(0,1.0){10}{\psline[linestyle=dashed,linecap=1,dash=1.5pt 1.5pt,linewidth=0.4pt,linecolor=lightgray]{c-c}(-4.3,0)(7.28,0)}
\multips(-4,0)(1.0,0){12}{\psline[linestyle=dashed,linecap=1,dash=1.5pt 1.5pt,linewidth=0.4pt,linecolor=lightgray]{c-c}(0,-3.02)(0,6.3)}
\psaxes[labelFontSize=\scriptstyle,xAxis=true,yAxis=true,Dx=1.,Dy=1.,ticksize=-2pt 0,subticks=2]{->}(0,0)(-4.3,-3.02)(7.28,6.3)
\psplot[linewidth=2.pt]{-4.3}{7.28}{(0+0.4*x)/1.}
%\psplot[linewidth=2.pt]{-4.3}{7.28}{(-2.--1.5*x)/1.}
\begin{scriptsize}
\rput[bl](5.08,5.98){$f$}
\end{scriptsize}
\end{pspicture*}













































\EXERCICE
\enonce{ %début énoncé 

Soit ABC un triangle rectangle en B. On sait que AB = 5 et BC = 5
} % fin énoncé 

\begin{description}
\item[1.] \enonce{ %début énoncé 
Calculer la longueur de AC \points{2.0} %en pourcentage
} % fin énoncé 

\correction{ %début énoncé 
 Le triangle ABC est rectangle en B,, donc avec le théorème de Pythagore on a : 
\begin{eqnarray*}
AB^2 + BC^2 &=& AC^2 \\
5^2 + 5^2 &=& AC^2 \\
50 &=&AC^2 \\
& & \mbox{ Donc : } AC = \sqrt{50} 
\end{eqnarray*}
} % fin correction 

\item[2.] \enonce{ %début énoncé 
Calculer l'angle $\widehat{ABC}$ au degré près. } % fin énoncé 
\points{2.0} %en pourcentage
\end{description}

\psset{xunit=1.0cm,yunit=1.0cm,algebraic=true,dimen=middle,dotstyle=o,dotsize=5pt 0,linewidth=2.pt,arrowsize=3pt 2,arrowinset=0.25}
\begin{pspicture*}(-4.3,-3.02)(7.28,6.3)
\multips(0,-3)(0,1.0){10}{\psline[linestyle=dashed,linecap=1,dash=1.5pt 1.5pt,linewidth=0.4pt,linecolor=lightgray]{c-c}(-4.3,0)(7.28,0)}
\multips(-4,0)(1.0,0){12}{\psline[linestyle=dashed,linecap=1,dash=1.5pt 1.5pt,linewidth=0.4pt,linecolor=lightgray]{c-c}(0,-3.02)(0,6.3)}
\psaxes[labelFontSize=\scriptstyle,xAxis=true,yAxis=true,Dx=1.,Dy=1.,ticksize=-2pt 0,subticks=2]{->}(0,0)(-4.3,-3.02)(7.28,6.3)
\psplot[linewidth=2.pt]{-4.3}{7.28}{(1+-0.4*x)/1.}
%\psplot[linewidth=2.pt]{-4.3}{7.28}{(-2.--1.5*x)/1.}
\begin{scriptsize}
\rput[bl](5.08,5.98){$f$}
\end{scriptsize}
\end{pspicture*}













































\EXERCICE
\enonce{ %début énoncé 

Soit ABC un triangle rectangle en B. On sait que AB = 5 et BC = 5
} % fin énoncé 

\begin{description}
\item[1.] \enonce{ %début énoncé 
Calculer la longueur de AC \points{2.0} %en pourcentage
} % fin énoncé 

\correction{ %début énoncé 
 Le triangle ABC est rectangle en B,, donc avec le théorème de Pythagore on a : 
\begin{eqnarray*}
AB^2 + BC^2 &=& AC^2 \\
5^2 + 5^2 &=& AC^2 \\
50 &=&AC^2 \\
& & \mbox{ Donc : } AC = \sqrt{50} 
\end{eqnarray*}
} % fin correction 

\item[2.] \enonce{ %début énoncé 
Calculer l'angle $\widehat{ABC}$ au degré près. } % fin énoncé 
\points{2.0} %en pourcentage
\end{description}

\psset{xunit=1.0cm,yunit=1.0cm,algebraic=true,dimen=middle,dotstyle=o,dotsize=5pt 0,linewidth=2.pt,arrowsize=3pt 2,arrowinset=0.25}
\begin{pspicture*}(-4.3,-3.02)(7.28,6.3)
\multips(0,-3)(0,1.0){10}{\psline[linestyle=dashed,linecap=1,dash=1.5pt 1.5pt,linewidth=0.4pt,linecolor=lightgray]{c-c}(-4.3,0)(7.28,0)}
\multips(-4,0)(1.0,0){12}{\psline[linestyle=dashed,linecap=1,dash=1.5pt 1.5pt,linewidth=0.4pt,linecolor=lightgray]{c-c}(0,-3.02)(0,6.3)}
\psaxes[labelFontSize=\scriptstyle,xAxis=true,yAxis=true,Dx=1.,Dy=1.,ticksize=-2pt 0,subticks=2]{->}(0,0)(-4.3,-3.02)(7.28,6.3)
\psplot[linewidth=2.pt]{-4.3}{7.28}{(1+-0.2*x)/1.}
%\psplot[linewidth=2.pt]{-4.3}{7.28}{(-2.--1.5*x)/1.}
\begin{scriptsize}
\rput[bl](5.08,5.98){$f$}
\end{scriptsize}
\end{pspicture*}













































\EXERCICE
\enonce{ %début énoncé 

Soit ABC un triangle rectangle en B. On sait que AB = 5 et BC = 5
} % fin énoncé 

\begin{description}
\item[1.] \enonce{ %début énoncé 
Calculer la longueur de AC \points{2.0} %en pourcentage
} % fin énoncé 

\correction{ %début énoncé 
 Le triangle ABC est rectangle en B,, donc avec le théorème de Pythagore on a : 
\begin{eqnarray*}
AB^2 + BC^2 &=& AC^2 \\
5^2 + 5^2 &=& AC^2 \\
50 &=&AC^2 \\
& & \mbox{ Donc : } AC = \sqrt{50} 
\end{eqnarray*}
} % fin correction 

\item[2.] \enonce{ %début énoncé 
Calculer l'angle $\widehat{ABC}$ au degré près. } % fin énoncé 
\points{2.0} %en pourcentage
\end{description}

\psset{xunit=1.0cm,yunit=1.0cm,algebraic=true,dimen=middle,dotstyle=o,dotsize=5pt 0,linewidth=2.pt,arrowsize=3pt 2,arrowinset=0.25}
\begin{pspicture*}(-4.3,-3.02)(7.28,6.3)
\multips(0,-3)(0,1.0){10}{\psline[linestyle=dashed,linecap=1,dash=1.5pt 1.5pt,linewidth=0.4pt,linecolor=lightgray]{c-c}(-4.3,0)(7.28,0)}
\multips(-4,0)(1.0,0){12}{\psline[linestyle=dashed,linecap=1,dash=1.5pt 1.5pt,linewidth=0.4pt,linecolor=lightgray]{c-c}(0,-3.02)(0,6.3)}
\psaxes[labelFontSize=\scriptstyle,xAxis=true,yAxis=true,Dx=1.,Dy=1.,ticksize=-2pt 0,subticks=2]{->}(0,0)(-4.3,-3.02)(7.28,6.3)
\psplot[linewidth=2.pt]{-4.3}{7.28}{(1+0.0*x)/1.}
%\psplot[linewidth=2.pt]{-4.3}{7.28}{(-2.--1.5*x)/1.}
\begin{scriptsize}
\rput[bl](5.08,5.98){$f$}
\end{scriptsize}
\end{pspicture*}













































\EXERCICE
\enonce{ %début énoncé 

Soit ABC un triangle rectangle en B. On sait que AB = 5 et BC = 5
} % fin énoncé 

\begin{description}
\item[1.] \enonce{ %début énoncé 
Calculer la longueur de AC \points{2.0} %en pourcentage
} % fin énoncé 

\correction{ %début énoncé 
 Le triangle ABC est rectangle en B,, donc avec le théorème de Pythagore on a : 
\begin{eqnarray*}
AB^2 + BC^2 &=& AC^2 \\
5^2 + 5^2 &=& AC^2 \\
50 &=&AC^2 \\
& & \mbox{ Donc : } AC = \sqrt{50} 
\end{eqnarray*}
} % fin correction 

\item[2.] \enonce{ %début énoncé 
Calculer l'angle $\widehat{ABC}$ au degré près. } % fin énoncé 
\points{2.0} %en pourcentage
\end{description}

\psset{xunit=1.0cm,yunit=1.0cm,algebraic=true,dimen=middle,dotstyle=o,dotsize=5pt 0,linewidth=2.pt,arrowsize=3pt 2,arrowinset=0.25}
\begin{pspicture*}(-4.3,-3.02)(7.28,6.3)
\multips(0,-3)(0,1.0){10}{\psline[linestyle=dashed,linecap=1,dash=1.5pt 1.5pt,linewidth=0.4pt,linecolor=lightgray]{c-c}(-4.3,0)(7.28,0)}
\multips(-4,0)(1.0,0){12}{\psline[linestyle=dashed,linecap=1,dash=1.5pt 1.5pt,linewidth=0.4pt,linecolor=lightgray]{c-c}(0,-3.02)(0,6.3)}
\psaxes[labelFontSize=\scriptstyle,xAxis=true,yAxis=true,Dx=1.,Dy=1.,ticksize=-2pt 0,subticks=2]{->}(0,0)(-4.3,-3.02)(7.28,6.3)
\psplot[linewidth=2.pt]{-4.3}{7.28}{(1+0.2*x)/1.}
%\psplot[linewidth=2.pt]{-4.3}{7.28}{(-2.--1.5*x)/1.}
\begin{scriptsize}
\rput[bl](5.08,5.98){$f$}
\end{scriptsize}
\end{pspicture*}













































\EXERCICE
\enonce{ %début énoncé 

Soit ABC un triangle rectangle en B. On sait que AB = 5 et BC = 5
} % fin énoncé 

\begin{description}
\item[1.] \enonce{ %début énoncé 
Calculer la longueur de AC \points{2.0} %en pourcentage
} % fin énoncé 

\correction{ %début énoncé 
 Le triangle ABC est rectangle en B,, donc avec le théorème de Pythagore on a : 
\begin{eqnarray*}
AB^2 + BC^2 &=& AC^2 \\
5^2 + 5^2 &=& AC^2 \\
50 &=&AC^2 \\
& & \mbox{ Donc : } AC = \sqrt{50} 
\end{eqnarray*}
} % fin correction 

\item[2.] \enonce{ %début énoncé 
Calculer l'angle $\widehat{ABC}$ au degré près. } % fin énoncé 
\points{2.0} %en pourcentage
\end{description}

\psset{xunit=1.0cm,yunit=1.0cm,algebraic=true,dimen=middle,dotstyle=o,dotsize=5pt 0,linewidth=2.pt,arrowsize=3pt 2,arrowinset=0.25}
\begin{pspicture*}(-4.3,-3.02)(7.28,6.3)
\multips(0,-3)(0,1.0){10}{\psline[linestyle=dashed,linecap=1,dash=1.5pt 1.5pt,linewidth=0.4pt,linecolor=lightgray]{c-c}(-4.3,0)(7.28,0)}
\multips(-4,0)(1.0,0){12}{\psline[linestyle=dashed,linecap=1,dash=1.5pt 1.5pt,linewidth=0.4pt,linecolor=lightgray]{c-c}(0,-3.02)(0,6.3)}
\psaxes[labelFontSize=\scriptstyle,xAxis=true,yAxis=true,Dx=1.,Dy=1.,ticksize=-2pt 0,subticks=2]{->}(0,0)(-4.3,-3.02)(7.28,6.3)
\psplot[linewidth=2.pt]{-4.3}{7.28}{(1+0.4*x)/1.}
%\psplot[linewidth=2.pt]{-4.3}{7.28}{(-2.--1.5*x)/1.}
\begin{scriptsize}
\rput[bl](5.08,5.98){$f$}
\end{scriptsize}
\end{pspicture*}











































\end{document}
