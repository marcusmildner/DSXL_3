
\section{Exercice\_001}

\index[notions]{Pythagore}
\index[notions]{triangle rectangle}
\index[notions]{triangle rectangle}
\index[notions]{droites perpendiculaires}
\index[notions]{aire triangle}
\index[notions]{hauteur triangle}
\begin{description}
\item[Date de création : ]10/01/2024
\item[Source : ]marcus
\item[Liste des notions : ]Pythagore : triangle rectangle : triangle rectangle : droites perpendiculaires : aire triangle : hauteur triangle : 
\item[Nombre de variantes de l'exercice : ]22
\end{description}

\centerline{\bf\large Barème utilisé pour l'exercice : 10 points / paramètre utilisé : numéro 0}



\EXERCICE \SeulementModeEnonce{\points{10.0}}
\enonce{ %début énoncé 

Soit ABC un triangle rectangle en B. On sait que AB = $3$ et BC = $4$

\definecolor{uuuuuu}{rgb}{0.26666666666666666,0.26666666666666666,0.26666666666666666}
\begin{tikzpicture}[line cap=round,line join=round,>=triangle 45,x=1.0cm,y=1.0cm]
\clip(-4.3,-3.02) rectangle (7.28,6.3);
\draw [line width=2.pt] (4.,3.)-- (-2.,0.);
\draw [line width=2.pt] (-2.,0.)-- (4.,0.);
\draw [line width=2.pt] (4.,0.)-- (4.,3.);
\draw [line width=2.pt,domain=-4.3:7.28] plot(\x,{(--24.-6.*\x)/3.});
\begin{scriptsize}
\draw [fill=black] (-2.,0.) circle (2.5pt);
\draw[color=black] (-1.86,0.37) node {$A$};
\draw [fill=black] (4.,0.) circle (2.5pt);
\draw[color=black] (4.14,0.37) node {$B$};
\draw [fill=black] (4.,3.) circle (2.5pt);
\draw[color=black] (4.14,3.37) node {$C$};
\draw[color=black] (1.3,6.15) node {(d)};
\draw [fill=uuuuuu] (2.8,2.4) circle (2.0pt);
\draw[color=uuuuuu] (2.22,2.67) node {$H$};
\end{scriptsize}
\end{tikzpicture}


} % fin énoncé 

\begin{description}
\item[1.] \enonce{ %début énoncé 
Calculer la longueur de AC 
On donnera une valeur approchée au millième. 
} % fin énoncé 

\correction{ %début énoncé 
\points{4.0} %en pourcentage 
 Le triangle ABC est rectangle en B, donc avec le théorème de Pythagore on a : 
\begin{eqnarray*}
AB^2 + BC^2 &=& AC^2 \\
3^2 + 4^2 &=& AC^2 \\
25 &=&AC^2 \\
& & \mbox{ Donc : } AC = \sqrt{25}\approx 5.000 
\end{eqnarray*}
} % fin correction 

\item[2.] \enonce{ %début énoncé 
 Calculer l'aire $S$ du triangle ABC
} % fin énoncé 

\correction{ %début énoncé 
\points{2.0} %en pourcentage
Comme le triangle ABC est rectangle en B on a :
\begin{eqnarray*} 
S &=& \frac{AB \times BC}{2} \\
S &=& \frac{3\times 4}{2} \\
S &=& 6 \mbox{ unité d'aire (u.a.)}
\end{eqnarray*}
} % fin correction 

\item[3.] \enonce{ %début énoncé 
 Soit H le pied de la hauteur du triangle ABC issu de B. Exprimer l'aire S en fonction de BH et AC. 
} % fin énoncé 


\correction{ %début énoncé 
\points{2.0} %en pourcentage
 On a : $ S = \frac{BH \times AC}{2} $ 
} % fin correction 

\item[4.] \enonce{ %début énoncé 
 Determiner HB (on donnera une valeur arrondie au centième).
} % fin énoncé 

\correction{ %début énoncé 
 \points{2.0} %en pourcentage
 On a : 
\begin{eqnarray*} 
S &=& \frac{BH \times AC}{2} \\
6  &=& \frac{BH \times 5.000}{2} \\
2\times 6 &=& BH \times 5.000 \\
\frac{2\times 6}{5.000} & =& BH \\
&& \mbox{Donc : }BH \approx 2.40
\end{eqnarray*}
} % fin correction 

\end{description}


\SeulementModeEnonce{}




\section{Exercice\_002}

\index[notions]{fonction logarithme}
\index[notions]{fonction exponentielle}
\index[notions]{limite de suite}
\index[notions]{programme de calcul}
\index[notions]{récurrence}
\index[notions]{script python}
\index[notions]{suite bornée}
\index[notions]{suite monotone}
\index[notions]{terminale spe}
\begin{description}
\item[Date de création : ]21/02/2024
\item[Source : ]annales apmep, Baccalauréat Métropole20 mars 2023, sujet 1, exercice 3
\item[Liste des notions : ]fonction logarithme : fonction exponentielle : limite de suite : programme de calcul : récurrence : script python : suite bornée : suite monotone : terminale spe : 
\item[Nombre de variantes de l'exercice : ]166
\end{description}

\centerline{\bf\large Barème utilisé pour l'exercice : 10 points / paramètre utilisé : numéro 0}



\EXERCICE

\medskip

Une entreprise a créé une Foire Aux Questions ("FAQ") sur son site internet.

\medskip

On étudie le nombre de questions qui y sont posées chaque mois.

\bigskip

\textbf{Partie A : Première modélisation}

\medskip

Dans cette partie, on admet que, chaque mois :

\begin{itemize}
\item[$\bullet~~$]$90$\,\% des questions déjà posées le mois précédent sont conservées sur la FAQ ;
\item[$\bullet~~$]$130$ nouvelles questions sont ajoutées à la FAQ.
\end{itemize}

Au cours du premier mois, $300$ questions ont été posées.

\medskip

Pour estimer le nombre de questions, en centaines, présentes sur la FAQ le $n$-ième
mois, on modélise la situation ci-dessus à l'aide de la suite $\left(u_n\right)$ définie par : 
\begin{center}$u_1 =3$ \quad et, pour tout entier naturel $n \geqslant 1,\:u_{n+1} = 0.9u_n + 1.3$.
\end{center}

\smallskip

\begin{enumerate}
\item 
\enonce{ %début énoncé 

Calculer $u_2$ et $u_3$ et proposer une interprétation dans le contexte de l'exercice. 
} % fin énoncé 

\correction{ %début correction 

$u_2 = 0.9\times u_1 + 1.3 = 0.9\times 3 +  1.3= 4$ et 
$u_3 =0.9\times u_2 + 1.3= 0.9\times  4 +  1.3= 4.9 $
% et proposer une interprétation dans le contexte de l'exercice.2. 
On peut estimer à $400$ le nombre de questions le 2ème mois, et à $490$ le 3ème mois.
} % fin correction 


\item 
\enonce{ %début énoncé 

Montrer par récurrence que pour tout entier naturel $n \geqslant 1$ :
\[u_n = 13 - \dfrac{100}{9} \times 0.9^n.\]
} % fin énoncé 

\correction{ %début correction 

On va montrer par récurrence la propriété  $u_n = 13 - \dfrac{100}{9} \times 0.9^n$ pour tout $n\geqslant1$.

\begin{list}{\textbullet}{}
\item \textbf{Initialisation}

Pour $n=1$, on a $u_1= 3$ et $13 -\dfrac{100}{9}\times 0.9^1=3$.

La propriété est vérifiée au rang 1.
\item \textbf{Hérédité}

On suppose la propriété vraie au rang $n$ avec $n\geqslant 1$; autrement dit 
$$u_n = 13 - \dfrac{100}{9} \times 0.9^n\mbox{ ;}$$
c'est l'hypothèse de récurrence.

\begin{eqnarray*}
u_{n+1} &=& 0.9u_n+1.3 = 0.9 \left (13 - \dfrac{100}{9} \times 0.9^n\right ) +1.3 \\
&=& 0.9\times 13 - \dfrac{100}{9}\times 0.9^{n+1}+1.3 \\
&=&  13- \dfrac{100}{9}\times 0.9^{n+1}
\end{eqnarray*}


Donc la propriété est vraie au rang $n+1$.
\item \textbf{Conclusion}

La propriété est vraie au rang 1 et elle est héréditaire pour tout $n\geqslant 1$; d'après le principe de récurrence, elle est vraie pour tout $n\geqslant 1$.
\end{list}

Donc pour tout $n\geqslant 1$, on a: $u_n = 13 - \dfrac{100}{9} \times 0.9^n$.
} % fin correction 

\end{enumerate}

\begin{minipage}{0.62\linewidth}
\begin{enumerate}
\item  
\enonce{ %début énoncé 

En déduire que la suite $\left(u_n\right)$ est croissante.
} % fin énoncé 

\correction{ %début correction 

Pour tout $n\geqslant 1$, on a:

\begin{eqnarray*}
u_{n+1}-u_n &=& \left (13-\dfrac{100}{9}\times 0.9^{n+1}\right ) - \left (13-\dfrac{100}{9}\times 0.9^{n}\right ) \\
&= &13-\dfrac{100}{9}\times 0.9^{n+1} - 13+\dfrac{100}{9}\times 0.9^{n}\\
\phantom{u_{n+1}-u_n} &=& \dfrac{100}{9}\times 0.9^{n}\left (1-0.9\right ) >0
\end{eqnarray*}

donc la suite $(u_n)_{n\in\N*}$ est croissante.

} % fin correction 

\item 
\enonce{ %début énoncé 

On considère le programme ci-contre, écrit en langage Python.

Déterminer la valeur renvoyée par la saisie de seuil($8.5$) et l'interpréter dans le contexte de l'exercice.
} % fin énoncé 

\correction{ %début correction 

On considère le programme ci-contre, écrit en langage Python.

\begin{minipage}{0.72\linewidth}

%Déterminer la valeur renvoyée par la saisie de seuil(8.5) et l'interpréter dans le contexte de l'exercice.

Ce programme renvoie la plus petite valeur $n$ telle que $u_n>\texttt{p}$; donc \texttt{seuil($8.5$)} renvoie la plus petite valeur $n$ telle que $u_n>8.5$. On résout cette inéquation.

\begin{eqnarray*}
u_n&>&8.5 \\
& \iff &13 -\dfrac{100}{9}\times 0.9^n > 8.5\\
& \iff & 4.5 > \dfrac{100}{9}\times 0.9^n\\
& \iff &\phantom{u_n>8.5}  \\
& \iff & \dfrac{4.5\times 9}{100}> 0.9^n \\
& \iff &\ln(0.405) > \ln \left (0.9^n\right )\\
& \iff &\phantom{u_n>8.5} \\
& \iff &\ln(0.405) >n \times \ln  (0.9) \\
& \iff &\dfrac{\ln(0.405)}{\ln(0.9)} <n
\end{eqnarray*}

\end{minipage}\hfill


$\dfrac{\ln(0.405)}{\ln(0.9)} \approx 8.6$ donc la valeur renvoyée par \texttt{seuil($8.5$)} est $9$.
} % fin correction 

\end{enumerate}
\end{minipage}\hfill
\begin{minipage}{0.32\linewidth}
\renewcommand\arraystretch{0.9}

\begin{tabular}{|l|} \hline
def seuil(p) :\\
\qquad n=1\\
\qquad u=$3$\\
\qquad while u<=p :\\
\qquad  \qquad n=n+1\\
\qquad \qquad u=$0.9$*u+$1.3$ \\
\qquad return n\\ \hline
\end{tabular}
\end{minipage}

\bigskip

\textbf{Partie B : Une autre modélisation}

\medskip

Dans cette partie, on considère une seconde modélisation à l'aide d'une nouvelle
suite $\left(v_n\right)$ définie pour tout entier naturel $n \geqslant 1$ par:
\[v_n = 9 - 6 \times \text{e}^{-0.19\times(n - 1)}.\]

Le terme $v_n$ est une estimation du nombre de questions, en centaines, présentes le $n$-ième mois sur la FAQ.

\medskip

\begin{enumerate}
\item 
\enonce{ %début énoncé 

Préciser les valeurs arrondies au centième de $v_1$ et $v_2$.
} % fin énoncé 

\correction{ %début correction 

$v_1=9-6 \e^{0}=3$ et $v_2=9-6\e^{-0.19}\approx 4.04$
} % fin correction 

\item 
\enonce{ %début énoncé 

Déterminer, en justifiant la réponse, la plus petite valeur de $n$ telle que $v_n > 8.5$.
} % fin énoncé 

\correction{ %début correction 

On résout l'inéquation $v_n > 8.5$.

\begin{eqnarray*}
v_n &>& 8.5 \\
\iff && 9 - 6 \times \e^{-0.19\times(n - 1)} > 8.5 \\
\iff &&0.5 >  6 \times \e^{-0.19\times(n - 1)} \\
\iff&&\dfrac{0.5}{6} > \e^{-0.19\times(n - 1)} \\
&&\mbox{ et par croissance de la fonction logarithme népérien } \\
\iff&& \ln \left (\dfrac{0.5}{6}\right ) > -0.19\times (n-1) \\
\iff&&-\dfrac{\ln \left (\frac{0.5}{6}\right )}{0.19} < n - 1 \\
n &>& 1 -\dfrac{\ln \left (\frac{0.5}{6}\right )}{0.19}
\end{eqnarray*}



Or $-\dfrac{\ln \left (\frac{0.5}{6}\right )}{0.19}\approx -13.08$ et $1 -( -13.08) = 14.08$ donc la plus petite valeur telle que $v_n > 8.5$ est $n = 15$.
} % fin correction 

\end{enumerate}

\bigskip

\textbf{Partie C : Comparaison des deux modèles}

\medskip

\begin{enumerate}
\item 
\enonce{ %début énoncé 

L'entreprise considère qu'elle doit modifier la présentation de son site lorsque plus de $850$ questions sont présentes sur la FAQ. 

Parmi ces deux modélisations, laquelle conduit à procéder le plus tôt à cette modification ?

Justifier votre réponse.
} % fin énoncé 

\correction{ %début correction 

\begin{list}{\textbullet}{}
\item Selon le 1er modèle, il y a plus de $850$ questions sur la FAQ quand $u_n>8.5$, c'est-à-dire le $9$ème mois.
\item Selon le 2ème modèle, il y a plus de $850$ questions sur la FAQ quand $v_n>8.5$, c'est-à-dire le $ 15$ème mois.
\end{list}

C'est la $1ere$ modélisation qui conduit à procéder le plus tôt à cette modification.

} % fin correction 

\item 
\enonce{ %début énoncé 

En justifiant la réponse, pour quelle modélisation y a-t-il le plus grand nombre de questions sur la FAQ à long terme?
} % fin énoncé 

\correction{ %début correction 


\begin{list}{\textbullet}{}
\item $u_n=13 - \dfrac{100}{9}\times 0.9^n$

$-1<0.9 <1$ donc $\lim_{n\to +\infty} 0.9^n=0$ donc $\lim_{n\to +\infty} u_n=13$.

À long terme il y aura $1300$ questions pour la 1er modélisation..
\item $v_n = 9 - 6 \times \e^{-0.19\times(n - 1)}$

$\lim_{n\to +\infty} -0.19\times (n-1) = -\infty$ et $\lim_{X\to -\infty} \e^{X}=0$ donc $\lim_{n\to +\infty} \e^{-0.19\times(n - 1)}=0$; on en déduit que $\lim_{n\to +\infty} v_n=9$.

À long terme il y aura $900$ questions pour la 2ème modélisation..
\end{list}

C'est donc pour la $1ere$ modélisation qu'il y aura le plus de questions à long terme.
} % fin correction 

\end{enumerate}

\bigskip



\section{Exercice\_003}

\index[notions]{3eme}
\index[notions]{aire triangle}
\index[notions]{calcul angle}
\index[notions]{droites parallèles}
\index[notions]{triangle rectangle}
\index[notions]{réciproque de Pythagore}
\index[notions]{Thalès}
\index[notions]{fonctions trigonométriques}
\begin{description}
\item[Date de création : ]20/03/2024
\item[Source : ]annales apmep, Brevet Amérique du Nord 31 mai 2023
\item[Liste des notions : ]3eme : aire triangle : calcul angle : droites parallèles : triangle rectangle : réciproque de Pythagore : Thalès : fonctions trigonométriques : 
\item[Nombre de variantes de l'exercice : ]55
\end{description}

\centerline{\bf\large Barème utilisé pour l'exercice : 10 points / paramètre utilisé : numéro 0}



\EXERCICE 
\SeulementModeEnonce{ \points{10.0}} 

\medskip
\begin{minipage}{0.48\linewidth}
On considère la figure ci-contre. On donne les mesures suivantes:

\begin{itemize}
\item[$\bullet~$] AN = $ 13$ cm
\item[$\bullet~$] LN = $5$ cm
\item[$\bullet~$] AL = $12$ cm
\item[$\bullet~$] ON = $3$  cm
\item[$\bullet~$] O appartient au segment [LN]
\item[$\bullet~$] H appartient au segment [NA]
\end{itemize}
\end{minipage}\hfill
\begin{minipage}{0.48\linewidth}
\definecolor{uuuuuu}{rgb}{0.26666666666666666,0.26666666666666666,0.26666666666666666}
\begin{tikzpicture}[line cap=round,line join=round,>=triangle 45,x=1.0cm,y=1.0cm]
\clip(-2.14,-1.62) rectangle (9.44,10.7);
\draw [line width=2.pt] (6.,0.)-- (0.,9.);
\draw [line width=2.pt] (0.,9.)-- (0.,0.);
\draw [line width=2.pt] (0.,0.)-- (6.,0.);
\draw [line width=2.pt] (3.36,0.)-- (3.36,3.96);
\draw [line width=2.pt] (3.36,0.)-- (3.36,0.36);
\draw [line width=2.pt] (3.36,0.36)-- (3.92,0.34);
\draw [line width=2.pt] (3.92,0.34)-- (3.92,0.);
\begin{scriptsize}
\draw [fill=uuuuuu] (0.,0.) circle (2.0pt);
\draw[color=uuuuuu] (-0.04,-0.35) node {$L$};
\draw [fill=black] (6.,0.) circle (2.5pt);
\draw[color=black] (6.14,0.37) node {$N$};
\draw [fill=black] (0.,9.) circle (2.5pt);
\draw[color=black] (0.14,9.37) node {$A$};
\draw [fill=black] (3.36,0.) circle (2.5pt);
\draw[color=black] (3.32,-0.31) node {$O$};
\draw [fill=uuuuuu] (3.36,3.96) circle (2.0pt);
\draw[color=uuuuuu] (3.5,4.29) node {$H$};
\end{scriptsize}
\end{tikzpicture}
\end{minipage}


\begin{enumerate}
% Question 1 : 
\item 
\enonce{ %début énoncé 

Montrer que le triangle LNA est rectangle en L.
} % fin énoncé 

\correction{ %début énoncé 

$ AN^{2} = 13^{2} =  169$ .

$LN^{2} + AL^{2} = 5^{2} + 12^{2} =25 + 144= 169$

donc $AN^{2} = LN^{2} + AL^{2}$.

D'après la réciproque du théorème de Pythagore, le triangle $LNA$ est bien rectangle en $L$. 
\points{2.0} %en pourcentage
} % fin correction 

% Question 2 : 
\item 
\enonce{ %début énoncé 

Montrer que la longueur OH est égale à $7.25$~cm.
} % fin énoncé 

\correction{ %début énoncé 

D'après la question précédente, $(AL) \perp (LN)$.

D'après le codage de l'énoncé, $(HO) \perp (LN)$.

Donc les droites $(AL)$ et $(HO)$ perpendiculaires à une même droite,
sont parallèles. D'autre part

Les points $N,H,A $ et $N, O, L $ sont alignés.

Les droites $(AL)$ et $(HO)$ sont parallèles.

D'après le théorème de Thalès

\(\displaystyle	\dfrac{NO}{NL}=\dfrac{NH}{NA}=\dfrac{OH}{AL}\) ~~ soit ~~ \(\displaystyle	\dfrac{3}{5}= \dfrac{NH}{13}=\dfrac{OH}{12}\),  d'où ~~\(\displaystyle OH = \dfrac{12 \times 3}{5}  = 7.25~(\text{cm}) \).
\points{2.0} %en pourcentage
} % fin correction 


% Question 3 : 
\item 
\enonce{ %début énoncé 

Calculer la mesure de l'angle $\widehat{\text{LNA}}$. Donner une valeur approchée à l'unité près. 
} % fin énoncé 

\correction{ %début énoncé 

Dans le triangle $ LNA $ rectangle en $ L $, 
\(\displaystyle \cos({\widehat{LNA}})=\dfrac{\text{côté adjacent}}{\text{hypoténuse}}=\dfrac{LN}{AN}= \dfrac{5}{13}\).

La calculatrice donne avec la fonction inverse de la fonction cosinus : $\widehat{LNA} \approx 67^\circ$.
\points{2.0} %en pourcentage
} % fin correction 

% Question 4 : 
\item 
\enonce{ %début énoncé 

Pourquoi les triangles LNA et ONH sont-ils semblables ?
} % fin énoncé 

\correction{ %début énoncé 

L'angle $\widehat{LNA}$ est un angle commun aux deux triangles.
	
$\widehat{HON}=\widehat{ALN}=90 ~ \text{degrés}$.

Donc les triangles $ LNA $ et $ OHN $ ont deux paires d'angles de même mesures, donc ils sont semblables.
\points{1.0} %en pourcentage
} % fin correction 

\item 
\begin{enumerate}
% Question 5 a. : 
\item 
\enonce{ %début énoncé 

Quelle est l'aire du quadrilatère LOHA ?
} % fin énoncé 

\correction{ %début énoncé 

 On calcule les différentes aires :

\(\displaystyle A_{LNA}=\dfrac{5\times 12}{2}=30~(\text{cm}^{2})\).

\(\displaystyle A_{OHN}=\dfrac{3\times  7.25}{2}=10.8~~(\text{cm}^{2})\).

\(\displaystyle A_{LOHA}=A_{LNA} - A_{OHN}=19.2~~(\text{cm}^{2})\).
\points{2.0} %en pourcentage
} % fin correction 

\item 
% Question 5 b. : 
\enonce{ %début énoncé 

Quelle proportion de l'aire du triangle LNA représente l'aire du quadrilatère LOHA ?
} % fin énoncé 

\correction{ %début énoncé 

\(\displaystyle \dfrac{A_{LOHA}}{A_{LAN}}=\dfrac{19.2}{30}=0.64=\dfrac{64}{100}\).

La proportion est donc \(\displaystyle \dfrac{64}{100}\).
\points{1.0} %en pourcentage
} % fin correction 

\end{enumerate}
\end{enumerate}



