Le résultat final de la création d'un exercice consiste en la production de deux fichiers : 

\begin{description}
 \item[Fichier Latex : ] $latex\_main\_exercice\_test.tex$
 \item[Fichier python : ] $fonction\_param\_exercice\_test.py$
\end{description}

L'intégration se fait maintenant très simplement : 

\begin{description}
 \item[Etape 1 : ] Il faut ouvrir $./DSXLmodule/list_function_param_exercices.py$ et copier à la fin le contenu de $fonction\_param\_exercice\_test.py$  en prenant soin de modifier le nom de la procedure en lui attribuant un numéro nouveau : 
 \begin{verbatim}
#------------------------------------------------------fonction_param_exercice_004

def fonction_param_exercice_004() : 

    dico_exercice = {
        'nom_exercice': 'exercice_004' ,
        'date_creation': '12/03/2024' ,
        'source': 'Marcus' ,
 \end{verbatim}
\item[Etape 2 : ]A la fin ajouter les lignes qui vont créer une instance de l'exercice (adapter en fonction du numéro de l'exercice) :

\begin{verbatim}
 dico_exercice, liste_liste_parametres = fonction_param_exercice_004()
Ex_004 = Exercice(dico_exercice, liste_liste_parametres)
\end{verbatim}

\item[Etape 3 : ] Copier le fichier $latex\_main\_exercice\_test.tex$, en le renommant $exercice\_004.tex$ (adapter le nom en fonction du numéro de l'exercice), dans le répertoire : $/DSXLmodule/text\_exercices$. 
\end{description}

Voilà! L'exercice est maintenant disponible pour être intégré dans un devoir!

\begin{todo} \index[dsxlindex]{To Do : insertion automatique d'un devoir dans la base de données.}
 L'insertion automatique d'un devoir dans la base de données faite ci-dessus peut être automatisée. C'est une fonctionnalité qu'il faudra à terme implémenter. 
\end{todo}
