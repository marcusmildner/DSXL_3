%DEBUT EnTete
\documentclass[10pt,a4paper]{article}
\usepackage[T1]{fontenc}
\usepackage{amsmath}
\usepackage{amsfonts}
\usepackage{amssymb}
\usepackage{graphicx}
\usepackage{calc}
\usepackage{pgf,tikz}
%https://stackoverflow.com/questions/67845020/how-to-avoid-latex-error-environment-axis-undefined-when-using-include-tikz : 
\usepackage{pgfplots}
% https://tex.stackexchange.com/questions/81899/what-does-running-in-backwards-compatibility-mode-mean-and-what-should-i-fix-t
\pgfplotsset{compat=1.17}
\usepackage{pstricks-add}
\usetikzlibrary{arrows}
\usepackage{mathrsfs}
\usepackage{ifthen}
\usepackage{pdfpages}
\usepackage[left=2cm,right=2cm,top=2cm,bottom=2cm]{geometry}
\usepackage{datatool}
\everymath{\displaystyle}
\usepackage{fancyhdr} 
\usepackage{hyperref}
\usepackage{calculator}
%\usepackage{sagetex}
%FIN EnTete
%DEBUT MesCommandes

\setlength{\parindent}{0em}



% *********************************************************************
% MODIFIER LA LIGNE SELON LE MODE VOULU : **********************
\newcommand{\EnonceCorrection}{C}	% E = mode énoncé   C = mode correction 
% **********************************************************************



%%%%%%%%%%%%% MetaDonnees.tex %%%%%%%%%%%%%%%%%%%%%
\newcommand{\AfficheMetaAuteur}{
% Pour imprimer les métadonnées (à effacer) : 
{\bf Auteur : \AuteurEx    \hfill Nom de l'exercice : \varnom  \hfill  Version :   \VersionEx }

{\bf Date de création : \DateCreationEx \hfill Date de modification : \DateModificationEx }

{\bf Source de l'exercice : \SourceEx }

{\bf Description : \DescriptionEx}

\hrulefill

}
%%%%%%%%%%%%% TitreDuTexte %%%%%%%%%%%%%%%%%%%%%
% #1 :  niveau                   #2 : Titre
\newcommand{\TitreDuTexte}{
{\bf \varnom \large   \hfill \NiveauEx Mathématiques \hfill   $ e^{i\pi}+1=0$ }

\bigskip
\centerline{\bf \large \NomEx}
\bigskip
}

%%%%% Quelques commandes personnalisées %%%%%%%%%%%%%%%%%%
\newcommand{\euro}{\eurologo{}} 
\newcommand{\e}{\mbox{e}}
\newcommand{\R}{\mathbb{R}}
\newcommand{\N}{\mathbb{N}}
\newcommand{\D}{\mathbb{D}}
\newcommand{\Z}{\mathbb{Z}}
\newcommand{\Q}{\mathbb{Q}}
%\newcommand{\C}{\mathbb{C}}
\renewcommand{\mathbf}{}
\renewcommand{\vec}[1]{\overrightarrow{#1}}

%============================================================
% POUR L'ENONCE : Le paramètre est l'énoncé \enonce{Ceci est l'énoncé}
\newcommand{\enonce}[1]{
% \EnonceCorrection = E (énoncé seulement) ou C (énoncé et correction)
\ifthenelse{\equal{\EnonceCorrection}{E}}{#1}{ {\bf \it #1}}{} }

% POUR LA CORRECTION : Le paramètre est la solution \correction{Ceci est la correction}
\newcommand{\correction}[1]{
% \EnonceCorrection = E (énoncé seulement) ou C (énoncé et correction)
\ifthenelse{\equal{\EnonceCorrection}{C}}{ \ \ \newline 
 {\bf SOLUTION : } #1}{}
}

% VARIABLE points par exercices 
%\newcounter{pointsexercice}
% Cette variable se remets à 0 à chaque début d'exercice par la commande : \setcounter{pointexercice}{0}
%\COPY{0}{\pointexercice}

% VARIABLE qui contient le total des points 
\newcounter{pointssujetTOTAL}
\setcounter{pointssujetTOTAL}{0}
\COPY{0}{\pointssujetTOTAL}

% VARIABLE qui contient le numéro de l'exercice : 
\newcounter{numeroexercice}

\newcommand{\Affichepointsexercice}{
% \EnonceCorrection = E (énoncé seulement) ou C (énoncé et correction)
\ifthenelse{\equal{\EnonceCorrection}{C}}{  \ \ \hfill {\bf  Cet exercice comporte  $\pointexercice$  points }}{}

}

\newcommand{\AffichepointssujetTOTAL}{
% \EnonceCorrection = E (énoncé seulement) ou C (énoncé et correction)
\ifthenelse{\equal{\EnonceCorrection}{E}}{ \ \ \newline  {\bf Le sujet compte un total de $\pointssujetTOTAL$ \ points.} }{}
}

% \newcommand{\points}[1]{\setcounter{pointsexercice}{\thepointsexercice + #1} {\bf [#1 point(s)]}} 
\newcommand{\points}[1]{\ADD{#1}{\pointexercice}{\pointexercice} \ADD{#1}{\pointssujetTOTAL}{\pointssujetTOTAL}  {\bf [#1 point(s)]}}

\newcommand{\poids}[1]{ {\bf [poids = #1 \%]}}

\newcommand{\EXERCICE}{
% On ajoute 1 au numéro de l'exercice :
\setcounter{numeroexercice}{\thenumeroexercice + 1}
% On remet à zéro les points de l'exercice 
\COPY{0}{\pointexercice}

\bigskip
{\bf EXERCICE \thenumeroexercice \ (Tous les résultats doivent être justifiés)} 

}
% Cette commande insère une nouvelle page en mode ENONCE 
\newcommand{\NouvellePageModeEnonce}{
% \EnonceCorrection = E (énoncé seulement) ou C (énoncé et correction)
\ifthenelse{\equal{\EnonceCorrection}{E}}{{\ } \newpage }{} }

% Cette commande insère une nouvelle page en mode CORRECTION 
\newcommand{\NouvellePageModeCorrection}{
% \EnonceCorrection = E (énoncé seulement) ou C (énoncé et correction)
\ifthenelse{\equal{\EnonceCorrection}{C}}{{\ }  \newpage }{} }


% Cette commande seulement mode ENONCE 
\newcommand{\SeulementModeEnonce}[1]{
% \EnonceCorrection = E (énoncé seulement) ou C (énoncé et correction)
\ifthenelse{\equal{\EnonceCorrection}{E}}{ #1 }{} }

% Cette commande seulement en mode CORRECTION 
\newcommand{\SeulementModeCorrection}[1]{
% \EnonceCorrection = E (énoncé seulement) ou C (énoncé et correction)
\ifthenelse{\equal{\EnonceCorrection}{C}}{ #1 }{} }

% Commande balise pour définir une variable : \var{p}{10} affichera juste 10, mais 
% le programme reconnaitra la variable p 
\newcommand{\var}[2]{\fbox{$#2$}_{\fbox{\mbox{\bf #1}}}}


%FIN MesCommandes
%DEBUT MetaAuteur
\newcommand{\AuteurEx}{} 
\newcommand{\VersionEx}{} 
\newcommand{\DateCreationEx}{} 
\newcommand{\DateModificationEx}{} 
\newcommand{\SourceEx}{} 
\newcommand{\NotionsEx}{} 
\newcommand{\NiveauEx}{} 
\newcommand{\NomEx}{}
%FIN MetaAuteur
